\documentclass{article}

\usepackage{amsmath}
\usepackage{graphicx}
\usepackage{ctable}
\usepackage{longtable}

% Define bold face typewriter font
\DeclareFontShape{OT1}{cmtt}{bx}{n}{
  <5><6><7><8><9><10><10.95><12><14.4><17.28><20.74><24.88>cmttb10}{}


\title{Sterile neutrinos and non-standard neutrino interactions in GLoBES}
\author{Joachim Kopp}
\date{v1.0 (Nov 2010)}

\begin{document}

\maketitle

Sterile neutrinos and non-standard neutrino interactions (NSI) are two of
the most generic types of new physics possible in the neutrino sector.
For an introduction to the formalism of NSI, see e.g.\ \cite{Kopp:2007ne}.

There are several implementations of NSI in GLoBES (e.g.\ the MonteCubes plugin
\cite{Blennow:2009pk}, for instance, can handle NSI), but to the best of
my knowledge, there is no publicly available implementation of sterile neutrinos.
The code discussed here can handle both: It works with up to 9 neutrino flavors,
and can handle NSI in the production and detection processes as well as
non-standard matter effects (NSI in propagation). The non-standard operators
can affect the active neutrino species as well as the sterile ones.

To use the code, include {\tt snu.c} in your project by modifying your
{\tt Makefile} accordingly, and {\tt \#include} the header file {\tt snu.h}
in your source code.

The first step is to initialize the sterile neutrino/NSI engine. If you only
need NSI, but no sterile neutrinos, you may use the simple command
\begin{verbatim}
  snu_init_probability_engine();
\end{verbatim}
On the other hand, if you need sterile neutrinos, you need to specify how
many neutrino species you would like to include, and how the mixing matrix
is defined. The syntax for this is
\begin{verbatim}
  int snu_init_probability_engine(int n_flavors,
    int rotation_order\left[2], int phase_order[]);
\end{verbatim}
\noindent The arguments are
\begin{longtable}{p{2.5cm}p{8.5cm}}
  \toprule
  {\tt n\_flavors}      & The number of neutrino flavors. Must be $\geq 3$.
                          Note that for ${\tt n\_flavors} = 3$, the optimized
                          algorithms from \cite{Kopp:2006wp} are used to
                          diagonalize the Hamiltonian, while for more than four
                          flavors, the GNU Scientific Library is used. \\\midrule
  {\tt rotation\_order}, {\tt phase\_order} &
    Specifies how the neutrino mixing matrix $U$ is composed of individual rotation
    matrices, and which of the rotation matrices carry the complex phases. A
    definition of the form
    \begin{flushleft}
      {\tt rotation\_order} = \{\{$i_1$, $j_1$\}, \{$i_2$, $j_2$\}, \dots \} \\
      {\tt phase\_order} = $\{k_1, k_2, \dots\}$
    \end{flushleft}
    means $U = R(\theta_{i_1 j_1},
    \delta_{k_1}) R(\theta_{i_2 j_2}, \delta_{k_2}) \cdots$, where
    \begin{displaymath}
      R(\theta_{ij},\delta_k) =
        \begin{pmatrix}
          0      & \cdots &   0             & \cdots &   0                    & \cdots & 0 \\
          \vdots &        & \vdots          &        & \vdots                 &        & \vdots \\
          0      & \cdots & \cos\theta_{ij} & \cdots & e^{i\delta_k} \sin\theta_{ij} & \cdots & 0 \\
          \vdots &        & \vdots          &        & \vdots                 &        & \vdots \\
          0      & \cdots & -e^{-i\delta_k} \sin\theta_{ij} & \cdots & \cos\theta_{ij} & \cdots & 0 \\
          \vdots &        & \vdots          &        & \vdots                 &        & \vdots \\
          0      & \cdots &   0             & \cdots &   0                    & \cdots & 0 \\
        \end{pmatrix}
    \end{displaymath}
    is a rotation matrix in the $ij$-sector with complex phase $\delta_k$. The
    indices $i$ and $j$ run from $1$ to {\tt n\_flavors}. Both {\tt
    rotation\_order} and {\tt phase\_order} should have ${\tt n\_flavors}
    ({\tt n\_flavors}-1)/2$ entries. Since there are only
    $({\tt n\_flavors}-1)({\tt n\_flavors}-2)/2$ phases, some of the entries
    of {\tt phase\_order} should be $-1$, indicating that the respective
    rotation matrices will be treated as real. In the above syntax, the
    standard three-flavor mixing matrix $U_{3\times3} = R(\theta_{23},0)
    R(\theta_{13},\delta) R(\theta_{12},0)$ would be given by
    \begin{flushleft}
      {\tt rotation\_order} = \{\{2, 3\}, \{1, 3\}, \{1, 2\} \} \\
      {\tt phase\_order} = \{-1, 0, -1\}\,.
    \end{flushleft} \\
  \bottomrule
\end{longtable}
After initializing the sterile neutrino/NSI engine, you have to make it known
to GLoBES by calling
\begin{verbatim}
  glbRegisterProbabilityEngine(6*SQR(n_flavors) - n_flavors,
                               &snu_probability_matrix,
                               &snu_set_oscillation_parameters,
                               &snu_get_oscillation_parameters,
                               NULL);
\end{verbatim}
Note the expression for the number of oscillation parameters in the first line.
As with any new probability engine, the call to {\tt
glbRegisterProbabilityEngine} has to occur before any calls to {\tt
glbAllocParams} or {\tt glbAllocProjection} to make sure that all parameter and
projection vectors have the correct length.

It is recommended to use {\tt glbSetParamName} to assign human-readable
names to the oscillation parameters by which they can be referred to in
subsequent calls to {\tt glbSetOscParamByName} and {\tt
glbSetProjectionFlagByName}. A list of suitable names is provided in the global
array {\tt char snu\_param\_strings[][64]}. The syntax is {\tt TH12},
{\tt TH13}, {\tt TH14}, etc. for the mixing angles, {\tt DELTA\_0},
{\tt DELTA\_1}, \dots for the CP phases, {\tt ABS\_EPS\_$c$\_$\alpha\beta$}
for the absolute values of the non-standard parameters, and
{\tt ARG\_EPS\_$c$\_$\alpha\beta$} for their phases. Here, $c = {\tt S},
{\tt M}, {\tt D}$ to identify NSI in the source, in propagation (non-standard
matter effects), and in detection. $\alpha$ and $\beta$ are neutrino flavors,
where the standard flavors are denoted by {\tt E}, {\tt MU}, {\tt TAU},
and the sterile flavors are {\tt S1}, {\tt S2}, {\tt S3}, \dots. For
example, {\tt ABS\_EPS\_M\_ETAU} refers to $|\varepsilon^m_{e\tau}|$
in the notation of \cite{Kopp:2007ne}.

Having initialized and registered the non-standard probability engine, you can
use all GLoBES functions in the same way as you would for standard
oscillations.  Note, however that at the moment the initial and final flavors
in the definition of an oscillation channel can only involve the three active
flavors. It is for instance not possible to compute sterile neutrino
appearance.

After using the sterile neutrino/NSI engine, you may want to release the
(small) amount of memory allocated by it by calling
\begin{verbatim}
  snu_free_probability_engine();
\end{verbatim}

\bibliographystyle{plain}
\bibliography{snu}


\end{document}

