\documentclass{article}

\usepackage{amsmath}
\usepackage{graphicx}
\usepackage{ctable}
\usepackage{longtable}

% Define bold face typewriter font
\DeclareFontShape{OT1}{cmtt}{bx}{n}{
  <5><6><7><8><9><10><10.95><12><14.4><17.28><20.74><24.88>cmttb10}{}


\title{{\tt degfinder} --- Event rate based degeneracy finding in GLoBES}
\author{Joachim Kopp}
\date{v1.0 (Nov 2010)}

\begin{document}

\maketitle

\begin{center}
\framebox{
\begin{minipage}{11cm}
{\bf IMPORTANT NOTICE:} It is important to keep in mind that degeneracy finding is
not a foolproof process. The performance of {\tt degfinder} depends critically
on the way it is configured by the user. For example, it is the user's
responsibility to determine which parameters need to be scanned over in
the pre-scan, which parameter correlations need to be taken into account,
etc. If a mistake is made, results may be too optimistic, and it is often
hard to notice when this happens. If your plots show
funny features (kinks, spikes, tiny islands of good sensitivity), this
is often an indication that degeneracy finding has gone wrong.
Once again, {\bf \texttt{\textbf{degfinder}} MUST NOT BE USED AS A BLACK BOX!}
\end{minipage}}
\end{center}

\section{Introduction}

In a fit to data from a high-precision neutrinon oscillation experiment, one
often has to deal with the problem of multiple degenerate solutions in the
high-dimensional space of oscillation parameters. For example, the two cases
$\Delta m_{31}^2 > 0$ (normal mass hierarchy) and $\Delta m_{31}^2 < 0$ (inverted
mass hierarchy) often cannot be distinguished
(``mass hierarchy degneracy''). The fitting algorithm in GLoBES is based on 
\emph{local} minimization of $\chi^2$ in the space of oscillation parameters
and systematic nuisance parameters. This local minimization can only find
one degenerate solution at a time. For example, if a $\Delta m_{31}^2 > 0$
is used as a starting value (the {\tt in} parameters passed to {\tt glbChiNP}
etc.), the minimizer will usually converge into a normal hierarchy solution
and ignore the inverted hierarchy solution. This may lead to results that
are too optimistic, and hence it is mandatory to address the problem of
degenerate solutions carefully.

This is the aim of the {\tt degfinder} add-pon. The function {\tt degfinder}
first performs a rough scan (the so-called \emph{pre-scan}) of part of the
parameter space during which many features like systematic uncertainties
and parameter correlations are switched off to improve speed. Which part of
the parameter space is scanned over is determined by the user. The pre-scan
provides rough estimates for the locations of the local $\chi^2$ minima,
and each of these is then used as a starting points for a full-fledged
minimization including all features.


\section{Installation}

To use degfinder in your GLoBES application, simply {\tt \#include} the header file
{\tt degfinder.h} in your source code and add the source file {\tt degfinder.c} to
your {\tt Makefile} so that it is compiled and linked to your executable.

Note that {\tt degfinder} is written in the GNU version of C99. To compile it
with {\tt gcc}, use the option {\tt -std=gnu99}. I haven't tested {\tt
degfinder} with other compilers, so I cannot exclude that some tweaking may be
necessary to compile it with them.

\section{Usage}

The syntax of {\tt degfinder} is

\begin{verbatim}
int degfinder(const glb_params base_values,
  const int n_prescan_params, const int *prescan_params,
  const double *prescan_min, const double *prescan_max,
  const int *prescan_steps, const glb_projection prescan_proj,
  const glb_projection fit_proj, int *n_deg, glb_params *deg_pos,
  double *deg_chi2, const long flags);
\end{verbatim}

\noindent The arguments are \\
\begin{longtable}{p{3cm}p{9cm}}
  \toprule
  {\tt base\_values} & Vector of oscillation parameters.
                       Parameters that are not scanned in the pre-scan (i.e.\ parameters
                       that are not included in {\tt prescan\_params}) are treated in
                       the same way as {\tt in} parameters passed to {\tt glbChiNP}
                       and similar function. This means that if a parameter is declared as
                       {\tt GLB\_FIXED} in {\tt prescan\_proj} or {\tt fit\_proj}, it is
                       assumed to be known with a zero uncertainty in the pre-scan or
                       during the final fit, respectively. The value of a parameter that
                       is declared as {\tt GLB\_FREE} in {\tt prescan\_proj} or {\tt fit\_proj}
                       is used as starting values for local minimizations of $\chi^2$ with
                       respect to that parameter during the pre-scan or during the final
                       minimization, respectively. The {\tt base\_values} of parameters included in
                       {\tt prescan\_params} are ignored. \\\midrule
  {\tt n\_prescan\_params} & The number of parameters that should be scanned over during
                             the pre-scan. \\\midrule
  {\tt prescan\_params}    & A list of parameters to be scanned over during the pre-scan.
                             Parameters are referred to by their numerical index, e.g.\
                             {\tt GLB\_THETA\_12}, {\tt GLB\_THETA\_13}, etc. There should
                             be {\tt n\_prescan\_params} entries in {\tt prescan\_params}. \\\midrule
  {\tt prescan\_min}, {\tt prescan\_max}, {\tt prescan\_steps}
                           & Parameter ranges for the pre-scan. For each entry of
                             {\tt prescan\_params}, a minimum value, a maximum value, and
                             a number of steps is given. The number of points sampled for
                             each parameter is the number of steps plus one. Note that all
                             parameters except $\theta_{13}$ are scanned on a linear scale,
                             while $\theta_{13}$ is scanned on a logarithmic scale in
                             $\sin^2 2\theta_{13}$. For example, if {\tt prescan\_min = -3},
                             {\tt prescan\_max = -1}, $\sin^2 2\theta_{13}$ will be varied
                             between $10^{-3}$ and $10^{-1}$, using {\tt prescan\_steps+1}
                             logarithmically spaced sampling points. \\\midrule
  {\tt prescan\_proj}      & The projection to be used in the pre-scan. Typically, most or
                             all entries will be declared as {\tt GLB\_FIXED} to make
                             the pre-scan as efficient as possible. Sometimes, however,
                             it may be necessary to declare one or several parameters as
                             {\tt GLB\_FREE} to tell {\tt degfinder} to do a local minimization
                             over that parameter even during the pre-scan. \\\midrule
  {\tt fit\_proj}          & The projection to be used in the final fit. Typically, most
                             parameters will be declared as {\tt GLB\_FREE} here. \\\midrule
  {\tt n\_deg}             & Input: Pointer to an integer giving the maximum number of degenerate
                             solutions to accept (the length of the vectors {\tt deg\_pos}
                             and {\tt deg\_chi}). \newline
                             Output: The number of degenerate solutions actually found. \\\midrule
  {\tt deg\_pos}           & An array of pointers to {\tt glb\_params} structures that will be
                             filled with the locations of the degenerate solutions in the space
                             of oscillation parameters. \\\midrule
  {\tt deg\_chi}           & This array will be filled with the $\chi^2$ values of the
                             degenerate solutions. \\\midrule
  {\tt flags}              & Some flags that control the behavior of {\tt degfinder}. \newline
                             \begin{tabular}{lp{6.3cm}}
                               {\tt DEG\_NO\_NH}   & Omit normal hierarchy solutions. \\
                               {\tt DEG\_NO\_IH}   & Omit inverted hierarchy solutions. \\
                               {\tt DEG\_NO\_SYS}  & Switch off systematics even during the final fit
                                                     (systematics are always off during the pre-scan). \\
                               {\tt DEG\_NO\_CORR} & Switch off parameter correlations. Equivalent to
                                                     declaring all parameters as {\tt GLB\_FIXED} in both
                                                     {\tt prescan\_proj} and {\tt fit\_proj}. \\
                               {\tt DEG\_NO\_DEG}  & Switch off degeneracies. Equivalent to
                                                     setting {\tt n\_prescan\_params = 0}. Note that in
                                                     this case, {\tt degfinder} should behave just like
                                                     {\tt glbChiNP}. \\
                             \end{tabular} \\
  \bottomrule
\end{longtable}


\section{{\tt degfinder} and non-standard interactions}

{\tt degfinder} works together with my non-standard interaction code. If you want to
use this feature, include {\tt nsi\_probability.h} in the header of {\tt degfinder.c}. If non-standard
parameters are included in the pre-scan, {\tt degfinder} will step through the non-standard phases
on a linear scale, while the absolute values of the non-standard parameters are scanned
on a log scale. Consequently, for the absolute values of the non-standard paremeters,
{\tt prescan\_min} and {\tt prescan\_max} should be logarithmic. For example, a range
$[-3, -1]$ means that this parameter will be varied between $10^{-3}$ and $10^{-1}$,
with logarithmically spaced sampling points. Note that degeneracy finding in high
dimensional NSI parameter spaces can be extremely inefficient.

\end{document}

