\documentclass{article}
\usepackage{verbatim}
\newcommand{\glb}{{\sf GLoBES}}
\begin{document}
%%%%%%%%%%%%%%%%%%%%%%%%%%%%%%%%%%%%%%%%%%%%%%%%%
\section{Purpose}
This document describes how to install \glb\ in an adverse
environment, ie. as user without root privileges. This 
situation may either arise if one just wants to quickly 
evaluate wether \glb\ is suitable for a certain task or
when the system administration proves to be un-cooperrative.

%%%%%%%%%%%%%%%%%%%%%%%%%%%%%%%%%%%%%%%%%%%%%%%%%%
\section{Prerequisites}

The dependencies of \glb\ have been reduced as far as 
possible and the only remaining hard prerequisite is the
GNU Scientific library, which itself is useful for many
projects and thus should be installed at any rate. However,
there are occasions where even this minimal requirement 
cannot be met. The simplest solution in these cases is, to
download the tar-ball from {\tt www.gnu.org/software/gsl/}
to unpack it, to change into the GSL directory and to run
\begin{quote}
\begin{verbatim}
./configure --prefix=path_to_gsl --disable-shared
make
make install
\end{verbatim}
\end{quote}
where \verb^path_to_gsl^ points to a location where the user
has write permission.

%%%%%%%%%%%%%%%%%%%%%%%%%%%%%%%%%%%%%%%%%%%%%%%%%%%%%
\section{Building \& Installing \glb}

Once GSL is made available the next step is to download
the \glb\ tar-ball, unpack it and to change into the \glb\
directory. There the following commands have to be executed
\begin{quote}
\begin{verbatim}
./configure --prefix=path_to_globes --with-gsl-prefix=path_to_gsl
make
make install
\end{verbatim}
\end{quote}
Now you can change into the \verb^examples^ subdirectory
and build the first example with
\begin{quote}
\begin{verbatim}
make
\end{verbatim}
\end{quote}
typing
\begin{quote}
\begin{verbatim}
./example-tour
\end{verbatim}
\end{quote}
shoud produce something like
\begin{quote}
{\small
\begin{verbatim}
Oscillation probabilities in vacuum: 1->1: 0.999955 1->2: 2.59018e-05 1->3: 1.92391e-05
Oscillation probabilities in matter: 1->1: 0.999965 1->2: 2.01594e-05 1->3: 1.49783e-05
...
\end{verbatim}
}
\end{quote}
in which case you have succeded to install \glb.

%%%%%%%%%%%%%%%%%%%%%%%%%%%%%%%%%%%%%%%%%%%%%%%%%%%%%%
\section{Using {\tt libglobes}}

The \verb^examples^ subdirectory does not only contain example C-code, but also a \verb^Makefile^ which can be used as a basis for building own projects. You just have to replace the
source and object files by the ones you are going to use. Moreover there is a script
called \verb^globes-config^ installed which can be used to find the necessary linker
flags by
\begin{quote}
\begin{verbatim}
globes-config --libs
\end{verbatim}
\end{quote}


%%%%%%%%%%%%%%%%%%%%%%%%%%%%%%%%%%%%%%%%%%%%%%%%%%%%%%
\section{Building static binaries}

Especially for CPU intensive task having the possibility to use many CPUs is desirable. This is much easier if one can buil static binaries, i.e. binaries which can be compiled on
one machine providing a reasonable build environment and at
the same can be run on another machine 
(or many other machines). Again the \verb^Makefile^ in the
\verb^examples^ subdirectory contains the necessary information. Change into \verb^examples^ and type
\begin{quote}
\begin{verbatim}
make example-static
\end{verbatim}
\end{quote}
The resulting binary \verb^example-static^ should run on most Linuxes and the corresponding parts of the \verb^Makefile^ should be easily adaptible to your problem. Note, that this approach uses \verb^libtool^ to do the linking. If \verb^libtool^ should not be available on your system, one can use the
\verb^libtool^ script in the \glb\ build directory as replacement. Just make sure it is in the path of the invocation of
\verb^make^.

All these options rely on \verb^libtool^ and a working gcc installation. It seems that \verb^gcc 3.x^
is broken in subtle way which makes it necessary to add a symbolic link in the gcc library
directory. The diagnostic for this requirement is that building static programs fails with
the error message \verb^cannot find -lgcc_s^. In those cases, find \verb^libgcc.a^ and add a symbolic 
link in the same directory where you found it (this requires probably root privileges)
\begin{quote}
\begin{verbatim}
ln -s libgcc.a libgcc_s.a
\end{verbatim}
\end{quote}

If you can not write to this directory just use the following work around. Add the same link  as above to 
the directory where you installed \verb^globes^ into
\begin{quote}
\begin{verbatim}
cd prefix/lib       
ln -s path_to_libgcc.a/libgcc.a libgcc_s.a
\end{verbatim}
\end{quote}
and then change back into the \verb^examples^ directory and type
 \begin{quote}
\begin{verbatim}
make LDFLAGS=-Lprefix/lib example-static
\end{verbatim}
\end{quote}
and you are done.

 
\end{document}
