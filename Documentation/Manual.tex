\documentclass[a4paper,12pt,twoside]{book}

\sloppy
% \frenchspacing
\setlength{\unitlength}{1cm}

\usepackage{cite}
\usepackage{graphicx}
\usepackage{amsmath}
\usepackage{amssymb}
\usepackage{amsfonts}
\usepackage{mathrsfs}
\usepackage{a4wide}
\usepackage{fancyheadings}
\usepackage[footnotesize,bf]{caption}
\usepackage{ae,aecompl} % only for Type 1 fonts


\newcommand{\bi}{\begin{itemize}}
\newcommand{\ei}{\end{itemize}}
\newcommand{\ra}{$\rightarrow$}
\newcommand{\be}{\begin{equation}}
\newcommand{\ee}{\end{equation}}
\newcommand{\bea}{\begin{eqnarray}}
\newcommand{\eea}{\end{eqnarray}}
\newcommand{\nn}{\nonumber}
\newcommand{\ldm}{$\Delta m_{31}^2$}
\newcommand{\sdm}{$\Delta m_{21}^2$}
\newcommand{\deltacp}{$\delta_{\mathrm{CP}}$}
\newcommand{\stheta}{$\sin^2 2 \theta_{13}$}

\newcommand{\ie}{{\it i.e.}}
\newcommand{\Ie}{{\it I.e.}}
\newcommand{\eg}{{\it e.g.}}
\newcommand{\Eg}{{\it E.g.}}
\newcommand{\cf}{{\it cf.}}
\newcommand{\etc}{{\it etc.}}
\newcommand{\eq}{Eq.}
\newcommand{\eqs}{Eqs.}
\newcommand{\Def}{Definition}
\newcommand{\fig}{Fig.}
\newcommand{\Fig}{Fig.}
\newcommand{\figs}{Figs.}
\newcommand{\Figs}{Figs.}
\newcommand{\Ref}{Ref.}
\newcommand{\Refs}{Refs.}
\newcommand{\Sec}{Sec.}
\newcommand{\Secs}{Secs.}
\newcommand{\App}{the Appendix}
\newcommand{\Apps}{Appendices}
\newcommand{\Tab}{Table}
\newcommand{\Tabs}{Tables}

\newcommand{\JHFSK}{{\sc JHF-SK}}
\newcommand{\NUMI }{{\sc NuMI}}
\newcommand{\ReactorI}{{\sc Reactor-I}}
\newcommand{\ReactorII}{{\sc Reactor-II}}
\newcommand{\JHFHK}{{\sc JHF-HK}}
\newcommand{\NuFactI}{{\sc NuFact-I}}
\newcommand{\NuFactII}{{\sc NuFact-II}}

\newcommand{\GLOBES}{{\sf GLoBES}}
\newcommand{\EDM}{{\sf EDM}}

\newcommand{\equ}[1]{\eq~(\ref{equ:#1})}
\newcommand{\figu}[1]{\fig~\ref{fig:#1}}

%-- page parameters -------------------------------------------------

\pagestyle{fancyplain}

% Olddefs:

%\addtolength{\textwidth}{-1truecm}
%\addtolength{\oddsidemargin}{1.0truecm}
%\addtolength{\evensidemargin}{-0.3truecm}

% Owndefs:

%\addtolength{\textwidth}{0.0truecm}
%\addtolength{\textheight}{1.8truecm}
%\addtolength{\topmargin}{-0.6truecm}
%\addtolength{\oddsidemargin}{0.65truecm}
%\addtolength{\evensidemargin}{-1.0truecm}

% Headdefs:

\advance \headheight by 3.0truept       % for 12pt mandatory...
\lhead[\fancyplain{}{\thepage}]{\fancyplain{}{\rightmark}}
\rhead[\fancyplain{}{\leftmark}]{\fancyplain{\thepage}{\thepage}}
\cfoot{}

\renewcommand{\chaptermark}[1]{
% chapter title im Seitenkopf
\markboth{\uppercase{\chaptername}\ \thechapter.\ \ #1}
          {\uppercase{\chaptername}\ \thechapter.\ \ #1}}
% section title im Seitenkopf
\renewcommand{\sectionmark}[1]{\markright{\thesection\ \ #1}}




\begin{document} 

\pagenumbering{Roman}
\thispagestyle{empty}
{
\setlength{\parindent}{0cm}

%\begin{titlepage}

\font\fa=cmssbx14 scaled 1440
\font\fb=cmbx10 scaled 1200
\font\fc=cmb10 scaled 1200
\font\fd=cmr10 scaled 1000  

{\setlength{\baselineskip}{1.2cm}}

% This is tumlogo.tex
%
% Neues TUM-Logo in TeX
%   by G. Teege, 19.10.89
% Benutzung:
%   Am Anfang des Dokuments (TeX oder LaTeX):
%     \input tumlogo
%   Dann beliebig oft:
%     \TUM{<breite>}
%   bzw.
%     \oTUM{<breite>}
%   \TUM setzt das Logo mit der Breite <breite> und der entsprechenden Hoehe.
%   <breite> muss eine <dimen> sein. \oTUM erzeugt eine "outline"-Version
%   des Logos, d.h. weiss mit schwarzem Rand. Bei \TUM ist es ganz schwarz.
%   \oTUM entspricht damit der offiziellen Version des Logos.
%   Das Logo kann wie ein einzelnes Zeichen verwendet werden.
%   Beispiel:
%     Dies ist das TUM-Logo: \oTUM{1cm}.
%
\def\TUM#1{%
\dimen1=#1\dimen1=.1143\dimen1%
\dimen2=#1\dimen2=.419\dimen2%
\dimen3=#1\dimen3=.0857\dimen3%
\dimen4=\dimen1\advance\dimen4 by\dimen2%
\setbox0=\vbox{\hrule width\dimen3 height\dimen1 depth0pt\vskip\dimen2}%
\setbox1=\vbox{\hrule width\dimen1 height\dimen4 depth0pt}%
\setbox2=\vbox{\hrule width\dimen3 height\dimen1 depth0pt}%
\setbox3=\hbox{\copy0\copy1\copy0\copy1\box2\copy1\copy0\copy1\box0\box1}%
\leavevmode\vbox{\box3}}
%
\def\oTUM#1{%
\dimen1=#1\dimen1=.1143\dimen1%
\dimen2=#1\dimen2=.419\dimen2%
\dimen3=#1\dimen3=.0857\dimen3%
\dimen0=#1\dimen0=.018\dimen0%
\dimen4=\dimen1\advance\dimen4 by-\dimen0%
\setbox1=\vbox{\hrule width\dimen0 height\dimen4 depth0pt}%
\advance\dimen4 by\dimen2%
\setbox8=\vbox{\hrule width\dimen0 height\dimen4 depth0pt}%
\advance\dimen4 by-\dimen2\advance\dimen4 by-\dimen0%
\setbox4=\vbox{\hrule width\dimen4 height\dimen0 depth0pt}%
\advance\dimen4 by\dimen1\advance\dimen4 by\dimen3%
\setbox6=\vbox{\hrule width\dimen4 height\dimen0 depth0pt}%
\advance\dimen4 by\dimen3\advance\dimen4 by\dimen0%
\setbox9=\vbox{\hrule width\dimen4 height\dimen0 depth0pt}%
\advance\dimen4 by\dimen1%
\setbox7=\vbox{\hrule width\dimen4 height\dimen0 depth0pt}%
\dimen4=\dimen3%
\setbox5=\vbox{\hrule width\dimen4 height\dimen0 depth0pt}%
\advance\dimen4 by-\dimen0%
\setbox2=\vbox{\hrule width\dimen4 height\dimen0 depth0pt}%
\dimen4=\dimen2\advance\dimen4 by\dimen0%
\setbox3=\vbox{\hrule width\dimen0 height\dimen4 depth0pt}%
\setbox0=\vbox{\hbox{\box9\lower\dimen2\copy3\lower\dimen2\copy5%
\lower\dimen2\copy3\box7}\kern-\dimen2\nointerlineskip%
\hbox{\raise\dimen2\box1\raise\dimen2\box2\copy3\copy4\copy3%
\raise\dimen2\copy5\copy3\box6\copy3\raise\dimen2\copy5\copy3\copy4\copy3%
\raise\dimen2\box5\box3\box4\box8}}%
\leavevmode\box0}
% End of tumlogo.tex



\begin{minipage}{7cm}
\begin{center}
{\bf Technische Universit{\"a}t M{\"u}nchen} \\
Physik-Department \\
Institut f{\"u}r Theoretische Physik, T30d \\
\end{center}
\end{minipage}

\vspace{-2cm}

\hfill \oTUM{3.5cm}

\vspace{3cm}

\begin{center}
{ \Large \bf
\GLOBES\ 1.X \\
Global Long Baseline Experiment Simulator \\ }
\vspace*{0.5cm}
{\large \bf User's and reference manual }
\end{center}

\vspace{1cm}

\begin{center}
{\large Patrick Huber, Manfred Lindner, Walter Winter}
\end{center}

\vspace{1cm}

\begin{center}
Here comes our logo!
\end{center}

\vspace{1cm}

\begin{center}
Version from \today
\end{center}

%\end{titlepage}

%\title{}
%\date{}
%\author{}
%\maketitle

}

\clearpage
\thispagestyle{empty}

\cleardoublepage
\setcounter{page}{1}

\chapter*{What is \GLOBES\ good for?}

Here comes a maximum of one page what \GLOBES\ is good for and what main features it has.
\cleardoublepage
\tableofcontents

\cleardoublepage
\setcounter{page}{1}
\pagenumbering{arabic}

\chapter*{How to use this manual}
\addcontentsline{toc}{chapter}{How to use this manual}

%%%%%%%%%%%%%%%%%%%%%%%%%%%%%%%%%%%
% PART I: User's manual
%%%%%%%%%%%%%%%%%%%%%%%%%%%%%%%%%%%

\part{User's manual}

\chapter*{Getting started}



%%%%%%%%%%%%%%%%%%%%%%%%%%%%%%%%%%%
% PART II: Experiment definition module
%%%%%%%%%%%%%%%%%%%%%%%%%%%%%%%%%%%

\part{The Experiment Definition Module (\EDM )}

\chapter{Basics: Source, Oscillation, Detection}

Here comes how a experiment is defined (what is a channel, a rule, how are they treated, ...). Splitting in source, oscillation, and detector, etc.

\chapter{Getting started: Editing existing experiments}

Here comes how to load and modify existing experiments;
also: how does the main screen work and the most important functions at exampled

\chapter{Features of the \EDM }

\section{Source properties}

\section{Oscillation properties}

\section{Detector properties}

\chapter{Creating new experiments}

Most advanced step: How to create a new experiment from scratch.

%%%%%%%%%%%%%%%%%%%%%%%%%%%%%%%%%%%
% PART III: Reference manual
%%%%%%%%%%%%%%%%%%%%%%%%%%%%%%%%%%%

\part{Reference manual}


Here comes a list of all functions and their function in alphabetical order.


%%%%%%%%%%%%%%%%%%%%%%%%%%%%%%%%%%%
% Appendix
%%%%%%%%%%%%%%%%%%%%%%%%%%%%%%%%%%%

\begin{appendix}

\chapter{\GLOBES\ installation}

\chapter{Statistics and \GLOBES }

Here comes a statistics summary (similar to Superbeams appendix):
\bi
\item
 What $\chi^2$ is used
\item
 How channels/rules are treated
\item
 How does the flux-folding work
\item
 ...
\ei

\end{appendix}
\end{document}
