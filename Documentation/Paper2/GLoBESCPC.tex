
%% This `backbone' file can be used to write a paper for
%% Computer Physics Communications using LaTeX.
%% For authors who want to write a computer program description,
%% an example Program Summary is included that only has to be
%% completed and which will give the correct layout in the
%% preprint and the journal.
%% The `elsart' style that is used and more information on this style
%% can be found at the Author Gateway (http://authors.elsevier.com) and follow %% the link to "Guide to publishing with Elsevier".
\documentclass{elsart}

%% This list environment is used for the references in the
%% Program Summary
%%
\newcounter{bla}
\newenvironment{refnummer}{%
\list{[\arabic{bla}]}%
{\usecounter{bla}%
 \setlength{\itemindent}{0pt}%
 \setlength{\topsep}{0pt}%
 \setlength{\itemsep}{0pt}%
 \setlength{\labelsep}{2pt}%
 \setlength{\listparindent}{0pt}%
 \settowidth{\labelwidth}{[9]}%
 \setlength{\leftmargin}{\labelwidth}%
 \addtolength{\leftmargin}{\labelsep}%
 \setlength{\rightmargin}{0pt}}}
 {\endlist}

\begin{document}
\begin{frontmatter}

\title{GLoBES: General Long Baseline Experiment Simulator}

\author[a]{Patrick Huber},
\author[b]{Joachim Kopp},
\author[b]{Manfred Lindner},
\author[c]{Mark Rolinec},
\author[d]{Walter Winter}

\address[a]{University of Wisconsin, Physics Department, 1150 University Av. Madison, WI 53706, USA}
\address[b]{Max-Planck-Institut f\"{u}r Kernphysik, Postfach 10 39 80, D-69029 Heidelberg, Germany}
\address[c]{Technische Universit\"{a}t M\"{u}nchen, Institut f\"{u}r
Theoretische Physik, Physik-Department, James-Franck-Strasse,
D-85748 Garching, Germany}
\address[d]{Universit\"{a}t W\"{u}rzburg, Lehrstuhl f\"{u}r theoretische Physik II, Institut f\"{u}r theoretische Physik und Astrophysik, Am Hubland, D-97074 W\"{u}rzburg, Germany}

\begin{abstract}
GLoBES (General Long Baseline Experiment Simulator) is a flexible software package to simulate
neutrino oscillation long baseline and reactor experiments. On the one hand, it contains a
comprehensive abstract experiment definition language (AEDL), which allows to describe most classes of
long baseline experiments at an abstract level. On the other hand, it provides a C-library to process
the experiment information in order to obtain oscillation probabilities, rate vectors, and
$\Delta_{\chi^{2}}$-values. Currently, GLoBES is available for GNU/Linux. Since the source code is
included, the port to other operating systems is in principle possible.

GLoBES is an open source code that has previously been described in Computer Physics Communications
167~(2005)~195 and in this issue, xx (2007) yyy). The source code and a comprehensive User Manual for
GLoBES v3.0.8 is now available from the CPC Program Library as described in the Program Summary below.
The home of GLobES is http://www.mpi-hd.mpg.de/$\sim$globes/.

\begin{flushleft}

PACS: 14.60.Pq

\end{flushleft}

\begin{keyword}
Neutrino oscillations; Long-baseline experiment; GLoBES;
\end{keyword}

\end{abstract}


\end{frontmatter}

% Computer program descriptions should contain the following
% PROGRAM SUMMARY.

{\bf PROGRAM SUMMARY}
  %Delete as appropriate.

\begin{small}
\noindent
{\em Manuscript Title:} GLoBES: General Long Baseline Experiment Simulator\\
{\em Authors:} Patrick Huber, Joachim Kopp, Manfred Lindner, Mark Rolinec, Walter Winter\\
{\em Program Title:} GLoBES version 3.0.8\\
{\em Journal Reference:}                                      \\
  %Leave blank, supplied by Elsevier.
{\em Catalogue identifier:}                                   \\
  %Leave blank, supplied by Elsevier.
{\em Licensing provisions:} none                                  \\
  %enter "none" if CPC non-profit use license is sufficient.
{\em Programming language:} C                                  \\
{\em Computer:} GLoBES builds and installs on 32bit and 64bit Linux systems.\\
  %Computer(s) for which program has been designed.
{\em Operating system:} 32bit or 64bit Linux                                      \\
  %Operating system(s) for which program has been designed.
{\em RAM:} Typically a few MBs                                         \\
  %RAM in bytes required to execute program with typical data.
{\em Keywords:} Neutrino oscillations, Long-baseline experiment,
GLoBES \\
{\em PACS:} 14.60Pq    \\
{\em Classification:} 11.1  General, High Energy Physics and Computing, 
  11.7 Detector Design and Simulation, 11.10 Accelerators and Particle Beams    \\
  %Classify using CPC Program Library Subject Index, see (
  % http://cpc.cs.qub.ac.uk/subjectIndex/SUBJECT_index.html)
  %e.g. 4.4 Feynman diagrams, 5 Computer Algebra.
{\em External routines/libraries:} GSL - The GNU Scientific Library,
www.gnu.org/software/gsl/\\

{\em Nature of problem:}\\
  Neutrino oscillations are now established as the leading flavor
transition mechanism for neutrinos. In a long history of many experiments, see, e.g., [1], two
oscillation frequencies have been identified: The fast atmospheric and the slow solar
oscillations, which are driven by the respective mass squared differences. In addition, there could be
interference effects between these two oscillations, provided that the coupling given by the small
mixing angle $\theta_{13}$ is large enough. Such interference effects include, for example, leptonic
CP violation. In order to test the unknown oscillation parameters, i.e. the mixing angle
$\theta_{13}$, the leptonic CP phase, and the neutrino mass hierarchy, new long-baseline and reactor
experiments are proposed. These experiments send an artificial neutrino beam to a detector, or detect
the neutrinos produced by a nuclear fission reactor. However, the presence of multiple solutions which
are intrinsic to neutrino oscillation probabilities [2-5] affect these measurements. Thus optimization
strategies are required which maximally exploit complementarity between experiments. Therefore, a
modern, complete experiment simulation and analysis tool does not only need to have a highly accurate
beam and detector simulation, but also powerful means to analyze correlations and degeneracies,
especially for the combination of several experiments. The GLoBES software package is such a
tool~[6-7].
   \\
{\em Solution method:}\\
   GLoBES is a flexible software tool to simulate and analyze
neutrino oscillation long-baseline and reactor experiments using a
complete three-flavor description. On the one hand, it contains a
comprehensive abstract experiment definition language (AEDL), which
makes it possible to describe most classes of long baseline and
reactor experiments at an abstract level. On the other hand, it
provides a C-library to process the experiment information in order
to obtain oscillation probabilities, rate vectors, and
$\Delta {\chi^{2}}$-values. In addition, it provides a binary
program to test experiment definitions very quickly, before they are
used by the application software.
   \\
{\em Restrictions:}\\
  Currently restriced to discrete sets of sources and detectors. For example, the
simulation of an atmospheric neutrino flux is not supported. 
  %Describe any restrictions on the complexity of the problem here.
   \\
{\em Unusual features:}\\
   Clear separation between experiment description and the simulation software.
  %Describe any unusual features of the program/problem here.
   \\
{\em Additional comments:}\\
   To find information on the latest version of the software and user manual, please check the author's
  web site,  http://www.mpi-hd.mpg.de/$\sim$globes
   \\
{\em Running time:}\\
The examples included in the distribution take only a few minutes to
complete. More sophisticated problems can take up to several days.
  %Give an indication of the typical running time here.
   \\
{\em References:}
\begin{refnummer}
\item V. Barger, D. Marfatia, K. Whisnant, Int. J. Mod. Phys. E 12(2003)569, and references therein,
hep-ph/0308123.
\item G.L. Fogli, E. Lisi, Phys. Rev. D 54 (1996) 3667, hepph/
9604415.
\item J. Burguet-Castell, M.B. Gavela, J.J. Gomez-Cadenas, P. Hernandez,
O. Mena, Nucl. Phys. B 608 (2001) 301, hep-ph/ 0103258.
\item H. Minakata, H. Nunokawa, JHEP 0110 (2001) 001, hep-ph/
0108085.
\item V. Barger, D. Marfatia, K. Whisnant, Phys. Rev. D 65 (2002)
073023, hep-ph/0112119.
\item P. Huber, M. Lindner and W. Winter, Comput. Phys. Commun. 167 (2005) 195.
\item Patrick Huber, Joachim Kopp, Manfred Lindner, Mark Rolinec, Walter Winter, Comput. Phys. Commun. xxx (2007) yyy.
\end{refnummer}

\end{small}

\newpage



\end{document}
