%%%%%%%%%%%%%%%%%%%%%%%%%%%%%%%%%%%%%%%%%%%%%%%%%%%%%%%%%%%%%%%%%%%%%%
\NeedsTeXFormat{LaTeX2e}
\documentclass[12pt,a4paper]{article}

%-- used packages ------------------------------------------------------

\usepackage{amsmath}
\usepackage{amssymb}
\usepackage{epsfig}
\usepackage{graphicx}
\usepackage{cite}
%\usepackage{mcite}
%-- page parameters -------------------------------------------------

\jot = 1.5ex
\parskip 5pt plus 1pt
\parindent 0pt
\evensidemargin -0.1in   \oddsidemargin  -0.1in
\textwidth  6.45in       \textheight 9.1in
\topmargin -1.0cm        \headsep    1.0cm

%-- command (re)definitions -----------------------------------------


\newcommand{\capdef}{}
%\newcommand{\mycaption}[2][\capdef]{\renewcommand{\capdef}{#2}%
%       \caption[#1]{{\itshape #2}}}
\newcommand{\mycaption}[2][\capdef]{\renewcommand{\capdef}{#2}%
       \caption[#1]{{\footnotesize #2}}}
\makeatletter
\renewcommand{\fnum@table}{\textbf{\tablename~\thetable}}
\renewcommand{\fnum@figure}{\textbf{\figurename~\thefigure}}
\makeatother
\def\ltap{\ \raisebox{-.4ex}{\rlap{$\sim$}} \raisebox{.4ex}{$<$}\ }
\def\gtap{\ \raisebox{-.4ex}{\rlap{$\sim$}} \raisebox{.4ex}{$>$}\ }

\newcounter{myenumi}
\newcommand{\myitem}{\refstepcounter{myenumi}\item}
\renewcommand{\themyenumi}{\roman{myenumi}}
\newenvironment{mylist}{%
        \setcounter{myenumi}{0}
        \begin{list}{\textit{\themyenumi)}}{%
        \setlength{\topsep}{0.2\baselineskip}%
        \setlength{\partopsep}{-\topsep}%
        \setlength{\itemsep}{\topsep}%
        \setlength{\parsep}{0\baselineskip}%
        \setlength{\leftmargin}{0em}%
        \setlength{\listparindent}{\parindent}%
        \setlength{\itemindent}{2.5em}%
        \setlength{\labelwidth}{1.5em}%
        \setlength{\labelsep}{0.75em}}}%
{\end{list}}

\newlength{\myem}
\settowidth{\myem}{m}
\newcommand{\sep}[1]{#1}
\newcounter{mysubequation}[equation]
\renewcommand{\themysubequation}{\alph{mysubequation}}
\newcommand{\mytag}{\stepcounter{mysubequation}%
\tag{\theequation\protect\sep{\themysubequation}}}
\newcommand{\globallabel}[1]{\refstepcounter{equation}\label{#1}}

\makeatletter
\renewcommand{\section}{\@startsection{section}{1}{0em}{-\baselineskip}%
{\baselineskip}{\normalfont\large\bfseries}}
\renewcommand{\subsection}%
{\@startsection{subsection}{2}{0em}{-0.7\baselineskip}%
{0.7\baselineskip}{\normalfont\bfseries}}
\makeatother

\newcommand{\ie}{{\it i.e.}}
\newcommand{\Ie}{{\it I.e.}}
\newcommand{\eg}{{\it e.g.}}
\newcommand{\Eg}{{\it E.g.}}
\newcommand{\cf}{{\it cf.}}
\newcommand{\etc}{{\it etc.}}
\newcommand{\eq}{Eq.}
\newcommand{\eqs}{Eqs.}
\newcommand{\Def}{Definition}
\newcommand{\fig}{Fig.}
\newcommand{\Fig}{Fig.}
\newcommand{\figs}{Figs.}
\newcommand{\Figs}{Figs.}
\newcommand{\Ref}{Ref.}
\newcommand{\Refs}{Refs.}
\newcommand{\Sec}{Sec.}
\newcommand{\Secs}{Secs.}
\newcommand{\Chapt}{Chapter}
\newcommand{\Chapts}{Chapters}
\newcommand{\Part}{Part}
\newcommand{\App}{Appendix}
\newcommand{\Apps}{Appendices}
\newcommand{\Tab}{Table}
\newcommand{\Tabs}{Tables}

\newcommand{\bi}{\begin{itemize}}
\newcommand{\ei}{\end{itemize}}
\newcommand{\ra}{$\rightarrow$}
\newcommand{\be}{\begin{equation}}
\newcommand{\ee}{\end{equation}}
\newcommand{\bea}{\begin{eqnarray}}
\newcommand{\eea}{\end{eqnarray}}
\newcommand{\nn}{\nonumber}
\newcommand{\ldm}{\Delta m_{31}^2}
\newcommand{\sdm}{\Delta m_{21}^2}
\newcommand{\deltacp}{\delta_{\mathrm{CP}}}
\newcommand{\stheta}{\sin^2 2 \theta_{13}}

\newcommand{\MINOS}{{\sf MINOS}}
\newcommand{\ICARUS}{{\sf ICARUS}}
\newcommand{\OPERA}{{\sf OPERA}}
\newcommand{\JHFSK}{{\sf JHF-SK}}
\newcommand{\NUMI }{{\sf NuMI}}
\newcommand{\ReactorI}{{\sf Reactor-I}}
\newcommand{\ReactorII}{{\sf Reactor-II}}
\newcommand{\JHFHK}{{\sf JHF-HK}}
\newcommand{\NuFactI}{{\sf NuFact-I}}
\newcommand{\NuFactII}{{\sf NuFact-II}}

\newcommand{\GLOBES}{{\sf GLoBES}}
\newcommand{\AEDL}{{\sf AEDL}}
\newcommand{\EDM}{{\sf EDM}}

\newcommand{\equ}[1]{\eq~(\ref{equ:#1})}
\newcommand{\figu}[1]{\fig~\ref{fig:#1}}
\newcommand{\tabl}[1]{\Tab~\ref{tab:#1}}
\newcommand{\tb}{\hspace*{3ex}}

\begin{document}
%%%%%%%%%%%%%%%%%%%%%%%%%%%%%%%%%%%%%%%%%%%%%%%%%%%%%%%%%%%%%%%%%%%%%
%%%%                     Title-page                              %%%%
%%%%%%%%%%%%%%%%%%%%%%%%%%%%%%%%%%%%%%%%%%%%%%%%%%%%%%%%%%%%%%%%%%%%%

\begin{titlepage}

% the footnote symbols are only redefined for the title page !
\renewcommand{\thefootnote}{\alph{footnote}}

\vspace*{-3.cm}
\begin{flushright}
TUM-HEP-YYY/0X\\
%hep-ph/
\end{flushright}

\vspace*{0.5cm}

\renewcommand{\thefootnote}{\fnsymbol{footnote}}
\setcounter{footnote}{-1}

{\begin{center}
{\Large\bf New features in the simulation of long baseline experiments with GLoBES 3.0 (working title)}
\end{center}}
{\begin{center}
{\large\bf (General Long Baseline Experiment Simulator)}
\end{center}}
\renewcommand{\thefootnote}{\alph{footnote}}

\vspace*{.8cm}
%\vspace*{.3cm}
{\begin{center} {\large{\sc
                P.~Huber\footnotemark[1],~
                J.~Kopp\footnotemark[2],~
                M.~Lindner\footnotemark[2],~
                M.~Rolinec\footnotemark[2],~
                W.~Winter\footnotemark[3]
                }}

\footnote{All correspondence should be addressed to {\tt globes@ph.tum.de}}

\end{center}}
\vspace*{0cm}
{\it
\begin{center}

\footnotemark[1]%${}^,$\footnotemark[2]%
       Department of Physics, University of Wisconsin, \\
       1150 University Avenue, Madison, WI 53706, USA

\vspace*{1mm}

\footnotemark[2]%
       Physik--Department, Technische Universit\"at M\"unchen, \\
       James--Franck--Strasse, 85748 Garching, Germany

\vspace*{1mm}

\footnotemark[3]%
       School of Natural Sciences, Institute for Advanced Study, \\
       Einstein Drive, Princeton, NJ 08540, USA

\end{center}}

\vspace*{1cm}


\begin{abstract}
Here comes the abstract.
\end{abstract}


\vspace*{.5cm}


\end{titlepage}

\newpage

\renewcommand{\thefootnote}{\arabic{footnote}}
\setcounter{footnote}{0}

\section{Introduction}

\nocite{Huber:2004ka}

\section{User-defined priors and systematics}

\bi
 \item
  User-defined priors
 \item
  User-defined systematics
\ei

\section{Simulation of non-standard physics with GLoBES}

HOWTO

\section{Simulation of new experiment features}

\bi
\item 
 Built-in $\beta$-beam fluxes
\item
 New AEDL features
\item
 New AEDL files in standard GLoBES delivery + changes to existing ones
\ei

\section{New platform and interpreter support}

\bi
\item
 PERL?
\item 
 Windows?
\item
 Better RedHat support
\item
 Changes in installation scheme
\ei

\section{New ways to improve performance}

\bi
\item 
 GLoBES on clusters, Condor; parallized code; static links
\item 
 64 bit support
\item 
 New minimization algorithm!?
\ei

\section{Summary and conclusions}

\begin{appendix}
\section{To do list for new feature release (early spring?); internal}

\subsection*{Programming [PH+JK?]}

\bi
\item
 Implement user-defined priors+ systematics (done?)
\item
 Clean up linking scheme: No dynamic linking of modules (physical module
 support) anymore; support static binaries, support Condor etc.;
 problem with append mode (files) [PH]
\item
 New algoriihm for minimzer [JK]? In this context: Is there any
 efficient algorithm for more than six parameters?
\item
 Can globes-command (!) take more than six parameters if GLB\_OSCP $>$ 6?
 Adjust, if necessary.
\item
 Implement Walter's or any modified bugfix for glb\_min\_sup.c
 (we had some negative feedback about that, and for more than
 six parameters ITMAX is far away from being sufficient)
\ei

\subsection*{Testing [MR+All?]}

\bi
\item
 Does release-version support Condor and more than six parameters?
\item 
 Do beta-beam fluxes etc. work?
\item
 Do examples produce same results as before?
\item
 Are AEDL files consistent?
\ei

\subsection*{Documentation [WW with some help?]}

\bi
\item
 Document all new features and make examples; 
 especially user-defined priors+systematics, Non-standard-Support
\item
 New examples files for new examples in manual
\item
 Describe what papers have to be cited
\ei

\subsection*{Paper [WW/ML?]}

\bi
 \item
  Write new release paper (this draft)
\ei

\subsection*{AEDL files [MR?]}

\bi
\item
 Provide new AEDL files for some/any finished (=published) projects!?
 \bi
  \item
   $\beta$-Beams
  \item
   New NuFact (from scoping study); our new NuFact; silver channels?
  \item
   Double Chooz
  \item
   NOvA, BNL beam?
 \ei
\item
 Cleaned up old AEDL files (e.g., 2.5\% background error NuFact etc.);
 document what changes
\item
 New AEDL file forum on webpage
\item
 Clear description in each AEDL file which papers have to be cited
\ei

\subsection*{Website [ML/JK?]}

\bi
 \item
  New layout with new release
 \item
  New FAQ section (e.g., about normalization in AEDL fluxes)
 \item
  New AEDL file forum
 \item
  Figure forum, at least from current projects (download)
\ei

\end{appendix}

\subsection*{Acknowledgments}

We would like to thank ... 

WW would like to
acknowledge support from the W.~M.~Keck Foundation and NSF grant PHY-0503584.

\bibliographystyle{apsrev}
\bibliography{references}

\end{document}

