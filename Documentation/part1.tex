%%%%%%%%%%%%%%%%%%%%%%%%%%%%%%%%%%%
% PART I: User's manual
%%%%%%%%%%%%%%%%%%%%%%%%%%%%%%%%%%%

\part{User's manual}
\chapter{A short \GLOBES\ tour}

To be written later: Short introduction such as ``With the following lines we obtain ...''
without detailed description of parameters. Should contain all most important functions of \GLOBES\ --
overview of main functions in \tabl{stdfunctions}.
Maybe: in form of long example which involves everything (for example: our famous bar plot, which involves
systematics, correlations, degeneracies and thus the full set of \GLOBES\ functions without being too
complicated in the application software part).

\begin{table}[t]
\begin{center}
\begin{tabular}{p{1.8cm}p{4.5cm}p{8.6cm}}
\hline
Function & Purpose & Parameters \ra\ Result \\
\hline
{\tt Chi} & $\chi^2$ with systematics only \newline (all initialized exps.) & ($\{ \theta_{12}, \theta_{13}, \theta_{23}, \deltacp , \sdm , \ldm, \frac{\rho_1}{\bar{\rho}_1},... , \frac{\rho_n}{\bar{\rho}_n} \}$)  \newline \ra\  $\chi^2$ \\[0.1cm]
{\tt SingleChi} & $\chi^2$ with systematics only \newline (only one experiment) & $(\{ \theta_{12}, \theta_{13}, \theta_{23}, \deltacp , \sdm , \ldm, \frac{\rho_{N_{\mathrm{exp}}}}{\bar{\rho}_{N_{\mathrm{exp}}}} \}, \, N_{\mathrm{exp}} )$   \newline \ra\ $\chi^2$ \\[0.1cm]
{\tt ChiTheta} & $\chi^2$ with systematics and correlations: Projection onto $\theta_{13}$-axis (all exps.) &  ($ \theta_{13}, \, \{ \theta_{12}, \theta_{23}, \deltacp , \sdm , \ldm, \frac{\rho_1}{\bar{\rho}_1}, ... , \frac{\rho_n}{\bar{\rho}_n} \}$) \newline \ra\  $\{ \chi^2, \theta_{12}, \theta_{23}, \deltacp , \sdm , \ldm, \frac{\rho_1}{\bar{\rho}_1}, ... , \frac{\rho_n}{\bar{\rho}_n} , N_{\mathrm{Iter}} \}$ \\[0.1cm]
{\tt Single-} \newline {\tt ChiTheta} & $\chi^2$ with systematics and correlations: Projection onto $\theta_{13}$-axis (one exp.) &  ($ \theta_{13}, \, \{ \theta_{12}, \theta_{23}, \deltacp , \sdm , \ldm,  \frac{\rho_{N_{\mathrm{exp}}}}{\bar{\rho}_{N_{\mathrm{exp}}}} \}, \, N_{\mathrm{exp}}$) \newline \ra\  $\{ \chi^2, \theta_{12}, \theta_{23}, \deltacp , \sdm , \ldm, \frac{\rho_{N_{\mathrm{exp}}}}{\bar{\rho}_{N_{\mathrm{exp}}}} , N_{\mathrm{Iter}} \}$ \\[0.1cm]
{\tt ChiDelta} & $\chi^2$ with systematics and correlations: Projection onto $\deltacp$-axis (all exps.) &  ($\deltacp, \, \{ \theta_{12}, \theta_{13}, \theta_{23},  \sdm , \ldm, \frac{\rho_1}{\bar{\rho}_1}, ... , \frac{\rho_n}{\bar{\rho}_n} \}$) \newline \ra\  $\{ \chi^2, \theta_{12}, \theta_{13}, \theta_{23}, \sdm , \ldm, \frac{\rho_1}{\bar{\rho}_1}, ... , \frac{\rho_n}{\bar{\rho}_n} , N_{\mathrm{Iter}} \}$ \\[0.1cm]
{\tt Single-} \newline {\tt ChiDelta} & $\chi^2$ with systematics and correlations: Projection onto $\deltacp$-axis (one exp.) &  ($ \deltacp, \, \{ \theta_{12}, \theta_{13}, \theta_{23}, \sdm , \ldm,  \frac{\rho_{N_{\mathrm{exp}}}}{\bar{\rho}_{N_{\mathrm{exp}}}} \}, \, N_{\mathrm{exp}}$) \newline \ra\  $\{ \chi^2, \theta_{12}, \theta_{13}, \theta_{23},  \sdm , \ldm, \frac{\rho_{N_{\mathrm{exp}}}}{\bar{\rho}_{N_{\mathrm{exp}}}} , N_{\mathrm{Iter}} \}$ \\[0.1cm]
{\tt ChiTheta-} {\tt Delta} & $\chi^2$ with systematics and correlations: Projection onto $\deltacp$-$\theta_{13}$-plane (all exps.) &  ($\theta_{13}, \, \deltacp, \, \{ \theta_{12}, \theta_{23},  \sdm , \ldm, \frac{\rho_1}{\bar{\rho}_1}, ... , \frac{\rho_n}{\bar{\rho}_n} \}$) \newline \ra\  $\{ \chi^2, \theta_{12}, \theta_{23}, \sdm , \ldm, \frac{\rho_1}{\bar{\rho}_1}, ... , \frac{\rho_n}{\bar{\rho}_n} , N_{\mathrm{Iter}} \}$ \\[0.1cm]
{\tt SingleChi-} {\tt Theta- } {\tt Delta} & $\chi^2$ with systematics and correlations: Projection onto $\deltacp$-$\theta_{13}$-plane (one exp.) &  ($ \theta_{13}, \, \deltacp, \, \{ \theta_{12}, \theta_{23}, \sdm , \ldm,  \frac{\rho_{N_{\mathrm{exp}}}}{\bar{\rho}_{N_{\mathrm{exp}}}} \}, \, N_{\mathrm{exp}}$) \newline \ra\  $\{ \chi^2, \theta_{12},  \theta_{23},  \sdm , \ldm, \frac{\rho_{N_{\mathrm{exp}}}}{\bar{\rho}_{N_{\mathrm{exp}}}} , N_{\mathrm{Iter}} \}$ \\[0.1cm]
{\tt ChiNP} & $\chi^2$ with systematics and correlations: Projection onto $N$-parameter hyper-plane (one exp.) & ??? to be defined \\[0.1cm]
{\tt Single-} {\tt ChiNP} & $\chi^2$ with systematics and correlations: Projection onto $N$-parameter hyper-plane (one exp.) & ??? to be defined \\[0.1cm]
{\tt ChiAll} & Lokal minimum of $\chi^2$ with respect to all parameters (all exps.) &
($ \{ \theta_{13}, \theta_{12}, \theta_{23}, \deltacp , \sdm , \ldm, \frac{\rho_1}{\bar{\rho}_1}, ... , \frac{\rho_n}{\bar{\rho}_n} \}$) \newline \ra\  $\{ \chi^2, \theta_{13}, \theta_{12}, \theta_{23},$ \newline \hspace*{1.4cm} $ \deltacp , \sdm , \ldm, \frac{\rho_1}{\bar{\rho}_1}, ... , \frac{\rho_n}{\bar{\rho}_n} , N_{\mathrm{Iter}} \}$ 
\\[0.1cm]
{\tt Single-} {\tt ChiAll} & Lokal minimum of $\chi^2$ with respect to all parameters (one exp.) &  ($ \{ \theta_{13}, \theta_{12}, \theta_{23}, \deltacp , \sdm , \ldm,  \frac{\rho_{N_{\mathrm{exp}}}}{\bar{\rho}_{N_{\mathrm{exp}}}} \}, \, N_{\mathrm{exp}}$) \newline \ra\  $\{ \chi^2, \theta_{13}, \theta_{12}, \theta_{23}, \deltacp , \sdm , \ldm, \frac{\rho_{N_{\mathrm{exp}}}}{\bar{\rho}_{N_{\mathrm{exp}}}} , N_{\mathrm{Iter}} \}$ \\[0.1cm]
\hline
\end{tabular}
\end{center}
\caption{\label{tab:stdfunctions} The \GLOBES\ standard function to obtain a $\chi^2$-value for all or one of the initialized experiments. The curly brackets refer to the parameters to be transferred in form of a list. Note that all functions but {\tt Chi} and {\tt SingleChi} are using minimizers which have to be initialiued with {\tt SetInputErrors} and {\tt SetStartingValues} first.}
\end{table}

\chapter{Getting started with \GLOBES }

In this first chapter of the user's manual, we assume that the \GLOBES\ software is readily installed on your computer system. We demonstrate how to load pre-defined experiments and re-obtain information about them. However, we only secondarily discuss the usage of \GLOBES\ in your specific programming language, such as C, Mathematica, or others. Thus, you should be familiar of how to load \GLOBES\ on your computer system before reading this chapter. An example of how to use \GLOBES\ with C can be found on page~\pageref{ex:c}. 

\example{Using \GLOBES\ with C}{\label{ex:c}

Here comes some complete C-code with a very simple example of how to use \GLOBES .
 
}

\begin{table}[t]
\begin{center}
\begin{tabular}{lll}
\hline
Quantities & Examples & Units \\
\hline
Angles & $\theta_{13}$, $\theta_{12}$, $\theta_{23}$, $\deltacp$ & Radians  \\
Mass squared differences & $\sdm$, $\ldm$ & $\mathrm{eV}^2$ \\
Matter densitities & $\rho_i$ & $\mathrm{g}/\mathrm{cm}^3$ \\
Baseline lengths & $L_i$ & $\mathrm{km}$ \\
Energies & $E_\nu$ & $\mathrm{GeV}$ \\  
Fiducial masses & $m_{\mathrm{Det}}$ & $\mathrm{kt}$ \\
Time intervals & $t_{\mathrm{run}}$ & $\mathrm{yr}$ \\
Source powers & $P_{\mathrm{Source}}$ & ??? \\
% Integrated luminosities & $m_{\mathrm{Det}} \, t_{\mathrm{run}}$ & $\mathrm{kt \cdot yr}$ \\
Cross sections & $\sigma_{\mathrm{CC}}$ &  ???? \\
\hline
\end{tabular}
\mycaption{\label{tab:units} Quantities used in \GLOBES , examples of these quantities, and their standard units in the application software.}
\end{center}
\end{table}

Throughout the programming interface of \GLOBES , the software needs to transfer parameters to and from the software core. Unless the Experiment Definition Module \EDM , the programming interface only uses one set of units for each type of quantity in order to avoid confusion about the definition of individual parameters. \tabl{units} summarizes the units of the most important quantities used in \GLOBES .

\begin{table}[t]
\begin{center}
\begin{tabular}{llp{7cm}c}
\hline
Experiment & Filename & Short description & Refs. \\
\hline 
\multicolumn{3}{l}{\underline{Conventional beams:}} \\
??? & & \\[0.1cm]

\multicolumn{3}{l}{\underline{First-generation superbeams:}} \\
\JHFSK\ ($\nu$) & {\tt JHFSK.exp} & JHF (J-PARC) to Super-Kamiokande, neutrino running &  \cite{Huber:2002mx,Huber:2002rs} \\
\JHFSK\ ($\bar\nu$)& {\tt JHFSKanti.exp} & JHF (J-PARC) to Super-Kamiokande, antineutrino running &  \cite{Huber:2002rs} \\
\NUMI\  ($\nu$), OA $9 \, \mathrm{km}$ & {\tt NUMI9.exp} & NuMI with off-axis angle of $9 \, \mathrm{km}$ for $L=712 \, \mathrm{km}$, neutrino running & \cite{Huber:2002rs} \\
\NUMI\  ($\bar{\nu}$), OA $9 \, \mathrm{km}$ & {\tt NUMI9anti.exp} & NuMI with off-axis angle of $9 \, \mathrm{km}$ for $L=712 \, \mathrm{km}$, antineutrino running & \cite{Huber:2002rs} \\
\NUMI\  ($\nu$), OA $12 \, \mathrm{km}$ & {\tt NUMI12.exp} & NuMI with off-axis angle of $12 \, \mathrm{km}$ for $L=712 \, \mathrm{km}$, neutrino running & \cite{Huber:2002rs} \\
\NUMI\  ($\bar{\nu}$), OA $12 \, \mathrm{km}$ & {\tt NUMI12anti.exp} & NuMI with off-axis angle of $12 \, \mathrm{km}$ for $L=712 \, \mathrm{km}$, antineutrino running & \cite{Huber:2002rs} \\
\SPL\  ($\nu$) & {\tt SPL.exp} & SPL (CERN), neutrino running &  ??? \\
\SPL\  ($\bar\nu$) & {\tt SPLanti.exp} & SPL (CERN), antineutrino running & ??? \\[0.1cm]
 
\multicolumn{3}{l}{\underline{Superbeam upgrades:}} \\
\JHFHK\ ($\nu$) & {\tt JHFHK.exp} & JHF (J-PARC) to Hyper-Kamiokande superbeam upgrade, neutrino running &  \cite{Huber:2002mx,Huber:2002rs} \\
\JHFHK\ ($\bar\nu$)& {\tt JHFHKanti.exp} & JHF (J-PARC) to Hyper-Kamiokande superbeam upgrade, antineutrino running &  \cite{Huber:2002mx,Huber:2002rs} \\[0.1cm]

\multicolumn{3}{l}{\underline{Neutrino factories:}} \\
\NuFactI\ & {\tt NuFact.exp} & Initial stage neutrino factory, symmetric operation in both polarities & \cite{Huber:2002mx} \\
\NuFactII\  & {\tt NuFact2.exp} & Advanced stage neutrino factory, symmetric operation in both polarities & \cite{Huber:2002mx,Huber:2003ak} \\[0.1cm]

\multicolumn{3}{l}{\underline{Reactor experiments:}} \\
\ReactorI\ & {\tt Reactor.exp} & Small reactor experiment with identical near and far detectors & \cite{Huber:2003pm} \\
\ReactorII\ & {\tt Reactor2.exp} & Large reactor experiment with identical near and far detectors & \cite{Huber:2003pm} \\[0.1cm]

\multicolumn{3}{l}{\underline{$\beta$-Beams:}} \\
\Beta\ ($\nu$) & {\tt BETA.exp} & $\beta$-Beam, neutrino running & ??? \\
\Beta\ ($\bar\nu$) & {\tt BETAanti.exp} & $\beta$-Beam, antineutrino running & ??? \\
\hline
\end{tabular}
\end{center}
\mycaption{\label{tab:experiments} Different pre-defined experiments, their filenames (to be used in {\tt LoadExperiment}), their short description, and the references in which they are defined. Note that all experiments use in their standard configurations one year of running time. Details about the experiment parameters can be obtained with {\tt InfoExperiment}(Experiment number) after they have been loaded.}
\end{table}

In principle, \GLOBES\ can handle any number of different long-baseline experiments simultaneously. This means that their $\chi^2$-values are added {\em after} the minimization over the independent systematics parameters and {\em before} any minimization over the oscillation parameters. Though the simplest case of only one experiment may be most often used, more experiments are useful in many cases. For example, running a superbeam some years in the neutrino mode and some years in the antineutrino mode can be, to a first approximation, simulated by the combination of two such experiments.\footnote{Note that in this case the systematics parameters are minimized over independently, which means that this approach does not allow correlations among the systematics parameters. Therefore, the neutrino factory with the symmetric operation of both polarities is encapsulated into a single experiment.} Another example is the test of synergetic effects among different experiment types. Thus, \GLOBES\ has an internal (initially empty) list of currently initialized experiments. To add a pre-defined experiment to this list, one can use the function {\tt LoadExperiment}:
\begin{function}
{\tt LoadExperiment}$($``filename'' $)$ adds a single experiment to the list of currently loaded experiments. All currently loaded experiments are evaluated simultaneously (if not explicitely stated otherwise), \ie , their $\chi^2$-values are added.
\end{function}
A list of pre-defined experiment types, their filenames, their short descriptions, and the references of their definitions can be found in \tabl{experiments}. To remove all experiments from the evaluation list, one uses {\tt ClearExp}:
\begin{function}
{\tt ClearExp}$()$ removes all experiments from the evaluation list.   
\end{function}
Both functions do not return anything. Thus, one can either add an experiment to the internal evaluation list, or remove all experiments from this list. After adding an experiment, it gets an internal experiment number $N_{\mathrm{exp}}$ assigned in the order of the addition, which is starting from zero and running to the number of experiments minus one. Therefore, one will be able to access the individual experiment by its number later.

Since the pre-defined experiments in \tabl{experiments} are given for one year running time, specific target masses, and specific source powers, it is useful to change these parameters of the individual experiments:
\begin{function}
{\tt SetRunningTime}$(N_{\mathrm{exp}},t_{\mathrm{run}})$ sets the running time of experiment number $N_{\mathrm{exp}}$ to $t_{\mathrm{run}}$ years.
\end{function}
 \begin{function}
{\tt SetTargetMass}$(N_{\mathrm{exp}},m_{\mathrm{Det}})$ sets the fiducial mass of experiment number $N_{\mathrm{exp}}$ to $m_{\mathrm{Det}}$ kilotons.
\end{function}
\begin{function}
{\tt SetSourcePower}$(N_{\mathrm{exp}},P_{\mathrm{Source}})$ sets the source power of experiment number $N_{\mathrm{exp}}$ to $P_{\mathrm{source}}$. The definition of the source power depends on the experiment type: ... (MISSING).
\end{function}
Thus, these functions also demonstrate how to use the assigned experiment number.

A useful function to re-obtain the information about the initialized experiments is the function {\tt InfoExperiment}:
\begin{function}
{\tt InfoExperiment}$()$ prints a list of the initialized functions with their experiment numbers and their most important parameters to the standard output.
\end{function} 
Especially, after changing individual parameters, such as baseline or target mass, this information can be useful to check the changes. Another useful function is {\tt ShowChannels}, which prints the initialized oscillation channels for a specified experiment:
\begin{function}
{\tt ShowChannels}$(N_{\mathrm{exp}})$ prints the information about the oscillation channels of the experiment with the number $N_{\mathrm{exp}}$ to the standard output.
\end{function}

Compared to an existing experiment, which uses real data, a future experiment uses simulated data. Thus, the {\em true parameter values} and their results in form of the reference rate vectors are simulated. After setting the true parameter values, the {\em fit parameter values} can be varied in order to obtain information on the measurement performance for the given set of true parameter values. Therefore, it is often useful to show the results of a future measurement as function of the true parameter values for which the reference rate vectors are computed -- at least within the currently allowed ranges. The true parameter values for the vacuum neutrino oscillation parameters have to be set by the functions {\tt SetVacuumParameters} and {\tt SetRates} {\em before} any evaluation function is used and {\em after} the experiments have been initialized and the experiment parameters have been adjusted which could change the rates (such as baseline or target mass). Any matter effects are then included automatically depending on the experiment definitions.
\begin{function}
{\tt SetVacuumParameters}$(\{\theta_{12}, \theta_{13}, \theta_{23}, \deltacp , \sdm , \ldm \})$ sets the neutrino oscillation parameters to be used to compute the reference rate vector in vacuum.
\end{function}
\begin{function}
{\tt SetRates}$( )$ computes the reference rate vector for the neutrino oscillation parameters set with {\tt SetVacuumParameters}. 
\end{function}
Finally, an initialization sequence for \GLOBES\ could look like this:
\begin{quote}
{\tt ClearExp}$()$; \\
{\tt InitExperiment}$($``{\tt JHFHK.exp}''$)$; \\
{\tt SetRunningTime}$(0,2.0)$; \\
{\tt InitExperiment}$($``{\tt JHFHKanti.exp}''$)$; \\
{\tt SetRunningTime}$(1,6.0)$;\\
{\tt MInfoExperiment}$()$; \\
{\tt SetVacuumParameters}$(\{ 0.55, \, 0.16, \, 3.14/4, \, 3.14/2, \, 7e-5 , \, 2e-3 \} )$; \\
{\tt SetRates();} 
\end{quote}
This piece of code initializes the JHF (J-PARC) to Hyper-Kamiokande superbeam upgrade with two years of neutrino running and six years of antineutrino running, \ie, an overall running time of eight years. The final configuration is then printed to the standard output and the reference rate vector is set to the chosen parameter values.

\chapter[Calculating $\chi^2$ with systematics only]{Calculating $\boldsymbol{\chi^2}$ with systematics only}

\bi
\item
 More technical: what does it do (minimizer); how to imagine (simple example); 
what functions are there; examples
\item
 Single/multiple experiments and the connection (sum of individual $\chi$'s )
\item
 Advanced topic: Changing the systematics treatment (SetSystematics)
\ei

\example{Correlation between $\stheta$ and $\deltacp$}{Text}

\chapter[Calculating $\chi^2$-projections: how to include correlations]{Calculating $\boldsymbol{\chi^2}$-projections: how to include correlations}

\section[Projection onto $\theta_{13}$ or $\delta$]{Projection onto $\boldsymbol{\theta_{13}}$ or $\boldsymbol{\delta}$}

\bi
\item
 Single and multiple experiments
\item
 Figure: how does that work: local routine
\item
 Treatment of external input
\ei

\section[Projection onto any set of $N \ge 1$ parameters]{Projection onto any set of $\boldsymbol{N \ge 1}$ parameters}

General form and especially: DeltaTheta-Minimizer; mention calculation time!

\chapter{Finding degenerate solutions}

\bi
\item
 Single- and multiple experiment support
\item
 What does that do?
\item
 Example: how to treat sign-degeneracy in th13-limit
\ei

\chapter{Obtaining low-level information}

\bi
\item
 Obtaining rate vectors
\item
 Obtaining fluxes/cross sections etc.
\item
 ``Check''-functions
\ei

\chapter{Changing experiment parameters at the running time}

\bi
\item
 Baseline, Target mass, Source power, running time
\item
 Matter density profile
\item
 Threshold function, efficiencies etc.
\ei