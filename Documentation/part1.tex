%%%%%%%%%%%%%%%%%%%%%%%%%%%%%%%%%%%
% PART I: User's manual
%%%%%%%%%%%%%%%%%%%%%%%%%%%%%%%%%%%

\part{User's manual}
\chapter{Getting started: A short \GLOBES\ tour}

Also: which units, which order of parameters (table for different functions!) etc. (somewhere)

\chapter{Setting up \GLOBES }

\bi
\item
 How to install and start \GLOBES
\item
 How to set the vacuum parameters 
\item
 Which units
\item
 How to load one or more experiments and re-obtain formation about them
\item
Table: pre-defined experiment types
\item
 How to clear experiments
\ei

\chapter[Calculating $\chi^2$ with systematics only]{Calculating $\boldsymbol{\chi^2}$ with systematics only}

\bi
\item
 More technical: what does it do; how to imagine; what functions are there; examples
 Single/multiple experiments and the connection (sum of individual $\chi$'s )
\item
 Advanced topic: Changing the systematics treatment (SetErrorDim)
\ei

\chapter[Calculating $\chi^2$-projections: how to include correlations]{Calculating $\boldsymbol{\chi^2}$-projections: how to include correlations}

\section[Projection onto $\theta_{13}$ or $\delta$]{Projection onto $\boldsymbol{\theta_{13}}$ or $\boldsymbol{\delta}$}

\bi
\item
 Single and multiple experiments
\item
 Figure: how does that work: local routine
\item
 Treatment of external input
\ei

\section[Projection onto any set of $N \ge 1$ parameters]{Projection onto any set of $\boldsymbol{N \ge 1}$ parameters}

General form and especially: DeltaTheta-Minimizer; mention calculation time!

\chapter{Finding degenerate solutions}

\bi
\item
 Single- and multiple experiment support
\item
 What does that do?
\item
 Example: how to treat sign-degeneracy in th13-limit
\ei

\chapter{Obtaining low-level information}

\bi
\item
 Obtaining rate vectors
\item
 Obtaining fluxes/cross sections etc.
\item
 ``Check''-functions
\ei

\chapter{Changing experiment parameters at the running time}

\bi
\item
 Baseline, Target mass
\item
 Matter density profile
\item
 Threshold function, efficiencies etc.
\ei