%%%%%%%%%%%%%%%%%%%%%%%%%%%%%%%%%%%
% PART I: User's manual
%%%%%%%%%%%%%%%%%%%%%%%%%%%%%%%%%%%

\part{User's manual}
\chapter{Getting started: A short \GLOBES\ tour}

Also: which units, which order of parameters (table for different functions!) etc. (somewhere)

\begin{table}[t]
\begin{center}
\begin{tabular}{lll}
\hline
Quantities & Examples & Units \\
\hline
Angles & $\theta_{13}$, $\theta_{12}$, $\theta_{23}$ & Radians  \\
Mass squared differences & $\sdm$, $\ldm$ & $\mathrm{eV}^2$ \\
Matter densitities & $\rho_i$ & $\mathrm{g}/\mathrm{cm}^3$ \\
Baseline lengths & $L_i$ & $\mathrm{km}$ \\
Energies & $E_\nu$ & $\mathrm{GeV}$ \\  
Fiducial masses & $m_{\mathrm{Det}}$ & $\mathrm{kt}$ \\
Time intervals & $t_{\mathrm{run}}$ & $\mathrm{yr}$ \\
Integrated luminosities & $m_{\mathrm{Det}} \, t_{\mathrm{run}}$ & $\mathrm{kt \cdot yr}$ \\
Cross sections & $\sigma_{\mathrm{CC}}$ &  ???? \\
\hline
\end{tabular}
\mycaption{\label{tab:units} Quantities used in \GLOBES , examples of these quantities, and their standard units in the application software.}
\end{center}
\end{table}

\begin{table}[t]
\begin{center}
\begin{tabular}{llp{7cm}c}
\hline
Experiment & Filename & Short description & Refs. \\
\hline 
\multicolumn{3}{l}{\underline{Conventional beams:}} \\
??? & & \\[0.1cm]

\multicolumn{3}{l}{\underline{First-generation superbeams:}} \\
\JHFSK\ ($\nu$) & {\tt JHFSK.exp} & JHF (J-PARC) to Super-Kamiokande, neutrino running &  \cite{Huber:2002mx,Huber:2002rs} \\
\JHFSK\ ($\bar\nu$)& {\tt JHFSKanti.exp} & JHF (J-PARC) to Super-Kamiokande, antineutrino running &  \cite{Huber:2002rs} \\
\NUMI\  ($\nu$), OA $9 \, \mathrm{km}$ & {\tt NUMI9.exp} & NuMI with off-axis angle of $9 \, \mathrm{km}$ for $L=712 \, \mathrm{km}$, neutrino running & \cite{Huber:2002rs} \\
\NUMI\  ($\bar{\nu}$), OA $9 \, \mathrm{km}$ & {\tt NUMI9anti.exp} & NuMI with off-axis angle of $9 \, \mathrm{km}$ for $L=712 \, \mathrm{km}$, antineutrino running & \cite{Huber:2002rs} \\
\NUMI\  ($\nu$), OA $12 \, \mathrm{km}$ & {\tt NUMI12.exp} & NuMI with off-axis angle of $12 \, \mathrm{km}$ for $L=712 \, \mathrm{km}$, neutrino running & \cite{Huber:2002rs} \\
\NUMI\  ($\bar{\nu}$), OA $12 \, \mathrm{km}$ & {\tt NUMI12anti.exp} & NuMI with off-axis angle of $12 \, \mathrm{km}$ for $L=712 \, \mathrm{km}$, antineutrino running & \cite{Huber:2002rs} \\
\SPL\  ($\nu$) & {\tt SPL.exp} & SPL (CERN), neutrino running &  ??? \\
\SPL\  ($\bar\nu$) & {\tt SPLanti.exp} & SPL (CERN), antineutrino running & ??? \\[0.1cm]
 
\multicolumn{3}{l}{\underline{Superbeam upgrades:}} \\
\JHFHK\ ($\nu$) & {\tt JHFHK.exp} & JHF (J-PARC) to Hyper-Kamiokande superbeam upgrade, neutrino running &  \cite{Huber:2002mx,Huber:2002rs} \\
\JHFHK\ ($\bar\nu$)& {\tt JHFHKanti.exp} & JHF (J-PARC) to Hyper-Kamiokande superbeam upgrade, antineutrino running &  \cite{Huber:2002mx,Huber:2002rs} \\[0.1cm]

\multicolumn{3}{l}{\underline{Neutrino factories:}} \\
\NuFactI\ & {\tt NuFact.exp} & Initial stage neutrino factory, symmetric operation in both polarities & \cite{Huber:2002mx} \\
\NuFactII\  & {\tt NuFact2.exp} & Advanced stage neutrino factory, symmetric operation in both polarities & \cite{Huber:2002mx,Huber:2003ak} \\[0.1cm]

\multicolumn{3}{l}{\underline{Reactor experiments:}} \\
\ReactorI\ & {\tt Reactor.exp} & Small reactor experiment with identical near and far detectors & \cite{Huber:2003pm} \\
\ReactorII\ & {\tt Reactor2.exp} & Large reactor experiment with identical near and far detectors & \cite{Huber:2003pm} \\[0.1cm]

\multicolumn{3}{l}{\underline{$\beta$-Beams:}} \\
\Beta\ ($\nu$) & {\tt BETA.exp} & $\beta$-Beam, neutrino running & ??? \\
\Beta\ ($\bar\nu$) & {\tt BETAanti.exp} & $\beta$-Beam, antineutrino running & ??? \\
\hline
\end{tabular}
\end{center}
\mycaption{\label{tab:experiments} Different pre-defined experiments, their filenames (to be used in {\tt LoadExperiment}), their short description, and the references in which they are defined. Note that all experiments use in their standard configurations one year of running time. Details about the experiment parameters can be obtained with {\tt InfoExperiment}(Experiment number) after they have been loaded.}
\end{table}

\chapter{Setting up \GLOBES }

\bi
\item
 How to install and start \GLOBES
\item
 How to set the vacuum parameters 
\item
 Which units
\item
 How to load one or more experiments and re-obtain formation about them
\item
Table: pre-defined experiment types
\item
 How to clear experiments
\ei

\chapter[Calculating $\chi^2$ with systematics only]{Calculating $\boldsymbol{\chi^2}$ with systematics only}

\bi
\item
 More technical: what does it do; how to imagine; what functions are there; examples
 Single/multiple experiments and the connection (sum of individual $\chi$'s )
\item
 Advanced topic: Changing the systematics treatment (SetErrorDim)
\ei

\chapter[Calculating $\chi^2$-projections: how to include correlations]{Calculating $\boldsymbol{\chi^2}$-projections: how to include correlations}

\section[Projection onto $\theta_{13}$ or $\delta$]{Projection onto $\boldsymbol{\theta_{13}}$ or $\boldsymbol{\delta}$}

\bi
\item
 Single and multiple experiments
\item
 Figure: how does that work: local routine
\item
 Treatment of external input
\ei

\section[Projection onto any set of $N \ge 1$ parameters]{Projection onto any set of $\boldsymbol{N \ge 1}$ parameters}

General form and especially: DeltaTheta-Minimizer; mention calculation time!

\chapter{Finding degenerate solutions}

\bi
\item
 Single- and multiple experiment support
\item
 What does that do?
\item
 Example: how to treat sign-degeneracy in th13-limit
\ei

\chapter{Obtaining low-level information}

\bi
\item
 Obtaining rate vectors
\item
 Obtaining fluxes/cross sections etc.
\item
 ``Check''-functions
\ei

\chapter{Changing experiment parameters at the running time}

\bi
\item
 Baseline, Target mass
\item
 Matter density profile
\item
 Threshold function, efficiencies etc.
\ei