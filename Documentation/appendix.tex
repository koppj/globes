%%%%%%%%%%%%%%%%%%%%%%%%%%%%%%%%%%%
% Appendix
%%%%%%%%%%%%%%%%%%%%%%%%%%%%%%%%%%%

\begin{appendix}

\chapter{\GLOBES\ installation}
\label{app:installation}
\index{norm}{Installation|(}

\section*{Prerequisites for installation of \GLOBES}

Besides the usual things like a working libc you need to have
 \begin{itemize}
        \item[gcc]      The GNU compiler collection\\
                        \verb+gcc.gnu.org+
        \item[GSL]      The GNU Scientific Library\\
                        \verb+www.gnu.org/software/gsl/+
\end{itemize}
The library \verb+libglobes+ should in principle compile with any C/C++
compiler but the \verb+globes+ binary uses the \verb+argp+ facility of \verb+glibc+
to parse its command line options. However, on platforms where \verb+argp+
is lacking, \GLOBES\ has replacement code, thus it should also work
there. \GLOBES\ is, however, using the C99 standard in order to handle
complex numbers, but that is the only feature of C99 used.

GSL is also available as rpm's from the various distributors of
GNU/Linux, see their web sites for downloads. Chances are that gcc and
GSL are already part of your installation.  For building \GLOBES\ from
source, however, not only working libraries for the above packages are
needed, but also the headers, especially for GSL. For some installations
of GSL, eg. on RedHat/Fedora, this may require to
additionally install a rpm-package named \verb+gsl-devel+. If GSL has been
installed from the tar-ball as provided by gnu.org, no problems should
occur. Furthermore you need a working \verb+make+ to build and install
\GLOBES.


\section*{Installation Instructions}


\GLOBES\ follows the standard GNU installation procedure.  To compile
\GLOBES\ you will need an ANSI C-compiler.  After unpacking the
distribution, the Makefiles can be prepared using the configure
command,\\
\verb+  ./configure+\\
You can then build the library by typing,\\
\verb+  make+\\
A shared  version of the library will be compiled by
default. 
 
The libraries and modules can be installed using the command,\\
\verb+  make install+\\
The install target also will install a program with name \verb+globes+
 to \verb+/usr/local/bin+

The default install directory prefix is \verb+/usr/local+.  Consult the
"Further Information" section below for instructions on installing the
library in another location or changing other default compilation
options.

Moreover a config-script called \verb+globes-config+ will be
installed. This script displays all information necessary to link any
program with \GLOBES.  For building static libraries and linking against
them see the corresponding section of this file.
 
                    

\section*{Basic Installation}



   The \verb+configure+ shell script attempts to guess correct values for
various system-dependent variables used during compilation.  It
uses those values to create a \verb+Makefile+ in each directory of the
package.  It may also create one or more \verb+.h+ files containing
system-dependent definitions.  Finally, it creates a shell script
\verb+config.status+ that you can run in the future to recreate the
current configuration, a file \verb+config.cache+ that saves the results
of its tests to speed up reconfiguring, and a file \verb+config.log+
containing compiler output (useful mainly for debugging
\verb+configure+).

   If you need to do unusual things to compile the package, please try
to figure out how \verb+configure+ could check whether to do them, and mail
diffs or instructions to the address given in the \verb+README+ so they can
be considered for the next release.  If at some point \verb+config.cache+
contains results you don't want to keep, you may remove or edit it.

   The file \verb+configure.in+ is used to create \verb+configure+ by a program
called \verb+autoconf+.  You only need \verb+configure.in+ if you want to change
it or regenerate \verb+configure+ using a newer version of \verb+autoconf+.

The simplest way to compile this package is:
\begin{enumerate}
 \item \verb+cd+ to the directory containing the package's source code and type
     \verb+./configure+ to configure the package for your system.  If you're
     using \verb+csh+ on an old version of System V, you might need to type
     \verb+sh ./configure+ instead to prevent \verb+csh+ from trying to execute
     \verb+configure+ itself.

     Running \verb+configure+ takes awhile.  While running, it prints some
     messages telling which features it is checking for.
\item Type \verb+make+ to compile the package.

 \item Type \verb+make install+ to install the programs and any data files and
     documentation.

\item You can remove the program binaries and object files from the
     source code directory by typing \verb+make clean+.  To also remove the
     files that \verb+configure+ created (so you can compile the package for
     a different kind of computer), type \verb+make distclean+.  There is
     also a \verb+make maintainer-clean+ target, but that is intended mainly
     for the package's developers.  If you use it, you may have to get
     all sorts of other programs in order to regenerate files that came
     with the distribution.

\item Since you have installed a library don't forget to run \verb+ldconfig+!
\end{enumerate}
\section*{Installation without root privilege}


Install \GLOBES\ to a directory of your choice \verb+GLB_DIR+. This is done by\\
\verb+  configure --prefix=GLB_DIR+\\ 
and then follow the usual
installation guide. The only remaining problem is that you
have to tell the compiler where to find the header files, and
the linker where to find the library. Furthermore you have to
make sure that the shared object files are found during
execution. Running \verb+configure+ also produces a \verb+Makefile+ in
the examples subdirectory which can serve as a template for
the compilation and linking process, since all necessary flags
are correctly filled in. Another solution is to set the
environment variable \verb+LD_RUN_PATH+ during linking to
\verb+GLB_DIR/lib/+. Best thing is to add this to your shell dot-file
(e.g. \verb+.bashrc+). Then you can use: A typical compiler command
like\\
\verb+  gcc -c my_program.c -IGLB_DIR/include/+\\
and a typical linker command like\\
\verb+  gcc my_program.o -lglobes -LGLB_DIR/lib/ -o my_executable+\\
More information on this issue can be obtained by having a look into
the output of make install.

{\bf CAVEAT}: It is in principle possible to have many installations on one
machine, especially the situation of having an installation by root
and by a user at the same time might occur. However it is strictly
warned against this possibility since it is \emph{extremely} likely to
create some versioning problem at some time!

\section*{Building and Using static versions of \GLOBES}


Under certain circumstances it may be useful to use a static version of
libglobes or any of the binaries, e.g.\ when running on a cluster.

The \verb+configure+ script accepts the option \verb+--disable-shared+, in
which case only static objects are built, i.e.\ only a static
version of libglobes. In case your system does not support shared
libraries the \verb+configure+ script recognizes this. If you give no
options to \verb+configure+, both shared and static versions are built
and will be installed. All binaries, however, will use dynamic
linking. If you want to build static binaries, use
\verb+LDFLAGS='-all-static'+ for building them.

Sometimes it is convenient, eg. for debugging purposes, to have a
statically linked version of a program using \GLOBES, which is easiest
achieved by just linking with \verb+libglobes.a+. If you need a completely
statically linked version, please, have a look at the Makefile in the
\verb+examples+ directory.\\
\verb+  make example-static+\\ 
produces a statically linked program that should in principle run on
most Linuxes.  It should be straightforward to adapt this example to
your needs.

All these options rely on a working \verb+gcc+ installation. It seems that
\verb+gcc 3.x+ is broken in a subtle way which makes it necessary to add a
symbolic link in the gcc library directory. The diagnostic for this
requirement is that building static programs fails with the error
message \verb+cannot find -lgcc_s+. In those cases, find \verb+libgcc.a+ and add
a symbolic link in the same directory where you found it (this
requires probably root privileges)\\
\verb+  ln -s libgcc.a libgcc_s.a+

If you can not write to this directory just use the following work
around. Add the same link as above to the directory where you
installed \GLOBES\ into\\
\verb+  cd prefix/lib+\\       
\verb+  ln -s path_to_libgcc.a/libgcc.a libgcc_s.a+\\
and then change back into the \verb+examples+ directory and type\\
\verb+  make LDFLAGS=-Lprefix/lib example-static+\\
and you are done.

\section*{\GLOBES\ and Condor}
\label{sec:condor}
\index{norm}{Condor|(}
\begin{quote}
  Condor is a specialized workload management system for
  compute-intensive jobs. Like other full-featured batch systems,
  Condor provides a job queuing mechanism, scheduling policy,
  priority scheme, resource monitoring, and resource management.
\end{quote} 

A Condor (\verb+www.cs.wisc.edu/condor/+) cluster is very well suited
to run large \GLOBES-based computation. The nature of the problems
addressed with \GLOBES\ is such that one typically ends up with a so
called 'embarrassingly parallel' program. That means, that one repeats
the same task $N$ times, where each execution is independent of the
other $N-1$. Therefore, this execution should become $M$ times faster
if one uses $M$ processors. For this class of problems running on a
dedicated cluster will not improve performance (but may reduce latency
and such).

In order to fully exploit the functionality offered by Condor one
should submit the jobs into the so called 'standard universe'. To do
this, it is necessary to re-link the application with the
Condor-library (this assumes that Condor is installed)\\ 
\verb+  condor_compile gcc your_object_files -static `globes-config --libs`+\\
It may be necessary to prefix the call of \verb+globes-config+ with
the path to it, in case that this location is not in \verb+$PATH+.
\index{norm}{Condor|)}

\section*{GSL requirements}

Sometimes, the GNU scientific library is not available or is installed
in a non-standard location. This situation can arise in an
installation without root privileges. In this case, one can specify
\verb+--with-gsl-prefix=path_to_gsl+ as option to the \verb+configure+ script.
If one wants to use a shared version of \verb+libgsl+ then one has to make
sure that the linker can find the library at run-time. This can be
achieved by setting the environment variable \verb+LD_LIBRARY_PATH+
correctly, i.e. (in bash)\\
\verb+  export LD_LIBRARY_PATH='path_to_gsl'+\\
You also can use a static version of GSL by either building \GLOBES\
with \verb+LDFLAG='-all-static'+ or by configuring GSL with
\verb+--disable-shared+. In both cases no further actions like setting any
environment variables is necessary.

\section*{Distributions}


\subsection*{RedHat (all versions)}

The standard rpm-based installation of GSL does not provide any header
files for GSL, which are however needed to compile \GLOBES. You have to
install an additional rpm-package called \verb+gsl-devel+. Alternatively
you can install GSL from a tar-ball and use the \verb+--with-gsl-prefix+
option to the configure script of \GLOBES.

\section*{Platforms}


\GLOBES\ builds and installs on 64bit Linux systems.
\GLOBES\ should work on Mac OS.

\subsection*{Windows}


Currently \GLOBES\ is only able to work under Cygwin \verb+www.cygwin.com+.
Inside Cygwin \GLOBES\ needs to be built with these commands\\
\verb+  configure+\\
\verb+  make LDFLAGS=-no-undefined'+

\section*{Compilers and Options}

   Some systems require unusual options for compilation or linking that
the \verb+configure+ script does not know about.  You can give \verb+configure+
initial values for variables by setting them in the environment.  Using
a Bourne-compatible shell, you can do that on the command line like
this\\
\verb+  CC=c89 CFLAGS=-O2 LIBS=-lposix ./configure+\\

Or on systems that have the \verb+env+ program, you can do it like this\\
\verb+  env CPPFLAGS=-I/usr/local/include LDFLAGS=-s ./configure+

\section*{Compiling For Multiple Architectures}


   You can compile the package for more than one kind of computer at the
same time, by placing the object files for each architecture in their
own directory.  To do this, you must use a version of \verb+make+ that
supports the \verb+VPATH+ variable, such as GNU \verb+make+.  \verb+cd+ to the
directory where you want the object files and executables to go and run
the \verb+configure+ script.  \verb+configure+ automatically checks for the
source code in the directory that \verb+configure+ is in and in \verb+..+.

   If you have to use a \verb+make+ that does not supports the \verb+VPATH+
variable, you have to compile the package for one architecture at a time
in the source code directory.  After you have installed the package for
one architecture, use \verb+make distclean+ before reconfiguring for another
architecture.

\section*{Installation Names}


   By default, \verb+make install+ will install the package's files in
\verb+/usr/local/bin+, \verb+/usr/local/man+, etc.  You can specify an
installation prefix other than \verb+/usr/local+ by giving \verb+configure+ the
option \verb+--prefix=PATH+.

   You can specify separate installation prefixes for
architecture-specific files and architecture-independent files.  If you
give \verb+configure+ the option \verb+--exec-prefix=PATH+, the package will use
\verb+PATH+ as the prefix for installing programs and libraries.
Documentation and other data files will still use the regular prefix.

   In addition, if you use an unusual directory layout you can give
options like \verb+--bindir=PATH+ to specify different values for particular
kinds of files.  Run \verb+configure --help+ for a list of the directories
you can set and what kinds of files go in them.

   If the package supports it, you can cause programs to be installed
with an extra prefix or suffix on their names by giving \verb+configure+ the
option \verb+--program-prefix=PREFIX+ or \verb+--program-suffix=SUFFIX+.

\section*{Optional Features}


   Some packages pay attention to \verb+--enable-FEATURE+ options to
\verb+configure+, where \verb+FEATURE+ indicates an optional part of the package.
They may also pay attention to \verb+--with-PACKAGE+ options, where \verb+PACKAGE+
is something like \verb+gnu-as+ or \verb+x+ (for the X Window System).  The
\verb+README+ should mention any \verb+--enable-+ and \verb+--with-+ options that the
package recognizes.

   For packages that use the X Window System, \verb+configure+ can usually
find the X include and library files automatically, but if it doesn't,
you can use the \verb+configure+ options \verb+--x-includes=DIR+ and
\verb+--x-libraries=DIR+ to specify their locations.


\subsection*{Building a perl extension}


This feature is experimental and your mileage may vary!

This feature allows to build a perl binding of \GLOBES\, i.e. you will
in the end have a perl module from which you can use \GLOBES\ from within
any perl program.

If(!) everything works as intended, all you have to do is to provide
\verb+--enable-perl+ to \verb+configure+ and type \verb+make install+. Now have a
look at \verb+globes/example.pl+ and you should see how that works in
principle. 

The trick here is, that we use SWIG (\verb+www.swig.org+) to generate
a wrapper file for \GLOBES. The wrapper file is part of the \GLOBES\
tar-ball (\verb+globes/globes_perl.c+) and hence you should not need SWIG to
be installed on your system.

All the tricks employed to get perl extension working should in some
form be applicable to building other extensions, like python. If
you want to try that you will need SWIG.

\subsection*{Building RPMs}


This feature is experimental and your mileage may vary!

Many people find binary RPMs useful, therefore we provide an optional
feature \verb+--enable-rpm-rules+ which should produce all the necessary
Makefile rules for RPM building. To actually build RPMs requires that
your system is properly setup for that. You can learn how to do that
at \verb+http://www.rpm.org+. You then can use \verb+make rpm+, most likely
you will need to be root to do that (\verb+sudo+ won't work!).

{\bf NOTE} to people packaging \GLOBES\ RPMs: Please, use the provided spec
file and do include the headers!

\section*{Specifying the System Type}


   There may be some features \verb+configure+ can not figure out
automatically, but needs to determine by the type of host the package
will run on.  Usually \verb+configure+ can figure that out, but if it prints
a message saying it can not guess the host type, give it the
\verb+--host=TYPE+ option. \verb+TYPE+ can either be a short name for the system
type, such as \verb+sun4+, or a canonical name with three fields\\
\verb+  CPU-COMPANY-SYSTEM+

See the file \verb+config.sub+ for the possible values of each field.  If
\verb+config.sub+ isn't included in this package, then this package doesn't
need to know the host type.

   If you are building compiler tools for cross-compiling, you can also
use the \verb+--target=TYPE+ option to select the type of system they will
produce code for and the \verb+--build=TYPE+ option to select the type of
system on which you are compiling the package.

\section*{Sharing Defaults}

   If you want to set default values for \verb+configure+ scripts to share,
you can create a site shell script called \verb+config.site+ that gives
default values for variables like \verb+CC+, \verb+cache_file+, and \verb+prefix+.
\verb+configure+ looks for \verb+PREFIX/share/config.site+ if it exists, then
\verb+PREFIX/etc/config.site+ if it exists.  Or, you can set the
\verb+CONFIG_SITE+ environment variable to the location of the site script.
A warning: not all \verb+configure+ scripts look for a site script.


\index{norm}{Installation|)}


%%%%%%%%%%%%%%%%%%%%%%%%%%%%%%%%%%%%%%%%%%%%%
\chapter{Catalogue of {\sf AEDL}-Files}
\label{app:aedlfiles}
Along with the \GLOBES\ package comes a catalogue of pre-defined experiment \AEDL\ files for different
future experiments and different beam and detector technologies. These include the planned superbeam experiments
and their possible upgrades, different reactor experiment setups, different $\beta$-beam setups, and different 
neutrino factory setups. A complete list
of all pre-defined experiment files an be found in \tabl{experiments}. More detailed descriptions of the
corresponding files, the assumptions, requirements, and references are given in the following. 
   
\section*{Superbeam experiments}
\subsection*{T2K -- {\tt T2K.glb}}

The \TtoK\ experiment can be simulated with the file {\tt T2K.glb}. This file tries to approximate as closely as
possible the LOI \cite{Itow:2001ee} and the basic version was used within \cite{Huber:2002mx}. These references
should be cited if the file {\tt T2K.glb} is used for a scientific publication or a talk. For calculations that
involve {\tt T2K.glb}, the following additional files are required: 
\bi
\item {\tt JHFplus.dat} (neutrino flux from J-PARC -- $\nu_\mu$)
\item {\tt JHFminus.dat} (neutrino flux from J-PARC -- $\bar{\nu}_\mu$)
\item {\tt XCC.dat} (charged current cross sections)
\item {\tt XNC.dat} (neutral current cross sections)
\item {\tt XQE.dat} (quasi elastic cross sections)
\ei
The \TtoK\ neutrino beam is produced at J-PARC and directed towards the Super-Kamiokande detector. The target
power is 0.77~MW and 2~years $\nu$-running and 6~years $\bar{\nu}$-running is assumed. The fiducial mass of the
Super-Kamiokande Water Cerenkov detector is taken to be $\mathrm{m_{det} = 22.5 \,kt}$ at a baseline of
L~=~295~km. The appearance measurement involves the total rates data from all CC events and the spectral data
from the QE sample with a free normalization at an energy resolution of $\mathrm{\sigma_e=0.085\, GeV}$ due to the
Fermi Motion as is described in more detail in~\cite{Huber:2002mx}. The treatment of systematics is similar to
\cite{Ishitsuka:2005qi}. The following rules are defined within {\tt T2K.glb}:  
\begin{center}
\begin{tabular}{|l|ll|c|c|}
\hline \hline
\multicolumn{3}{|l|}{Disappearance (+)} & $\sigma_\mathrm{norm}$ & $\sigma_\mathrm{cal}$ \\ \hline 
Signal & $0.9 \, \otimes \, (\nu_\mu\rightarrow\nu_\mu)_{\mathrm{QE}}$ & & 0.025 & $10^{-4}$\\
 & &  & &\\
Background & $0.0056 \, \otimes \, (\nu_\mu \rightarrow \nu_x)_\mathrm{NC}$ & & 0.2 & $10^{-4}$ \\ \hline \hline 
\multicolumn{3}{|l|}{Appearance (+) -- Spectrum}  & & \\ \hline
Signal & $0.505 \, \otimes \, (\nu_\mu \rightarrow \nu_e)_\mathrm{QE}$ & & 10.0 & $10^{-4}$\\
 & & & & \\
Background & $0.0056 \, \otimes \, (\nu_\mu \rightarrow \nu_x)_\mathrm{NC}$ & $3.3\cdot 10^{-4} \, \otimes \, (\nu_\mu\rightarrow\nu_\mu)_{\mathrm{CC}}$ & 0.05 & 0.05\\
Beam background & $0.505 \, \otimes \, (\nu_e\rightarrow \nu_e)_\mathrm{CC}$ & $0.505 \, \otimes \, (\bar{\nu}_e\rightarrow \bar{\nu}_e)_\mathrm{CC}$  & 0.05 & 0.05\\ \hline \hline
\multicolumn{3}{|l|}{Appearance (+) -- Total Rates}  & & \\ \hline
Signal & $0.505 \, \otimes \, (\nu_\mu \rightarrow \nu_e)_\mathrm{CC}$ & & 0.05 & $10^{-4}$\\
 & & & & \\
Background & $0.0056 \, \otimes \, (\nu_\mu \rightarrow \nu_x)_\mathrm{NC}$ & $3.3\cdot 10^{-4} \, \otimes \, (\nu_\mu\rightarrow\nu_\mu)_{\mathrm{CC}}$ & 0.05 & $10^{-4}$\\
Beam background & $0.505 \, \otimes \, (\nu_e\rightarrow \nu_e)_\mathrm{CC}$ & $0.505 \, \otimes \, (\bar{\nu}_e\rightarrow \bar{\nu}_e)_\mathrm{CC}$  & 0.05 & $10^{-4}$\\ \hline \hline
\multicolumn{3}{|l|}{Disappearance (--)} & & \\ \hline 
Signal & $0.9 \, \otimes \, (\nu_\mu\rightarrow\nu_\mu)_{\mathrm{QE}}$ & & 0.025 & $10^{-4}$\\
 & &  & &\\
Background & $0.0056 \, \otimes \, (\nu_\mu \rightarrow \nu_x)_\mathrm{NC}$ & & 0.2 & $10^{-4}$ \\ \hline \hline 
\multicolumn{3}{|l|}{Appearance (--) -- Spectrum}  & & \\ \hline
Signal & $0.505 \, \otimes \, (\bar{\nu}_\mu \rightarrow \bar{\nu}_e)_\mathrm{QE}$ & & 10.0 & $10^{-4}$\\
 & & & & \\
Background & $0.0056 \, \otimes \, (\bar{\nu}_\mu \rightarrow \bar{\nu}_x)_\mathrm{NC}$ & $3.3\cdot 10^{-4} \, \otimes \, (\bar{\nu}_\mu\rightarrow\bar{\nu}_\mu)_{\mathrm{CC}}$ & 0.05 & 0.05\\
Beam background & $0.505 \, \otimes \, (\bar{\nu}_e\rightarrow \bar{\nu}_e)_\mathrm{CC}$ & $0.505 \, \otimes \, (\nu_e\rightarrow \nu_e)_\mathrm{CC}$  & 0.05 & 0.05\\ \hline \hline
\multicolumn{3}{|l|}{Appearance (--) -- Total Rates}  & & \\ \hline
Signal & $0.505 \, \otimes \, (\bar{\nu}_\mu \rightarrow \bar{\nu}_e)_\mathrm{CC}$ & & 0.05 & $10^{-4}$\\
 & & & & \\
Background & $0.0056 \, \otimes \, (\bar{\nu}_\mu \rightarrow \bar{\nu}_x)_\mathrm{NC}$ & $3.3\cdot 10^{-4} \, \otimes \, (\bar{\nu}_\mu\rightarrow\bar{\nu}_\mu)_{\mathrm{CC}}$ & 0.05 & $10^{-4}$\\
Beam background & $0.505 \, \otimes \, (\bar{\nu}_e\rightarrow \bar{\nu}_e)_\mathrm{CC}$ & $0.505 \, \otimes \, (\nu_e\rightarrow \nu_e)_\mathrm{CC}$  & 0.05 & $10^{-4}$\\ \hline \hline
\end{tabular}
\end{center}

\subsection*{T2HK -- {\tt T2HK.glb}}

\TtoHK\ is the superbeam upgrade of the \TtoK\ experiment and can be simulated with the file {\tt T2HK.glb}. 
The target power is 4~MW and 
4~years $\nu$-running and 4~years $\bar{\nu}$-running is assumed.
The fiducial mass of the Water Cerenkov detector is taken to be $\mathrm{m_{det} = 500 \,kt}$ at the same
baseline as the \TtoK\ experiment. Besides these changes, the file {\tt T2HK.glb} is similar to {\tt T2K.glb}
and the same additional files are required. 
The basic version was used within \cite{Huber:2002mx} which should be cited if the file {\tt T2K.glb} is 
used for a scientific publication or a talk. 

\subsection*{NO$\nu$A -- {\tt NOvA.glb}}

The \NOVA\ experiment can be simulated with the file {\tt NOvA.glb}. The description of the disapperance 
channels is taken from \cite{Yang_2004} and the description of the appearance channels follows 
the proposal \cite{Ambats:2004js}. These references
should be cited if the file {\tt NOvA.glb} is used for a scientific publication or a talk. For calculations that
involve {\tt NOvA.glb}, the following additional files are required: 
\bi
\item {\tt NOvAplus.dat} (NuMI neutrino flux -- $\nu_\mu$)
\item {\tt NOvAminus.dat} (NuMI neutrino flux -- $\bar{\nu}_\mu$)
\item {\tt XCC.dat} (charged current cross sections)
\item {\tt XNC.dat} (neutral current cross sections)
\ei
The \NOVA\ experiment uses a neutrino beam from the Fermilab NuMI beamline. The target power is 
1.12~MV which results in $10^{21}$~pot~$\mathrm{yr^{-1}}$ and 3~years $\nu$-running and 3~years $\bar{\nu}$-running 
is assumed. The fiducial mass of the Totally Liquid Scintillator Detector (TASD) is taken to be 
$\mathrm{m_{det} = 25 \,kt}$ at a baseline of L~=~812~km approximately 12~km off-axis to the beamline. The
energy resolution is $\mathrm{\sigma_e=10 \% \cdot \sqrt{E}}$ for electrons and  $\mathrm{\sigma_e=5 \% \cdot \sqrt{E}}$
for muons. The following rules are defined within {\tt NOvA.glb}: 
\begin{center}
\begin{tabular}{|l|ll|c|c|}
\hline \hline
\multicolumn{3}{|l|}{Disappearance (+)} & $\sigma_\mathrm{norm}$ & $\sigma_\mathrm{cal}$ \\ \hline
Signal & $0.8 \, \otimes \, (\nu_\mu\rightarrow\nu_\mu)_{\mathrm{CC}}$ & & 0.05 & 0.025 \\
 & & & & \\
Background & $0.0015 \, \otimes \, (\nu_\mu \rightarrow \nu_x)_\mathrm{NC}$ & & 0.05 & 0.025 \\ \hline \hline 
\multicolumn{3}{|l|}{Appearance (+)} & & \\ \hline
Signal & $0.24 \, \otimes \, (\nu_\mu \rightarrow \nu_e)_\mathrm{CC}$ & & 0.05 & 0.025\\
 & & & & \\
Background & $0.0015 \, \otimes \, (\nu_\mu \rightarrow \nu_x)_\mathrm{NC}$ & $1.0\cdot 10^{-4} \, \otimes \,
(\nu_\mu\rightarrow\nu_\mu)_{\mathrm{CC}}$ & 0.05 & 0.025 \\
Beam background & $0.12 \, \otimes \, (\nu_e\rightarrow \nu_e)_\mathrm{CC}$ & & 0.05 & 0.025 \\ \hline \hline
\multicolumn{3}{|l|}{Disappearance (--)} &  &  \\ \hline
Signal & $0.8 \, \otimes \, (\bar{\nu}_\mu\rightarrow\bar{\nu}_\mu)_{\mathrm{CC}}$ & & 0.05 & 0.025 \\
 & & & & \\
Background & $0.0015 \, \otimes \, (\bar{\nu}_\mu \rightarrow \bar{\nu}_x)_\mathrm{NC}$ & & 0.05 & 0.025 \\ \hline \hline 
\multicolumn{3}{|l|}{Appearance (--)} & & \\ \hline
Signal & $0.37 \, \otimes \, (\bar{\nu}_\mu \rightarrow \bar{\nu}_e)_\mathrm{CC}$ & & 0.05 & 0.025\\
 & & & & \\
Background & $0.0037 \, \otimes \, (\bar{\nu}_\mu \rightarrow \bar{\nu}_x)_\mathrm{NC}$ & $1.0\cdot 10^{-4} \, \otimes \,
(\bar{\nu}_\mu\rightarrow\bar{\nu}_\mu)_{\mathrm{CC}}$ & 0.05 & 0.025 \\
Beam background & $0.12 \, \otimes \, (\bar{\nu}_e\rightarrow \bar{\nu}_e)_\mathrm{CC}$ & & 0.05 & 0.025 \\ \hline \hline
\end{tabular}
\end{center}

\subsection*{SPL - {\tt SPL.glb}}

The \SPL\ experiment can be simulated with the file {\tt SPL.glb}. This file was used in \cite{Campagne:2006yx}
and follows the experiment description from \cite{Mezzetto:2003mm, Campagne:2004wt}. These references
should be cited if the file {\tt SPL.glb} is used for a scientific publication or a talk. For calculations that
involve {\tt SPL.glb}, the following additional files are required: 
\bi
\item {\tt SPLplus.dat} (neutrino flux from CERN -- $\nu_\mu$)
\item {\tt SPLminus.dat} (neutrino flux from CERN -- $\bar{\nu}_\mu$)
\item {\tt Mig\_WC\_numu.dat} (migration matrix -- $\nu_\mu$)
\item {\tt Mig\_WC\_numubar.dat} (migration matrix -- $\bar{\nu}_\mu$)
\item {\tt Mig\_WC\_nue.dat} (migration matrix -- $\nu_e$)
\item {\tt Mig\_WC\_nuebar.dat} (migration matrix -- $\bar{\nu}_e$)
\item {\tt XCC\_spl.dat} (charged current cross sections)
\item {\tt XNC\_spl.dat} (neutral current cross sections)
\ei
The \SPL\ experiment uses a neutrino beam from the CERN to Fr\'{e}jus. The target power is 
4~MV and 2~years $\nu$-running and 8~years $\bar{\nu}$-running 
is assumed. The fiducial mass of the Water Cerenkov detector is taken to be 
$\mathrm{m_{det} = 500 \,kt}$ at a baseline of L~=~130~km. The energy resolution is introduced manually by
four migration matrices (for $\nu_e , \bar{\nu}_e ,\nu_{\mu} ,\bar{\nu}_\mu$) that describe energy smearing mainly
from Fermi Motion. The following rules are defined within {\tt SPL.glb}: 
\begin{center}
\begin{tabular}{|l|ll|c|c|}
\hline \hline
\multicolumn{3}{|l|}{Disappearance (+)} & $\sigma_\mathrm{norm}$ & $\sigma_\mathrm{cal}$ \\ \hline 
Signal & $1.0 \, \otimes \, (\nu_\mu\rightarrow\nu_\mu)_{\mathrm{CC}}$ & (energy dep. efficiency) & 0.02 & $10^{-4}$\\
 & &  & &\\
Background & $4.3\cdot 10^{-5} \, \otimes \, (\nu_\mu \rightarrow \nu_\mu)_\mathrm{CC}$ & & 0.02 & $10^{-4}$ \\ \hline \hline 
\multicolumn{3}{|l|}{Appearance (+)} &  & \\ \hline
Signal & $0.707 \, \otimes \, (\nu_\mu\rightarrow \nu_e)_\mathrm{CC}$ & & 0.02 & $10^{-4}$\\
&&&&\\
Background & $6.5\cdot 10^{-4} \, \otimes \, (\nu_\mu \rightarrow \nu_x)_\mathrm{NC}$ & $5.4\cdot 10^{-4} \, \otimes \, (\nu_\mu \rightarrow \nu_\mu)_\mathrm{CC}$ & 0.02 & $10^{-4}$ \\
 & $0.7 \, \otimes \, (\bar{\nu}_\mu \rightarrow \bar{\nu}_e )_\mathrm{CC}$ & & 0.02 &  $10^{-4}$ \\
Beam Background & $0.677 \, \otimes \, (\bar{\nu}_e \rightarrow \bar{\nu}_e )_\mathrm{CC}$ & $0.707 \, \otimes \, (\nu_e \rightarrow \nu_e )_\mathrm{CC}$ & 0.02 & $10^{-4}$ \\ \hline \hline
\end{tabular}
\end{center}
\begin{center}
\begin{tabular}{|l|ll|c|c|}
\hline \hline
\multicolumn{3}{|l|}{Disappearance (--)} & & \\ \hline 
Signal & $1.0 \, \otimes \, (\bar{\nu}_\mu\rightarrow\bar{\nu}_\mu)_{\mathrm{CC}}$ & (energy dep. efficiency) & 0.02 & $10^{-4}$\\
 & &  & &\\
Background & $4.3\cdot 10^{-5} \, \otimes \, (\bar{\nu}_\mu \rightarrow \bar{\nu}_\mu)_\mathrm{CC}$ & & 0.02 & $10^{-4}$ \\ \hline \hline 
\multicolumn{3}{|l|}{Appearance (--)} &  & \\ \hline
Signal & $0.677 \, \otimes \, (\bar{\nu}_\mu\rightarrow \bar{\nu}_e)_\mathrm{CC}$ & & 0.02 & $10^{-4}$\\
&&&&\\
Background & $0.0025 \, \otimes \, (\bar{\nu}_\mu \rightarrow \bar{\nu}_x)_\mathrm{NC}$ & $5.4\cdot 10^{-4} \, \otimes \, (\bar{\nu}_\mu \rightarrow \bar{\nu}_\mu)_\mathrm{CC}$ & 0.02 & $10^{-4}$ \\
 & $0.7 \, \otimes \, (\nu_\mu \rightarrow \nu_e )_\mathrm{CC}$ & & 0.02 &  $10^{-4}$ \\
Beam Background & $0.677 \, \otimes \, (\bar{\nu}_e \rightarrow \bar{\nu}_e )_\mathrm{CC}$ & $0.707 \, \otimes \, (\nu_e \rightarrow \nu_e )_\mathrm{CC}$ & 0.02 & $10^{-4}$ \\ \hline \hline
\end{tabular}
\end{center}

\section*{Reactor experiments}
\subsection*{Small reactor experiment -- {\tt Reactor1.glb}}

The file {\tt Reactor1.glb} allows to simulate a small $\bar{\nu}_e$-disappearance reactor experiment. The basic
version of this file was used within \cite{Huber:2003pm} which should be cited if the file {\tt Reactor1.glb} is
used for a scientific publication or a talk. For calculations that involve {\tt Reactor1.glb}, the following
additional files are required:
\bi
\item {\tt Reactor.dat} (neutrino flux from reactor)
\item {\tt XCCreactor.dat} (charged current cross sections for low energies)
\ei
The neutrino source is the core of a nuclear power reactor. The integrated luminosity is assumed to be
$\mathrm{\mathcal{L} = 400 \,t \, GW\, yr}$, \eg\ a 20~t detector, a reactor with a thermal power of 4~GW
and a running period of 5 years. As detector technology a liquid scintillator 
detector is assumed, a far detector at a baseline of L~=~1.7~km and a near detector which is
assumed to be identical to the far detector (maybe apart from the size) in order to minimize the impact of systematical uncertainties. The normalization error
used in the file {\tt Reactor1.glb} has to be considered as an effective error, receiving contributions from
individual uncertainties (see~\cite{Huber:2003pm}). The energy resolution is
$\mathrm{\sigma_e=5\%\cdot\sqrt{E_{vis}}}$ and the choice for {\tt sigma\_function} is {\tt \#inverse\_beta}. The following rules are defined within {\tt Reactor1.glb}: 
\begin{center}
\begin{tabular}{|l|ll|c|c|}
\hline \hline
\multicolumn{3}{|l|}{Disappearance} & $\sigma_\mathrm{norm}$ & $\sigma_\mathrm{cal}$ \\ \hline 
Signal & $1.0 \, \otimes \, (\bar{\nu}_e\rightarrow\bar{\nu}_e)_{\mathrm{CC}}$ & \hspace{5.5cm} & 0.008 & 0.005\\
 & &  & &\\
Background & $5.8\cdot 10^{-5} \, \otimes \, (\bar{\nu}_e\rightarrow\bar{\nu}_e)_\mathrm{CC}$ & \hspace{5.5cm} &
$10^{-6}$ & $10^{-6}$ \\ \hline \hline 
\end{tabular}
\end{center}

\subsection*{Large reactor experiment -- {\tt Reactor2.glb}}

The file {\tt Reactor2.glb} allows to simulate a large $\bar{\nu}_e$-disappearance reactor experiment. The basic
version of this file was used within \cite{Huber:2003pm} which should be cited if the file {\tt Reactor2.glb} is
used for a scientific publication or a talk. The integrated luminosity is assumed to be
$\mathrm{\mathcal{L} = 8000 \,t \, GW\, yr}$, \eg\ a 100~t detector, a reactor with a thermal power of 10~GW
and a running period of 8 years. Besides the higher integrated luminosity, the attributes of {\tt Reactor2.glb}
are similar to the ones of {\tt Reactor1.glb}.

\subsection*{DoubleCHOOZ -- {\tt D-Chooz\_near.glb} and {\tt D-Chooz\_far.glb}}

The files {\tt D-Chooz\_near.glb} and {\tt D-Chooz\_far.glb} allow to simulate the \DC\ reactor experiment in France. 
The basic version of these files were used within \cite{Huber:2006vr} which should be cited if the files 
{\tt D-Chooz\_near.glb} and {\tt D-Chooz\_far.glb} are
used for a scientific publication or a talk. For calculations that involve {\tt D-Chooz\_near.glb} and/or {\tt D-Chooz\_far.glb}, the following
additional files are required:
\bi
\item {\tt Reactor.dat} (neutrino flux from reactor)
\item {\tt XCCreactor.dat} (charged current cross sections for low energies)
\ei
The \DC\ experiment is located at the Chooz reactor complex and the two reactor cores serve as $\bar{\nu}_e$
neutrino source, so the thermal power is $\mathrm{2\cdot4.2\,GW}$. Two identical liquid scintillator detectors
with a fiducial mass of $\mathrm{m_{det} = 10.16 \,t}$ are used as near and far detector. The far detector
is planned to be located in the {\sc Chooz} cavern at a baseline of L~=~1.05~km from the two reactor cores and
the near detector is assumed to be located at a distance of 0.1~km to the cores. The total running time of the
experiment is assumed to be 5 years. So, the integrated luminosity at the far detector yields $\mathrm{\mathcal{L} \approx
427 \,t \, GW\, yr}$.
Here, the total running time of 5 years is assumed within {\tt D-Chooz\_near.glb} and {\tt D-Chooz\_far.glb}, so near
and far detector are assumed to start the mode of operation simultaneously. For the simulation of \DC\ and
considering a delayed start of data taking at the near detector the file {\tt D-Chooz\_near.glb} has to be
modified. The cancellation of systematical uncertainties is considered by the manual definition of a $\chi^2$
as described in \cite{Huber:2006vr} with the treatment of user-defined systematics as described in \Sec~\ref{sec:userchi}   
The energy resolution is
$\mathrm{\sigma_e=5\%\cdot\sqrt{E_{vis}}}$ and the choice for {\tt sigma\_function} is {\tt \#inverse\_beta}. The following rules are defined within {\tt D-Chooz\_near.glb} and {\tt D-Chooz\_far.glb}: 
\begin{center}
\begin{tabular}{|l|ll|c|c|c|c|c|}
\hline \hline
\multicolumn{3}{|l|}{Disappearance} & $\sigma_\mathrm{flux}$ & $\sigma^N_\mathrm{fid}$ & $\sigma^F_\mathrm{fid}$ &
$\sigma^N_\mathrm{cal}$ & $\sigma^F_\mathrm{cal}$ \\ \hline 
Signal & $1.0 \, \otimes \, (\bar{\nu}_e\rightarrow\bar{\nu}_e)_{\mathrm{CC}}$ & & 0.02 & 0.006 & 0.006 & 0.005 &
0.005 \\
 & & & & & & & \\
Background & neglected & (sys. dominates) & -- & -- & -- & -- & -- \\ \hline \hline 
\end{tabular}
\end{center}

\section*{Beta beam experiments}
\subsection*{CERN-Fr\'{e}jus baseline scenario -- {\tt BB\_100.glb}}

The $\gamma=100$ $\beta$-beam baseline scenario from CERN to Fr\'{e}jus can be simulated with the file {\tt
BB\_100.glb}. The basic version of this file was used within \cite{Campagne:2006yx}. This reference
should be cited if the file {\tt BB\_100.glb} is used for a scientific publication or a talk. For calculations that
involve {\tt BB\_100.glb}, the following additional files are required: 
\bi
\item {\tt BB100flux\_Ne.dat} ($\beta$-beam neutrino flux -- $^{18}$Ne stored at $\gamma=100$)
\item {\tt BB100flux\_He.dat} ($\beta$-beam neutrino flux -- $^{6}$He stored at $\gamma=100$)
\item {\tt BeamBckg\_100.dat} (beam background)
\item {\tt AtmBckg\_100.dat} (atmospheric background)
\item {\tt Mig\_WC\_numu.dat} (migration matrix -- $\nu_\mu$)
\item {\tt Mig\_WC\_numubar.dat} (migration matrix -- $\bar{\nu}_\mu$)
\item {\tt Mig\_WC\_nue.dat} (migration matrix -- $\nu_e$)
\item {\tt Mig\_WC\_nuebar.dat} (migration matrix -- $\bar{\nu}_e$)
\item {\tt XCC\_Nuance.dat} (charged current cross sections)
\item {\tt XNC\_Nuance.dat} (neutral current cross sections)
\item {\tt Null.dat} (auxiliary file)
\ei
The neutrino beam is produced at CERN and directed towards a a megaton Water Cerenkov detector at Fr\'{e}jus. 
The neutrinos originate from the decays of accelerated isotopes $^{18}$Ne ($\nu_e$) and $^6$He ($\bar{\nu}_e$).
The acceleration factor is $\gamma=100$ for both types of isotopes and $2.2\cdot10^{18}$ $^{18}$Ne decays per
year and $5.8\cdot10^{18}$ $^{6}$He decays per year are assumed. The CERN-Fr\'{e}jus baseline is L~=~130~km, the
fiducial mass of the detector is $\mathrm{m_{det} = 500 \,kt}$ and 4~years $\nu$-running and 4~years
$\bar{\nu}$-running are assumed. The energy resolution is introduced manually by
four migration matrices (for $\nu_e , \bar{\nu}_e ,\nu_{\mu} ,\bar{\nu}_\mu$) that describes energy smearing mainly
from Fermi Motion. The following rules are defined within {\tt BB\_100.glb}:
\begin{center}
\begin{tabular}{|l|ll|c|c|}
\hline \hline
\multicolumn{3}{|l|}{Disappearance -- $^{18}$Ne stored} & $\sigma_\mathrm{norm}$ & $\sigma_\mathrm{cal}$ \\ \hline
Signal & $0.707 \, \otimes \, (\nu_e\rightarrow\nu_e)_{\mathrm{CC}}$ & & 0.02 & $10^{-4}$ \\
 & & & & \\
Background & $4.3\cdot 10^{-5} \, \otimes \, (\nu_e \rightarrow \nu_e)_\mathrm{CC}$ & & 0.02 & $10^{-4}$ \\ \hline \hline 
\multicolumn{3}{|l|}{Appearance -- $^{18}$Ne stored} & & \\ \hline
Signal & $1.0 \, \otimes \, (\nu_e \rightarrow \nu_\mu)_\mathrm{CC}$ & (energy dep. efficiency) & 0.02 & $10^{-4}$ \\
 & & & & \\
Background & $1.0 \, \otimes \, (\nu_e \rightarrow \nu_x)_\mathrm{NC}$ & (from external file) & 0.02 & $10^{-4}$ \\
Atm. background & & (from external file) & 0.02 & $10^{-4}$ \\ \hline \hline
\end{tabular}
\end{center}
\begin{center}
\begin{tabular}{|l|ll|c|c|}
\hline \hline
\multicolumn{3}{|l|}{Disappearance -- $^6$He stored} & &  \\ \hline
Signal & $0.677 \, \otimes \, (\bar{\nu}_e\rightarrow\bar{\nu}_e)_{\mathrm{CC}}$ & & 0.02 & $10^{-4}$ \\
 & & & & \\
Background & $4.3\cdot 10^{-5} \, \otimes \, (\bar{\nu}_e \rightarrow \nu_e)_\mathrm{CC}$ & & 0.02 & $10^{-4}$ \\ \hline \hline 
\multicolumn{3}{|l|}{Appearance -- $^6$He stored} & & \\ \hline
Signal & $1.0 \, \otimes \, (\bar{\nu}_e \rightarrow \bar{\nu}_\mu)_\mathrm{CC}$ & (energy dep. efficiency) & 0.02 & $10^{-4}$ \\
 & & & & \\
Background & $1.0 \, \otimes \, (\bar{\nu}_e \rightarrow \bar{\nu}_x)_\mathrm{NC}$ & (from external file) & 0.02 & $10^{-4}$ \\
Atm. background & & (from external file) & 0.02 & $10^{-4}$ \\ \hline \hline
\end{tabular}
\end{center}

\subsection*{Higher gamma scenario -- {\tt BB\_350.glb}}

A $\gamma=350$ medium gamma $\beta$-beam green-field scenario involving a megaton Water Cerenkov detector can be simulated with the file {\tt
BB\_350.glb}. This file tries to approximate as closely as possible the scenario {\sf Setup III} from
\cite{Burguet-Castell:2005pa}. This reference
should be cited if the file {\tt BB\_350.glb} is used for a scientific publication or a talk. For calculations that
involve {\tt BB\_350.glb}, the following additional files are required: 
\bi
\item {\tt BB350flux.dat} ($\beta$-beam neutrino flux -- $\gamma=350$ )
\item {\tt NeEffMig350.dat} (migration matrix -- $\nu_\mu$)
\item {\tt NeBckgRej350.dat} (migration matrix -- background)
\item {\tt NeDisEff350.dat} (migration matrix -- $\nu_e$)
\item {\tt HeEffMig350.dat} (migration matrix -- $\bar{\nu}_\mu$)
\item {\tt HeBckgRej350.dat} (migration matrix -- background)
\item {\tt HeDisEff350.dat} (migration matrix -- $\bar{\nu}_e$)
\item {\tt XCC.dat} (charged current cross sections)
\item {\tt XNC.dat} (neutral current cross sections)
\ei
The neutrinos originate from the decays of accelerated isotopes $^{18}$Ne ($\nu_e$) and $^6$He ($\bar{\nu}_e$).
The acceleration factor is $\gamma=350$ for both types of isotopes and $2.2\cdot10^{18}$ $^{18}$Ne decays per
year and $5.8\cdot10^{18}$ $^{6}$He decays per year are assumed. The baseline is L~=~730~km, the
fiducial mass of the detector is $\mathrm{m_{det} = 500 \,kt}$ and 4~years $\nu$-running and 4~years
$\bar{\nu}$-running are assumed. The energy resolution is introduced manually by
six migration matrices (for $\nu_e , \bar{\nu}_e ,\nu_{\mu} ,\bar{\nu}_\mu$, and the background from NC events
for $^{18}$Ne and $^6$He) that describes energy smearing. These migration matrices also already include energy
dependend efficiencies and background rejection factors. They are taken from the appendix of
\cite{Burguet-Castell:2005pa}. The following rules are defined within {\tt BB\_350.glb}:
\begin{center}
\begin{tabular}{|l|ll|c|c|}
\hline \hline
\multicolumn{3}{|l|}{Disappearance -- $^{18}$Ne stored} & $\sigma_\mathrm{norm}$ & $\sigma_\mathrm{cal}$ \\ \hline
Signal & $1.0 \, \otimes \, (\nu_e\rightarrow\nu_e)_{\mathrm{CC}}$ & (migration matrix) & 0.025 & $10^{-4}$ \\
 & & & & \\
Background & neglected & (systematic uncertainty dominates) & -- & -- \\ \hline \hline 
\multicolumn{3}{|l|}{Appearance -- $^{18}$Ne stored} & & \\ \hline
Signal & $1.0 \, \otimes \, (\nu_e \rightarrow \nu_\mu)_\mathrm{CC}$ & (migration matrix) & 0.025 & $10^{-4}$ \\
 & & & & \\
Background & $1.0 \, \otimes \, (\nu_e \rightarrow \nu_x)_\mathrm{NC}$ & (migration matrix) & 0.05 & $10^{-4}$
\\ \hline \hline
\multicolumn{3}{|l|}{Disappearance -- $^6$He stored} & &  \\ \hline
Signal & $1.0 \, \otimes \, (\bar{\nu}_e\rightarrow\bar{\nu}_e)_{\mathrm{CC}}$ & (migration matrix) & 0.025 & $10^{-4}$ \\
 & & & & \\
Background & neglected & (systematic uncertainty dominates) & -- & -- \\ \hline \hline 
\multicolumn{3}{|l|}{Appearance -- $^6$He stored} & & \\ \hline
Signal & $1.0 \, \otimes \, (\bar{\nu}_e \rightarrow \bar{\nu}_\mu)_\mathrm{CC}$ & (migration matrix)  &
0.025 & $10^{-4}$ \\
 & & & & \\
Background & $1.0 \, \otimes \, (\bar{\nu}_e \rightarrow \bar{\nu}_x)_\mathrm{NC}$ & (migration matrix) & 0.05 & $10^{-4}$ \\ \hline \hline
\end{tabular}
\end{center}

\subsection*{Variable beta beam (Water Cerenkov) -- {\tt BBvar\_WC.glb}}

A variable $\beta$-beam scenario involving a megaton Water Cerenkov detector can be simulated with the file 
{\tt BBvar\_WC.glb}. The basic file was used within \cite{Huber:2005jk}. This reference
should be cited if the file {\tt BBvar\_WC.glb} is used for a scientific publication or a talk. For calculations that
involve {\tt BBvar\_WC.glb}, the following additional files are required:
\bi
\item {\tt BckgMig\_var.dat} (migration matrix -- background)
\item {\tt XCC.dat} (charged current cross sections)
\item {\tt XNC.dat} (neutral current cross sections)
\item {\tt XQE.dat} (quasi elastic cross sections)
\ei
and the values of the following {\sf AEDL}-Variables have to be set:
\bi
\item {\tt gammafactor} (acceleration factor $\gamma$)
\item {\tt EXP\_FACTOR} (parameter of ion decay scaling)
\item {\tt baselinefactor} (baseline parameter L/$\gamma\,\left[\mathrm{km}\right]$)
\ei
The neutrinos originate from the decays of accelerated isotopes $^{18}$Ne ($\nu_e$) and $^6$He ($\bar{\nu}_e$).
The acceleration factor is $\gamma=${\tt gammafactor} for both types of isotopes and
$(100/\gamma)^\alpha\cdot2.2\cdot10^{18}$ $^{18}$Ne decays per
year and $(60/\gamma)^\alpha\cdot5.8\cdot10^{18}$ $^{6}$He decays per year are assumed where $\alpha$~=~{\tt
EXP\_FACTOR} is a parameter that describes ion decay scaling. As default value {\tt EXP\_FACTOR=0} should be chosen.
See \cite{Huber:2005jk} for a detailed discussion of this parameter. The baseline is
L~=~{\tt baselinefactor}$\cdot\gamma$~km, the
fiducial mass of the detector is $\mathrm{m_{det} = 500 \,kt}$ and 4~years $\nu$-running and 4~years
$\bar{\nu}$-running are assumed. The appearance measurement involves the total rates data from all CC events and the spectral data
from the QE sample with a free normalization at an energy resolution of $\mathrm{\sigma_e=0.085GeV}$ due to the
Fermi Motion identical to the treatment of systematics within the \TtoK\ and \TtoHK\ files. The following rules are defined within {\tt BBvar\_WC.glb}:
\begin{center}
\begin{tabular}{|l|ll|c|c|}
\hline \hline
\multicolumn{3}{|l|}{Disappearance -- $^{18}$Ne stored} & $\sigma_\mathrm{norm}$ & $\sigma_\mathrm{cal}$ \\ \hline
Signal & $0.55 \, \otimes \, (\nu_e\rightarrow\nu_e)_{\mathrm{QE}}$ & \hspace{6cm} & 0.025 & $10^{-4}$ \\
 & & & & \\
Background & $0.003 \, \otimes \, (\nu_e \rightarrow \nu_x)_\mathrm{NC}$ & & 0.05 & $10^{-4}$ \\ \hline \hline 
\multicolumn{3}{|l|}{Appearance -- $^{18}$Ne stored -- Spectrum} & & \\ \hline
Signal & $0.55 \, \otimes \, (\nu_e \rightarrow \nu_\mu)_\mathrm{QE}$ & & 10.0 & $10^{-4}$ \\
 & & & & \\
Background & $0.003 \, \otimes \, (\nu_e \rightarrow \nu_x)_\mathrm{NC}$ & & 0.05 & $10^{-4}$
\\ \hline \hline
\multicolumn{3}{|l|}{Appearance -- $^{18}$Ne stored -- Total Rates} & & \\ \hline
Signal & $0.55 \, \otimes \, (\nu_e \rightarrow \nu_\mu)_\mathrm{CC}$ & & 0.025 & $10^{-4}$ \\
 & & & & \\
Background & $0.003 \, \otimes \, (\nu_e \rightarrow \nu_x)_\mathrm{NC}$ & & 0.05 & $10^{-4}$
\\ \hline \hline\multicolumn{3}{|l|}{Disappearance -- $^6$He stored} & &  \\ \hline
Signal & $0.75 \, \otimes \, (\bar{\nu}_e\rightarrow\bar{\nu}_e)_{\mathrm{QE}}$ & & 0.025 & $10^{-4}$ \\
 & & & & \\
Background & $0.0025 \, \otimes \, (\bar{\nu}_e \rightarrow \bar{\nu}_x)_\mathrm{NC}$ & & 0.05 & $10^{-4}$ \\ \hline \hline 
\multicolumn{3}{|l|}{Appearance -- $^6$He stored -- Spectrum} & & \\ \hline
Signal & $0.75 \, \otimes \, (\bar{\nu}_e \rightarrow \bar{\nu}_\mu)_\mathrm{QE}$ & &
10.0 & $10^{-4}$ \\
 & & & & \\
Background & $0.0025 \, \otimes \, (\bar{\nu}_e \rightarrow \bar{\nu}_x)_\mathrm{NC}$ & & 0.05 & $10^{-4}$ \\ \hline \hline
\multicolumn{3}{|l|}{Appearance -- $^6$He stored -- Total Rates} & & \\ \hline
Signal & $0.75 \, \otimes \, (\bar{\nu}_e \rightarrow \bar{\nu}_\mu)_\mathrm{CC}$ & &
0.025 & $10^{-4}$ \\
 & & & & \\
Background & $0.0025 \, \otimes \, (\bar{\nu}_e \rightarrow \bar{\nu}_x)_\mathrm{NC}$ & & 0.05 & $10^{-4}$ \\ \hline \hline
\end{tabular}
\end{center}

\subsection*{Variable beta beam (TASD) -- {\tt BBvar\_TASD.glb}}

A variable $\beta$-beam scenario involving a \NOVA-like TASD detector can be simulated with the file 
{\tt BBvar\_TASD.glb}. The basic file was used within \cite{Huber:2005jk}. This reference
should be cited if the file {\tt BBvar\_TASD.glb} is used for a scientific publication or a talk. For calculations that
involve {\tt BBvar\_TASD.glb}, the following additional files are required:
\bi
\item {\tt XCC.dat} (charged current cross sections)
\item {\tt XNC.dat} (neutral current cross sections)
\ei
and the values of the following {\sf AEDL}-Variables have to be set:
\bi
\item {\tt gammafactor} (acceleration factor $\gamma$)
\item {\tt EXP\_FACTOR} (parameter of ion decay scaling)
\item {\tt baselinefactor} (baseline parameter L/$\gamma\,\left[\mathrm{km}\right]$)
\ei
The neutrinos originate from the decays of accelerated isotopes $^{18}$Ne ($\nu_e$) and $^6$He ($\bar{\nu}_e$).
The acceleration factor is $\gamma=${\tt gammafactor} for both types of isotopes and
$(100/\gamma)^\alpha\cdot2.2\cdot10^{18}$ $^{18}$Ne decays per
year and $(60/\gamma)^\alpha\cdot5.8\cdot10^{18}$ $^{6}$He decays per year are assumed where $\alpha$~=~{\tt
EXP\_FACTOR} is a parameter that describes ion decay scaling. As default value {\tt EXP\_FACTOR=0} should be chosen.
See \cite{Huber:2005jk} for a detailed discussion of this parameter. The baseline is L~=~{\tt baselinefactor}$\cdot\gamma$~km, the
fiducial mass of the detector is $\mathrm{m_{det} = 50 \,kt}$ and 4~years $\nu$-running and 4~years
$\bar{\nu}$-running are assumed. The energy resolution is $\mathrm{\sigma_e=6 \% \cdot \sqrt{E}}$ for electrons and  $\mathrm{\sigma_e=3 \% \cdot \sqrt{E}}$
for muons.  The following rules are defined within {\tt BBvar\_TASD.glb}:
\begin{center}
\begin{tabular}{|l|ll|c|c|}
\hline \hline
\multicolumn{3}{|l|}{Disappearance -- $^{18}$Ne stored} & $\sigma_\mathrm{norm}$ & $\sigma_\mathrm{cal}$ \\ \hline
Signal & $0.2 \, \otimes \, (\nu_e\rightarrow\nu_e)_{\mathrm{CC}}$ & \hspace{6cm} & 0.025 & $10^{-4}$ \\
 & & & & \\
Background & $0.001 \, \otimes \, (\nu_e \rightarrow \nu_x)_\mathrm{NC}$ & & 0.05 & $10^{-4}$ \\ \hline \hline 
\multicolumn{3}{|l|}{Appearance -- $^{18}$Ne stored} & & \\ \hline
Signal & $0.8 \, \otimes \, (\nu_e \rightarrow \nu_\mu)_\mathrm{CC}$ & & 0.025 & $10^{-4}$ \\
 & & & & \\
Background & $0.001 \, \otimes \, (\nu_e \rightarrow \nu_x)_\mathrm{NC}$ & & 0.05 & $10^{-4}$
\\ \hline \hline
\multicolumn{3}{|l|}{Disappearance -- $^6$He stored} & &  \\ \hline
Signal & $0.2 \, \otimes \, (\bar{\nu}_e\rightarrow\bar{\nu}_e)_{\mathrm{CC}}$ & & 0.025 & $10^{-4}$ \\
 & & & & \\
Background & $0.001 \, \otimes \, (\bar{\nu}_e \rightarrow \bar{\nu}_x)_\mathrm{NC}$ & & 0.05 & $10^{-4}$  \\ \hline \hline 
\multicolumn{3}{|l|}{Appearance -- $^6$He stored} & & \\ \hline
Signal & $0.8 \, \otimes \, (\bar{\nu}_e \rightarrow \bar{\nu}_\mu)_\mathrm{CC}$ & &
0.025 & $10^{-4}$ \\
 & & & & \\
Background & $0.001 \, \otimes \, (\bar{\nu}_e \rightarrow \bar{\nu}_x)_\mathrm{NC}$ & & 0.05 & $10^{-4}$ \\ \hline \hline
\end{tabular}
\end{center}

\section*{Neutrino factory experiments}
\subsection*{Standard neutrino factory -- {\tt NFstandard.glb}}

A standard neutrino factory scenario can be simulated with the file {\tt NFstandard.glb}. The basic version was
used within \cite{Huber:2002mx} (NuFact-II scenario). This reference should be cited if the file {\tt
NFstandard.glb} is used for a scientific publication or a talk. For calculations that involve {\tt
NFstandard.glb}, the following additional files are required:
\bi
\item {\tt XCC.dat} (charged current cross sections)
\item {\tt XNC.dat} (neutral current cross sections)
\ei
The neutrino beam is produced by the decay of muons stored in a storage ring at a parent energy of
$\mathrm{E_\mu = 50\,GeV}$. A number of $1.06\cdot10^{21}$ useful muon decays per year are assumed in each
polarity (corresponding to $5.3\cdot10^{20}$ useful muon decays per year and polarity for simultaneous operation
with both polarities) and 4~years $\nu$-running and 4~years $\bar{\nu}$-running is
assumed. The fiducial mass of the MID detector is taken to be $\mathrm{m_{det} = 50 \,kt}$ at a baseline of
L~=~3000~km. The energy resolution is $\mathrm{\sigma_e=15\%\cdot E}$. The following rules are defined within
{\tt NFstandard.glb}:
\begin{center}
\begin{tabular}{|l|ll|c|c|}
\hline \hline
\multicolumn{3}{|l|}{Disappearance -- $\mu^+$-stored} & $\sigma_\mathrm{norm}$ & $\sigma_\mathrm{cal}$ \\ \hline
Signal & $0.35 \, \otimes \, (\bar{\nu}_\mu \rightarrow \bar{\nu}_\mu)_\mathrm{CC}$ & & 0.025 & $10^{-4}$ \\
 & & & & \\
Background & $1.0\cdot 10^{-5} \, \otimes \, (\bar{\nu}_\mu \rightarrow \bar{\nu}_x)_\mathrm{NC}$ & & 0.2 & $10^{-4}$ \\ \hline \hline
\multicolumn{3}{|l|}{Appearance -- $\mu^+$-stored} & & \\ \hline
Signal &  $0.45 \, \otimes \, (\nu_e \rightarrow \nu_\mu)_\mathrm{CC}$ & & 0.025 & $10^{-4}$ \\
 & & & & \\
Background &  $5.0\cdot 10^{-6} \, \otimes \, (\bar{\nu}_\mu \rightarrow \bar{\nu}_x)_\mathrm{NC}$ &  $5.0\cdot
10^{-6} \, \otimes \, (\bar{\nu}_\mu \rightarrow\bar{\nu}_\mu)_\mathrm{CC}$ & 0.2& $10^{-4}$\\ \hline \hline
\multicolumn{3}{|l|}{Disappearance -- $\mu^-$-stored} & & \\ \hline
Signal &  $0.45 \, \otimes \, (\nu_\mu \rightarrow \nu_\mu)_\mathrm{CC}$ & & 0.025& $10^{-4}$\\
 & & & & \\
Background &  $1.0\cdot 10^{-5} \, \otimes \, (\nu_\mu \rightarrow \nu_x)_\mathrm{NC}$ & & 0.2& $10^{-4}$\\ \hline \hline
\multicolumn{3}{|l|}{Appearance -- $\mu^-$-stored} & & \\ \hline
Signal & $0.35 \, \otimes \, (\bar{\nu}_e \rightarrow \bar{\nu}_\mu)_\mathrm{CC}$  & & 0.025& $10^{-4}$\\
 & & & & \\
Background &  $5.0\cdot 10^{-6} \, \otimes \, (\nu_\mu \rightarrow \nu_x)_\mathrm{NC}$ & $5.0\cdot 10^{-6} \, \otimes \, (\nu_\mu \rightarrow
\nu_\mu)_\mathrm{CC}$  & 0.2& $10^{-4}$\\ \hline \hline
\end{tabular}
\end{center}

\subsection*{Variable neutrino factory -- {\tt NFvar.glb}}

A variable neutrino factory scenario can be simulated with the file {\tt NFvar.glb}. The basic version was
used within \cite{Huber:2006wb} and follows the neutrino factory scenarios from \cite{Huber:2002mx}. These
references should be cited if the file {\tt
NFvar.glb} is used for a scientific publication or a talk. For calculations that involve {\tt
NFvar.glb}, the following additional files are required:
\bi
\item {\tt ThresholdEnergies.dat} (list of energies)
\item {\tt CIDthreshold.dat} (list for CID threshold)
\item {\tt MINOSthreshold.dat} (list for MINOS threshold)
\item {\tt XCC.dat} (charged current cross sections)
\item {\tt XNC.dat} (neutral current cross sections)
\ei
and the values of the following {\sf AEDL}-Variables have to be set:
\bi
\item {\tt emax} (parent energy of the stored muons $\left[\mathrm{km}\right]$)
\item {\tt BASELINE} (experiment baseline $\left[\mathrm{GeV}\right]$)
\ei
The neutrino beam is produced by the decay of muons stored in a storage ring at a parent energy of
$\mathrm{E_\mu =}$~{\tt emax}. The parent energy of the muons can be set in the range
$\mathrm{10\,GeV\lesssim E_\mu\lesssim 80\, GeV}$. The baseline of the scenario is L~=~{\tt BASELINE}. 
Besides these settings the other attributes of {\tt NFvar.glb} are similar to the ones from {\tt
NFstandard.glb}. Only the treatment of the disappearance channels is different. Whereas the CID threshold is
also applied in the disappearance channels of {\tt NFstandard.glb} this is not the case for the disapperance
channel definition within {\tt NFvar.glb} where no CID is applied and a threshold similar to the threshold of
the MINOS experiment \cite{Ables:1995wq} is applied. The following rules are defined within
{\tt NFvar.glb}:
\begin{center}
\begin{tabular}{|l|ll|c|c|}
\hline \hline
\multicolumn{3}{|l|}{Disappearance -- $\mu^+$-stored} & $\sigma_\mathrm{norm}$ & $\sigma_\mathrm{cal}$ \\ \hline
Signal & $0.9 \, \otimes \, (\bar{\nu}_\mu \rightarrow \bar{\nu}_\mu)_\mathrm{CC}$ & $0.9 \, \otimes \, (\nu_e \rightarrow \nu_\mu)_\mathrm{CC}$& 0.025 & $10^{-4}$ \\
 & & & & \\
Background & $1.0\cdot 10^{-5} \, \otimes \, (\bar{\nu}_\mu \rightarrow \bar{\nu}_x)_\mathrm{NC}$ & & 0.2 & $10^{-4}$ \\ \hline \hline
\multicolumn{3}{|l|}{Appearance -- $\mu^+$-stored} & & \\ \hline
Signal &  $0.45 \, \otimes \, (\nu_e \rightarrow \nu_\mu)_\mathrm{CC}$ & & 0.025 & $10^{-4}$ \\
 & & & & \\
Background &  $5.0\cdot 10^{-6} \, \otimes \, (\bar{\nu}_\mu \rightarrow \bar{\nu}_x)_\mathrm{NC}$ &  $5.0\cdot
10^{-6} \, \otimes \, (\bar{\nu}_\mu \rightarrow\bar{\nu}_\mu)_\mathrm{CC}$ & 0.2& $10^{-4}$\\ \hline \hline
\multicolumn{3}{|l|}{Disappearance -- $\mu^-$-stored} & & \\ \hline
Signal &  $0.9 \, \otimes \, (\nu_\mu \rightarrow \nu_\mu)_\mathrm{CC}$ & $0.9 \, \otimes \, (\bar{\nu}_e
\rightarrow \bar{\nu}_\mu)_\mathrm{CC}$& 0.025& $10^{-4}$\\
 & & & & \\
Background &  $1.0\cdot 10^{-5} \, \otimes \, (\nu_\mu \rightarrow \nu_x)_\mathrm{NC}$ & & 0.2& $10^{-4}$\\ \hline \hline
\multicolumn{3}{|l|}{Appearance -- $\mu^-$-stored} & & \\ \hline
Signal & $0.35 \, \otimes \, (\bar{\nu}_e \rightarrow \bar{\nu}_\mu)_\mathrm{CC}$  & & 0.025& $10^{-4}$\\
 & & & & \\
Background &  $5.0\cdot 10^{-6} \, \otimes \, (\nu_\mu \rightarrow \nu_x)_\mathrm{NC}$ & $5.0\cdot 10^{-6} \, \otimes \, (\nu_\mu \rightarrow
\nu_\mu)_\mathrm{CC}$  & 0.2& $10^{-4}$\\ \hline \hline
\end{tabular}
\end{center}

\subsection*{Variable neutrino factory with Silver Channel -- \\{\tt NF\_GoldSilver.glb}}

A variable neutrino factory scenario that includes the golden and silver appearance channels can be 
simulated with the file {\tt NF\_GoldSilver.glb}. The basic version was
used within \cite{Huber:2006wb} and the golden channel follows the neutrino factory scenarios from \cite{Huber:2002mx}
and the description of the silver channel follows \cite{Autiero:2003fu}. These
references should be cited if the file {\tt
NFvar\_GoldSilver.glb} is used for a scientific publication or a talk. For calculations that involve {\tt
NFvar\_GoldSilver.glb}, the following additional files are required:
\bi
\item {\tt ThresholdEnergies.dat} (list of energies)
\item {\tt CIDthreshold.dat} (list for CID threshold)
\item {\tt MINOSthreshold.dat} (list for MINOS threshold)
\item {\tt ECCthreshold.dat} (list for ECC threshold)
\item {\tt XCC.dat} (charged current cross sections)
\item {\tt XNC.dat} (neutral current cross sections)
\ei
and the values of the following {\sf AEDL}-Variables have to be set:
\bi
\item {\tt emax} (parent energy of the stored muons $\left[\mathrm{km}\right]$)
\item {\tt BASELINE} (experiment baseline $\left[\mathrm{GeV}\right]$)
\ei
The beam and golden channel attributes are similar to {\tt NFvar.glb}. The parent energy of the muons can be set
in the range $\mathrm{10\,GeV\lesssim E_\mu\lesssim 80\, GeV}$. The baseline is set by the {\sf
AEDL}-Variable {\tt BASELINE}. For the silver channel an additional ECC detector with a fiducial mass
$\mathrm{m_{ECC} = 5\,kt}$ is assumed to be located at the same baseline as the MID detector. The energy
resolution of the silver channel is set to $\mathrm{\sigma_e=20\%\cdot E}$. The following additional rule
compared to {\tt NFvar.glb} is introduced to {\tt NF\_GoldSilver.glb}:
\begin{center}
\begin{tabular}{|l|ll|c|c|}
\hline \hline
\multicolumn{3}{|l|}{$\tau$-Appearance -- $\mu^+$-stored} & $\sigma_\mathrm{norm}$ & $\sigma_\mathrm{cal}$ \\ \hline
Signal & $0.096 \, \otimes \, (\nu_e \rightarrow \nu_\tau)_\mathrm{CC}$ & & 0.15 & $10^{-4}$ \\
 & & & & \\
Background & $3.1\cdot 10^{-8} \, \otimes \, (\nu_e \rightarrow \nu_e)_\mathrm{CC}$ & $2.0\cdot 10^{-8} \,
\otimes \, (\nu_e \rightarrow \nu_\mu)_\mathrm{CC}$& 0.2 & $10^{-4}$ \\ 
 & $3.7\cdot 10^{-6} \, \otimes \, (\bar{\nu}_\mu \rightarrow \bar{\nu}_\mu)_\mathrm{CC}$ & $1.0\cdot 10^{-3} \,
 \otimes \, (\bar{\nu}_\mu \rightarrow \bar{\nu}_\tau)_\mathrm{CC}$& 0.2 & $10^{-4}$ \\ 
 & $7.0 \cdot 10^{-7} \, \otimes \, (\bar{\nu}_\mu \rightarrow \bar{\nu}_x)_\mathrm{NC}$ & $7.0\cdot 10^{-7} \, \otimes \, (\nu_e \rightarrow \nu_x)_\mathrm{NC}$& 0.2 & $10^{-4}$ \\ \hline \hline
\end{tabular}
\end{center}

\subsection*{High Resolution/Low Threshold neutrino factory -- \\{\tt NF\_hR\_lT.glb}}

A variable neutrino factory hybrid detector scenario can be simulated with the file {\tt NF\_hR\_lT.glb}. The basic version was
used within \cite{Huber:2006wb} and follows the neutrino factory scenarios from \cite{Huber:2002mx}. These
references should be cited if the file {\tt NF\_hR\_lT.glb} is used for a scientific publication or a talk.
For calculations that involve {\tt NF\_hR\_lT.glb}, the same additional files as for {\tt NFvar.glb} are
required and the {\sf AEDL}-Variables {\tt emax} and {\tt BASELINE} have to be set. {\tt NF\_hR\_lT.glb} 
implements a lower threshold ($\mathrm{\sim\,1\,GeV}$) at a higher energy resolution
$\mathrm{\sigma_e=15\%\cdot E + 0.085\, MeV}$ where the constant term represents 
the effects from Fermi Motion. The background rejection is introduced energy dependent as 
$10^{-3}/\mathrm{E^2}$ and matches {\tt NFvar.glb} at higher energies. The following rules are defined within {\tt NF\_hR\_lT.glb}: 
\begin{center}
\begin{tabular}{|l|ll|c|c|}
\hline \hline
\multicolumn{3}{|l|}{Disappearance -- $\mu^+$-stored} & $\sigma_\mathrm{norm}$ & $\sigma_\mathrm{cal}$ \\ \hline
Signal & $0.9 \, \otimes \, (\bar{\nu}_\mu \rightarrow \bar{\nu}_\mu)_\mathrm{CC}$ & $0.9 \, \otimes \, (\nu_e \rightarrow \nu_\mu)_\mathrm{CC}$& 0.025 & $10^{-4}$ \\
 & & & & \\
Background & $1.0 \, \otimes \, (\bar{\nu}_\mu \rightarrow \bar{\nu}_x)_\mathrm{NC}$ & & 0.2 & $10^{-4}$ \\ \hline \hline
\multicolumn{3}{|l|}{Appearance -- $\mu^+$-stored} & & \\ \hline
Signal &  $0.5 \, \otimes \, (\nu_e \rightarrow \nu_\mu)_\mathrm{CC}$ & & 0.025 & $10^{-4}$ \\
 & & & & \\
Background &  $1.0 \, \otimes \, (\bar{\nu}_\mu \rightarrow \bar{\nu}_x)_\mathrm{NC}$ & (energy dep. rejection)
& 0.2 & $10^{-4}$\\
 &  $1.0 \, \otimes \, (\bar{\nu}_\mu \rightarrow\bar{\nu}_\mu)_\mathrm{CC}$ & (energy dep. rejection)
&0.2& $10^{-4}$\\ \hline \hline
\multicolumn{3}{|l|}{Disappearance -- $\mu^-$-stored} & & \\ \hline
Signal &  $0.9 \, \otimes \, (\nu_\mu \rightarrow \nu_\mu)_\mathrm{CC}$ & $0.9 \, \otimes \, (\bar{\nu}_e
\rightarrow \bar{\nu}_\mu)_\mathrm{CC}$& 0.025& $10^{-4}$\\
 & & & & \\
Background &  $1.0\cdot 10^{-5} \, \otimes \, (\nu_\mu \rightarrow \nu_x)_\mathrm{NC}$ & & 0.2& $10^{-4}$\\ \hline \hline
\multicolumn{3}{|l|}{Appearance -- $\mu^-$-stored} & & \\ \hline
Signal & $0.5 \, \otimes \, (\bar{\nu}_e \rightarrow \bar{\nu}_\mu)_\mathrm{CC}$  & & 0.025& $10^{-4}$\\
 & & & & \\
Background &  $1.0 \, \otimes \, (\nu_\mu \rightarrow \nu_x)_\mathrm{NC}$ & (energy dep. rejection)
& 0.2 & $10^{-4}$\\
 & $1.0 \, \otimes \, (\nu_\mu \rightarrow
\nu_\mu)_\mathrm{CC}$  & (energy dep. rejection)
&0.2& $10^{-4}$\\ \hline \hline
\end{tabular}
\end{center}









%%%%%%%%%%%%%%%%%%%%%%%%%%%%%%%%%%%%%%%%%%%%%
\chapter{Flux normalization in \GLOBES }
\index{norm}{Normalization of fluxes}
\label{app:flux}

A common issue with \GLOBES\ is confusion about the proper units for
the input flux files for use in \AEDL\ experiment descriptions. Source of
the confusion is an undocumented factor 5.2 with which the fluxes
are multiplied in \GLOBES\ versions older than 3.0 (see below). In Version 3.0 and 
higher, the alternative flux environment {\tt nuflux} is provided, 
which does not contain this factor. The following material is
based on the old environment {\tt flux}. For the use of {\tt nuflux},
replace the factor 5.2 by unity.

\section*{Historical problem}

One problem for the design of \AEDL\ was initially that meaningful
units for flux data strongly depend on the given type of experiment,
but also on relatively arbitrary decisions. For accelerator beams
based on pion decay, one frequently defines the beam luminosity in
{\it protons on target (pot)} since this number has a one-to-one
correspondence with the number of neutrinos produced. Another sensible
unit could be {\it megawatt on target (MW)}, again this number is directly
correlated with the number of neutrinos and moreover there is a unique
relation to {\it pot} for a given accelerator. Of course,
what matters is the integrated luminosity. In some  cases the
neutrino flux is given per $10^7\,\mathrm{s}$. However,
most experiments will run for several years, hence also this number
has to enter somewhere. For neutrino factories the proper number is
useful muon decays per unit time, and for reactor experiments it is
the thermal power of the reactor {\it asf}. This demonstrates that it is
reasonable to keep the flux definition flexible.

\section*{Implementation in \GLOBES }

In understanding how one still can figure out what the correct units
are for each case, it is a good starting point to look at what
\GLOBES\ does with the input files. The cross section in the file is given as
differential cross section divided by energy $x=\sigma/E$, and the flux file
gives $f$. The differential number of events per GeV $n$ as computed in
\GLOBES\ without oscillation and efficiencies is given by
\begin{eqnarray}
n &=& 5.2\times x\times E\times f\times \nonumber \\
&&\mathtt{@norm}\times\mathtt{@power}\times\mathtt{@stored\_muons}\times\mathtt{@time}\times\mathtt{\$target\_mass}\times(\mathtt{\$baseline})^{-2} \nonumber
\end{eqnarray}
Note that $5.2$ is a undocumented fudge factor!

It is the sole responsibility of the author of the \AEDL\ file and its
supporting files, to ensure that the result makes sense. In
principle, it is possible to divide, for example, {\tt @time}  by
$\pi$ and fix that by redefining the flux file by multiplying
it with $\pi$. Modifications like that have happened in the past
and still happen, and many of them are not properly commented.
 
\section*{Writing \AEDL\ files}

The task is to choose the value of  {\tt @norm} such that all the
variables in the \AEDL\ file have the proper units, \eg , {\tt
  @time} has proper unit years.

\GLOBES\ assumes that the cross
section $x$ is given in $10^{-38}\,\mathrm{cm}^2$ and that all fluxes are
given at a distance of $1\,\mathrm{km}$. In addition, it assumes that
the number of target nuclei $\tau$ (or protons or whatever applies to the
given cross section) per unit target mass $m_u$ (which usually is $kt$)
 are properly accounted for.\footnote{Note that the cross sections which are delivered with
  \GLOBES\ always are per nucleon.}  Assuming that in the flux file
the data is given as number of neutrinos per unit area $A$ and
energy bin of width $\Delta E$ at a distance $L$ from the source, one
obtains
\begin{eqnarray}
\mathtt{@norm}=\frac{1}{5.2}\left(\frac{\mathrm{GeV}}{\Delta
      E}\right)
\left(\frac{\mathrm{cm}^2}{A}\right)\left(\frac{L}{\mathrm{km}}\right)^2\left(\frac{\tau}{m_u}\right)\times10^{-38}\times\left(\frac{\mathcal{L}_u}{\mathcal{L}}\right)
\end{eqnarray}
where $\mathcal{L}$ absorbs all factors in the flux file related to
the integrated luminosity, and $\mathcal{L}_u$ is the unit chosen for
it.  The concept of integrated luminosity is
nicely described in the \GLOBES\ manual in \Sec~\ref{sec:source}.
A little example illustrates this concept:
The flux is given for $10^{21}\, \mathrm{pot}\,\mathrm{y}^{-1}$ of
$10\,\mathrm{GeV}$ protons, thus a good choice for the units $\mathcal{L}_u$
is $\mathrm{MW}\,\mathrm{y}^{-1}$, which means that
$\mathcal{L}/\mathcal{L}_u$ is given by (assuming a $10^7\,\mathrm{s}$
year)
\begin{equation}
\frac{\mathcal{L}}{\mathcal{L}_u}=\frac{10\,\mathrm{GeV}\,10^{21}\,\mathrm{pot}\,\mathrm{y}}{10^7\,\mathrm{s}}\times(\mathrm{MW}\,\mathrm{y}^{-1})^{-1}=0.16\ldots
\end{equation}

\section*{Moving from {\tt flux} to {\tt nuflux}}

In order to change the older {\tt flux} environment to the
new {\tt nuflux} (\GLOBES\ 3.0 and higher), replace all user-defined fluxes, such as
\begin{quote}
{\tt flux(\#user)<}\\
{\tt \tb @flux\_file = "user\_file\_1.dat"\\
\tb @time = 2.0\\
\tb @power = 4.0\\
\tb @norm = 1e+8}\\
{\tt >}
\end{quote}
by
\begin{quote}
{\tt FF=5.1989} \\
\\
{\tt nuflux(\#user)<}\\
{\tt \tb @flux\_file = "user\_file\_1.dat"\\
\tb @time = 2.0\\
\tb @power = 4.0\\
\tb @norm = FF*1e+8}\\
{\tt >}
\end{quote}

This replacement is not necessary for neutrino factory built-in fluxes,
and built-in beta beam fluxes were not supported by earlier versions
of \GLOBES .



%%%%%%%%%%%%%%%%%%%%%%%%%%%%%%%%%%%%%%%%%%%%

{\footnotesize
\chapter{The GNU General Public License}

\begin{center}
{\parindent 0in

Version 2, June 1991

Copyright \copyright\ 1989, 1991 Free Software Foundation, Inc.

\bigskip

59 Temple Place - Suite 330, Boston, MA  02111-1307, USA

\bigskip

Everyone is permitted to copy and distribute verbatim copies
of this license document, but changing it is not allowed.
}
\end{center}

\begin{center}
{\bf\large Preamble}
\end{center}


The licenses for most software are designed to take away your freedom to
share and change it.  By contrast, the GNU General Public License is
intended to guarantee your freedom to share and change free software---to
make sure the software is free for all its users.  This General Public
License applies to most of the Free Software Foundation's software and to
any other program whose authors commit to using it.  (Some other Free
Software Foundation software is covered by the GNU Library General Public
License instead.)  You can apply it to your programs, too.

When we speak of free software, we are referring to freedom, not price.
Our General Public Licenses are designed to make sure that you have the
freedom to distribute copies of free software (and charge for this service
if you wish), that you receive source code or can get it if you want it,
that you can change the software or use pieces of it in new free programs;
and that you know you can do these things.

To protect your rights, we need to make restrictions that forbid anyone to
deny you these rights or to ask you to surrender the rights.  These
restrictions translate to certain responsibilities for you if you
distribute copies of the software, or if you modify it.

For example, if you distribute copies of such a program, whether gratis or
for a fee, you must give the recipients all the rights that you have.  You
must make sure that they, too, receive or can get the source code.  And
you must show them these terms so they know their rights.

We protect your rights with two steps: (1) copyright the software, and (2)
offer you this license which gives you legal permission to copy,
distribute and/or modify the software.

Also, for each author's protection and ours, we want to make certain that
everyone understands that there is no warranty for this free software.  If
the software is modified by someone else and passed on, we want its
recipients to know that what they have is not the original, so that any
problems introduced by others will not reflect on the original authors'
reputations.

Finally, any free program is threatened constantly by software patents.
We wish to avoid the danger that redistributors of a free program will
individually obtain patent licenses, in effect making the program
proprietary.  To prevent this, we have made it clear that any patent must
be licensed for everyone's free use or not licensed at all.

The precise terms and conditions for copying, distribution and
modification follow.

\begin{center}
{\Large \sc Terms and Conditions For Copying, Distribution and
  Modification}
\end{center}


%\renewcommand{\theenumi}{\alpha{enumi}}
\begin{enumerate}

\addtocounter{enumi}{-1}

\item 

This License applies to any program or other work which contains a notice
placed by the copyright holder saying it may be distributed under the
terms of this General Public License.  The ``Program'', below, refers to
any such program or work, and a ``work based on the Program'' means either
the Program or any derivative work under copyright law: that is to say, a
work containing the Program or a portion of it, either verbatim or with
modifications and/or translated into another language.  (Hereinafter,
translation is included without limitation in the term ``modification''.)
Each licensee is addressed as ``you''.

Activities other than copying, distribution and modification are not
covered by this License; they are outside its scope.  The act of
running the Program is not restricted, and the output from the Program
is covered only if its contents constitute a work based on the
Program (independent of having been made by running the Program).
Whether that is true depends on what the Program does.

\item You may copy and distribute verbatim copies of the Program's source
  code as you receive it, in any medium, provided that you conspicuously
  and appropriately publish on each copy an appropriate copyright notice
  and disclaimer of warranty; keep intact all the notices that refer to
  this License and to the absence of any warranty; and give any other
  recipients of the Program a copy of this License along with the Program.

You may charge a fee for the physical act of transferring a copy, and you
may at your option offer warranty protection in exchange for a fee.

\item

You may modify your copy or copies of the Program or any portion
of it, thus forming a work based on the Program, and copy and
distribute such modifications or work under the terms of Section 1
above, provided that you also meet all of these conditions:

\begin{enumerate}

\item 

You must cause the modified files to carry prominent notices stating that
you changed the files and the date of any change.

\item

You must cause any work that you distribute or publish, that in
whole or in part contains or is derived from the Program or any
part thereof, to be licensed as a whole at no charge to all third
parties under the terms of this License.

\item
If the modified program normally reads commands interactively
when run, you must cause it, when started running for such
interactive use in the most ordinary way, to print or display an
announcement including an appropriate copyright notice and a
notice that there is no warranty (or else, saying that you provide
a warranty) and that users may redistribute the program under
these conditions, and telling the user how to view a copy of this
License.  (Exception: if the Program itself is interactive but
does not normally print such an announcement, your work based on
the Program is not required to print an announcement.)

\end{enumerate}


These requirements apply to the modified work as a whole.  If
identifiable sections of that work are not derived from the Program,
and can be reasonably considered independent and separate works in
themselves, then this License, and its terms, do not apply to those
sections when you distribute them as separate works.  But when you
distribute the same sections as part of a whole which is a work based
on the Program, the distribution of the whole must be on the terms of
this License, whose permissions for other licensees extend to the
entire whole, and thus to each and every part regardless of who wrote it.

Thus, it is not the intent of this section to claim rights or contest
your rights to work written entirely by you; rather, the intent is to
exercise the right to control the distribution of derivative or
collective works based on the Program.

In addition, mere aggregation of another work not based on the Program
with the Program (or with a work based on the Program) on a volume of
a storage or distribution medium does not bring the other work under
the scope of this License.

\item
You may copy and distribute the Program (or a work based on it,
under Section 2) in object code or executable form under the terms of
Sections 1 and 2 above provided that you also do one of the following:

\begin{enumerate}

\item

Accompany it with the complete corresponding machine-readable
source code, which must be distributed under the terms of Sections
1 and 2 above on a medium customarily used for software interchange; or,

\item

Accompany it with a written offer, valid for at least three
years, to give any third party, for a charge no more than your
cost of physically performing source distribution, a complete
machine-readable copy of the corresponding source code, to be
distributed under the terms of Sections 1 and 2 above on a medium
customarily used for software interchange; or,

\item

Accompany it with the information you received as to the offer
to distribute corresponding source code.  (This alternative is
allowed only for noncommercial distribution and only if you
received the program in object code or executable form with such
an offer, in accord with Subsection b above.)

\end{enumerate}


The source code for a work means the preferred form of the work for
making modifications to it.  For an executable work, complete source
code means all the source code for all modules it contains, plus any
associated interface definition files, plus the scripts used to
control compilation and installation of the executable.  However, as a
special exception, the source code distributed need not include
anything that is normally distributed (in either source or binary
form) with the major components (compiler, kernel, and so on) of the
operating system on which the executable runs, unless that component
itself accompanies the executable.

If distribution of executable or object code is made by offering
access to copy from a designated place, then offering equivalent
access to copy the source code from the same place counts as
distribution of the source code, even though third parties are not
compelled to copy the source along with the object code.

\item
You may not copy, modify, sublicense, or distribute the Program
except as expressly provided under this License.  Any attempt
otherwise to copy, modify, sublicense or distribute the Program is
void, and will automatically terminate your rights under this License.
However, parties who have received copies, or rights, from you under
this License will not have their licenses terminated so long as such
parties remain in full compliance.

\item
You are not required to accept this License, since you have not
signed it.  However, nothing else grants you permission to modify or
distribute the Program or its derivative works.  These actions are
prohibited by law if you do not accept this License.  Therefore, by
modifying or distributing the Program (or any work based on the
Program), you indicate your acceptance of this License to do so, and
all its terms and conditions for copying, distributing or modifying
the Program or works based on it.

\item
Each time you redistribute the Program (or any work based on the
Program), the recipient automatically receives a license from the
original licensor to copy, distribute or modify the Program subject to
these terms and conditions.  You may not impose any further
restrictions on the recipients' exercise of the rights granted herein.
You are not responsible for enforcing compliance by third parties to
this License.

\item
If, as a consequence of a court judgment or allegation of patent
infringement or for any other reason (not limited to patent issues),
conditions are imposed on you (whether by court order, agreement or
otherwise) that contradict the conditions of this License, they do not
excuse you from the conditions of this License.  If you cannot
distribute so as to satisfy simultaneously your obligations under this
License and any other pertinent obligations, then as a consequence you
may not distribute the Program at all.  For example, if a patent
license would not permit royalty-free redistribution of the Program by
all those who receive copies directly or indirectly through you, then
the only way you could satisfy both it and this License would be to
refrain entirely from distribution of the Program.

If any portion of this section is held invalid or unenforceable under
any particular circumstance, the balance of the section is intended to
apply and the section as a whole is intended to apply in other
circumstances.

It is not the purpose of this section to induce you to infringe any
patents or other property right claims or to contest validity of any
such claims; this section has the sole purpose of protecting the
integrity of the free software distribution system, which is
implemented by public license practices.  Many people have made
generous contributions to the wide range of software distributed
through that system in reliance on consistent application of that
system; it is up to the author/donor to decide if he or she is willing
to distribute software through any other system and a licensee cannot
impose that choice.

This section is intended to make thoroughly clear what is believed to
be a consequence of the rest of this License.

\item
If the distribution and/or use of the Program is restricted in
certain countries either by patents or by copyrighted interfaces, the
original copyright holder who places the Program under this License
may add an explicit geographical distribution limitation excluding
those countries, so that distribution is permitted only in or among
countries not thus excluded.  In such case, this License incorporates
the limitation as if written in the body of this License.

\item
The Free Software Foundation may publish revised and/or new versions
of the General Public License from time to time.  Such new versions will
be similar in spirit to the present version, but may differ in detail to
address new problems or concerns.

Each version is given a distinguishing version number.  If the Program
specifies a version number of this License which applies to it and ``any
later version'', you have the option of following the terms and conditions
either of that version or of any later version published by the Free
Software Foundation.  If the Program does not specify a version number of
this License, you may choose any version ever published by the Free Software
Foundation.

\item
If you wish to incorporate parts of the Program into other free
programs whose distribution conditions are different, write to the author
to ask for permission.  For software which is copyrighted by the Free
Software Foundation, write to the Free Software Foundation; we sometimes
make exceptions for this.  Our decision will be guided by the two goals
of preserving the free status of all derivatives of our free software and
of promoting the sharing and reuse of software generally.

\begin{center}
{\Large\sc
No Warranty
}
\end{center}

\item
{\sc Because the program is licensed free of charge, there is no warranty
for the program, to the extent permitted by applicable law.  Except when
otherwise stated in writing the copyright holders and/or other parties
provide the program ``as is'' without warranty of any kind, either expressed
or implied, including, but not limited to, the implied warranties of
merchantability and fitness for a particular purpose.  The entire risk as
to the quality and performance of the program is with you.  Should the
program prove defective, you assume the cost of all necessary servicing,
repair or correction.}

\item
{\sc In no event unless required by applicable law or agreed to in writing
will any copyright holder, or any other party who may modify and/or
redistribute the program as permitted above, be liable to you for damages,
including any general, special, incidental or consequential damages arising
out of the use or inability to use the program (including but not limited
to loss of data or data being rendered inaccurate or losses sustained by
you or third parties or a failure of the program to operate with any other
programs), even if such holder or other party has been advised of the
possibility of such damages.}

\end{enumerate}


\begin{center}
{\Large\sc End of Terms and Conditions}
\end{center}


\pagebreak[2]

\section*{Appendix: How to Apply These Terms to Your New Programs}

If you develop a new program, and you want it to be of the greatest
possible use to the public, the best way to achieve this is to make it
free software which everyone can redistribute and change under these
terms.

  To do so, attach the following notices to the program.  It is safest to
  attach them to the start of each source file to most effectively convey
  the exclusion of warranty; and each file should have at least the
  ``copyright'' line and a pointer to where the full notice is found.

\begin{quote}
one line to give the program's name and a brief idea of what it does. \\
Copyright (C) yyyy  name of author \\

This program is free software; you can redistribute it and/or modify
it under the terms of the GNU General Public License as published by
the Free Software Foundation; either version 2 of the License, or
(at your option) any later version.

This program is distributed in the hope that it will be useful,
but WITHOUT ANY WARRANTY; without even the implied warranty of
MERCHANTABILITY or FITNESS FOR A PARTICULAR PURPOSE.  See the
GNU General Public License for more details.

You should have received a copy of the GNU General Public License
along with this program; if not, write to the Free Software
Foundation, Inc., 59 Temple Place - Suite 330, Boston, MA  02111-1307, USA.
\end{quote}

Also add information on how to contact you by electronic and paper mail.

If the program is interactive, make it output a short notice like this
when it starts in an interactive mode:

\begin{quote}
Gnomovision version 69, Copyright (C) yyyy  name of author \\
Gnomovision comes with ABSOLUTELY NO WARRANTY; for details type `show w'. \\
This is free software, and you are welcome to redistribute it
under certain conditions; type `show c' for details.
\end{quote}


The hypothetical commands {\tt show w} and {\tt show c} should show the
appropriate parts of the General Public License.  Of course, the commands
you use may be called something other than {\tt show w} and {\tt show c};
they could even be mouse-clicks or menu items---whatever suits your
program.

You should also get your employer (if you work as a programmer) or your
school, if any, to sign a ``copyright disclaimer'' for the program, if
necessary.  Here is a sample; alter the names:

\begin{quote}
Yoyodyne, Inc., hereby disclaims all copyright interest in the program \\
`Gnomovision' (which makes passes at compilers) written by James Hacker. \\

signature of Ty Coon, 1 April 1989 \\
Ty Coon, President of Vice
\end{quote}


This General Public License does not permit incorporating your program
into proprietary programs.  If your program is a subroutine library, you
may consider it more useful to permit linking proprietary applications
with the library.  If this is what you want to do, use the GNU Library
General Public License instead of this License.



\chapter{GNU Free Documentation License}
%\label{label_fdl}

 \begin{center}

       Version 1.2, November 2002


 Copyright \copyright 2000,2001,2002  Free Software Foundation, Inc.
 
 \bigskip
 
     59 Temple Place, Suite 330, Boston, MA  02111-1307  USA
  
 \bigskip
 
 Everyone is permitted to copy and distribute verbatim copies
 of this license document, but changing it is not allowed.
\end{center}


\begin{center}
{\bf\large Preamble}
\end{center}

The purpose of this License is to make a manual, textbook, or other
functional and useful document "free" in the sense of freedom: to
assure everyone the effective freedom to copy and redistribute it,
with or without modifying it, either commercially or noncommercially.
Secondarily, this License preserves for the author and publisher a way
to get credit for their work, while not being considered responsible
for modifications made by others.

This License is a kind of "copyleft", which means that derivative
works of the document must themselves be free in the same sense.  It
complements the GNU General Public License, which is a copyleft
license designed for free software.

We have designed this License in order to use it for manuals for free
software, because free software needs free documentation: a free
program should come with manuals providing the same freedoms that the
software does.  But this License is not limited to software manuals;
it can be used for any textual work, regardless of subject matter or
whether it is published as a printed book.  We recommend this License
principally for works whose purpose is instruction or reference.


\begin{center}
{\Large\bf 1. APPLICABILITY AND DEFINITIONS}
%\addcontentsline{toc}{section}{1. APPLICABILITY AND DEFINITIONS}
\end{center}

This License applies to any manual or other work, in any medium, that
contains a notice placed by the copyright holder saying it can be
distributed under the terms of this License.  Such a notice grants a
world-wide, royalty-free license, unlimited in duration, to use that
work under the conditions stated herein.  The \textbf{"Document"}, below,
refers to any such manual or work.  Any member of the public is a
licensee, and is addressed as \textbf{"you"}.  You accept the license if you
copy, modify or distribute the work in a way requiring permission
under copyright law.

A \textbf{"Modified Version"} of the Document means any work containing the
Document or a portion of it, either copied verbatim, or with
modifications and/or translated into another language.

A \textbf{"Secondary Section"} is a named appendix or a front-matter section of
the Document that deals exclusively with the relationship of the
publishers or authors of the Document to the Document's overall subject
(or to related matters) and contains nothing that could fall directly
within that overall subject.  (Thus, if the Document is in part a
textbook of mathematics, a Secondary Section may not explain any
mathematics.)  The relationship could be a matter of historical
connection with the subject or with related matters, or of legal,
commercial, philosophical, ethical or political position regarding
them.

The \textbf{"Invariant Sections"} are certain Secondary Sections whose titles
are designated, as being those of Invariant Sections, in the notice
that says that the Document is released under this License.  If a
section does not fit the above definition of Secondary then it is not
allowed to be designated as Invariant.  The Document may contain zero
Invariant Sections.  If the Document does not identify any Invariant
Sections then there are none.

The \textbf{"Cover Texts"} are certain short passages of text that are listed,
as Front-Cover Texts or Back-Cover Texts, in the notice that says that
the Document is released under this License.  A Front-Cover Text may
be at most 5 words, and a Back-Cover Text may be at most 25 words.

A \textbf{"Transparent"} copy of the Document means a machine-readable copy,
represented in a format whose specification is available to the
general public, that is suitable for revising the document
straightforwardly with generic text editors or (for images composed of
pixels) generic paint programs or (for drawings) some widely available
drawing editor, and that is suitable for input to text formatters or
for automatic translation to a variety of formats suitable for input
to text formatters.  A copy made in an otherwise Transparent file
format whose markup, or absence of markup, has been arranged to thwart
or discourage subsequent modification by readers is not Transparent.
An image format is not Transparent if used for any substantial amount
of text.  A copy that is not "Transparent" is called \textbf{"Opaque"}.

Examples of suitable formats for Transparent copies include plain
ASCII without markup, Texinfo input format, LaTeX input format, SGML
or XML using a publicly available DTD, and standard-conforming simple
HTML, PostScript or PDF designed for human modification.  Examples of
transparent image formats include PNG, XCF and JPG.  Opaque formats
include proprietary formats that can be read and edited only by
proprietary word processors, SGML or XML for which the DTD and/or
processing tools are not generally available, and the
machine-generated HTML, PostScript or PDF produced by some word
processors for output purposes only.

The \textbf{"Title Page"} means, for a printed book, the title page itself,
plus such following pages as are needed to hold, legibly, the material
this License requires to appear in the title page.  For works in
formats which do not have any title page as such, "Title Page" means
the text near the most prominent appearance of the work's title,
preceding the beginning of the body of the text.

A section \textbf{"Entitled XYZ"} means a named subunit of the Document whose
title either is precisely XYZ or contains XYZ in parentheses following
text that translates XYZ in another language.  (Here XYZ stands for a
specific section name mentioned below, such as \textbf{"Acknowledgements"},
\textbf{"Dedications"}, \textbf{"Endorsements"}, or \textbf{"History"}.)  
To \textbf{"Preserve the Title"}
of such a section when you modify the Document means that it remains a
section "Entitled XYZ" according to this definition.

The Document may include Warranty Disclaimers next to the notice which
states that this License applies to the Document.  These Warranty
Disclaimers are considered to be included by reference in this
License, but only as regards disclaiming warranties: any other
implication that these Warranty Disclaimers may have is void and has
no effect on the meaning of this License.


\begin{center}
{\Large\bf 2. VERBATIM COPYING}
%\addcontentsline{toc}{section}{2. VERBATIM COPYING}
\end{center}

You may copy and distribute the Document in any medium, either
commercially or noncommercially, provided that this License, the
copyright notices, and the license notice saying this License applies
to the Document are reproduced in all copies, and that you add no other
conditions whatsoever to those of this License.  You may not use
technical measures to obstruct or control the reading or further
copying of the copies you make or distribute.  However, you may accept
compensation in exchange for copies.  If you distribute a large enough
number of copies you must also follow the conditions in section 3.

You may also lend copies, under the same conditions stated above, and
you may publicly display copies.


\begin{center}
{\Large\bf 3. COPYING IN QUANTITY}
%\addcontentsline{toc}{section}{3. COPYING IN QUANTITY}
\end{center}


If you publish printed copies (or copies in media that commonly have
printed covers) of the Document, numbering more than 100, and the
Document's license notice requires Cover Texts, you must enclose the
copies in covers that carry, clearly and legibly, all these Cover
Texts: Front-Cover Texts on the front cover, and Back-Cover Texts on
the back cover.  Both covers must also clearly and legibly identify
you as the publisher of these copies.  The front cover must present
the full title with all words of the title equally prominent and
visible.  You may add other material on the covers in addition.
Copying with changes limited to the covers, as long as they preserve
the title of the Document and satisfy these conditions, can be treated
as verbatim copying in other respects.

If the required texts for either cover are too voluminous to fit
legibly, you should put the first ones listed (as many as fit
reasonably) on the actual cover, and continue the rest onto adjacent
pages.

If you publish or distribute Opaque copies of the Document numbering
more than 100, you must either include a machine-readable Transparent
copy along with each Opaque copy, or state in or with each Opaque copy
a computer-network location from which the general network-using
public has access to download using public-standard network protocols
a complete Transparent copy of the Document, free of added material.
If you use the latter option, you must take reasonably prudent steps,
when you begin distribution of Opaque copies in quantity, to ensure
that this Transparent copy will remain thus accessible at the stated
location until at least one year after the last time you distribute an
Opaque copy (directly or through your agents or retailers) of that
edition to the public.

It is requested, but not required, that you contact the authors of the
Document well before redistributing any large number of copies, to give
them a chance to provide you with an updated version of the Document.


\begin{center}
{\Large\bf 4. MODIFICATIONS}
%\addcontentsline{toc}{section}{4. MODIFICATIONS}
\end{center}

You may copy and distribute a Modified Version of the Document under
the conditions of sections 2 and 3 above, provided that you release
the Modified Version under precisely this License, with the Modified
Version filling the role of the Document, thus licensing distribution
and modification of the Modified Version to whoever possesses a copy
of it.  In addition, you must do these things in the Modified Version:

\begin{itemize}
\item[A.] 
   Use in the Title Page (and on the covers, if any) a title distinct
   from that of the Document, and from those of previous versions
   (which should, if there were any, be listed in the History section
   of the Document).  You may use the same title as a previous version
   if the original publisher of that version gives permission.
   
\item[B.]
   List on the Title Page, as authors, one or more persons or entities
   responsible for authorship of the modifications in the Modified
   Version, together with at least five of the principal authors of the
   Document (all of its principal authors, if it has fewer than five),
   unless they release you from this requirement.
   
\item[C.]
   State on the Title page the name of the publisher of the
   Modified Version, as the publisher.
   
\item[D.]
   Preserve all the copyright notices of the Document.
   
\item[E.]
   Add an appropriate copyright notice for your modifications
   adjacent to the other copyright notices.
   
\item[F.]
   Include, immediately after the copyright notices, a license notice
   giving the public permission to use the Modified Version under the
   terms of this License, in the form shown in the Addendum below.
   
\item[G.]
   Preserve in that license notice the full lists of Invariant Sections
   and required Cover Texts given in the Document's license notice.
   
\item[H.]
   Include an unaltered copy of this License.
   
\item[I.]
   Preserve the section Entitled "History", Preserve its Title, and add
   to it an item stating at least the title, year, new authors, and
   publisher of the Modified Version as given on the Title Page.  If
   there is no section Entitled "History" in the Document, create one
   stating the title, year, authors, and publisher of the Document as
   given on its Title Page, then add an item describing the Modified
   Version as stated in the previous sentence.
   
\item[J.]
   Preserve the network location, if any, given in the Document for
   public access to a Transparent copy of the Document, and likewise
   the network locations given in the Document for previous versions
   it was based on.  These may be placed in the "History" section.
   You may omit a network location for a work that was published at
   least four years before the Document itself, or if the original
   publisher of the version it refers to gives permission.
   
\item[K.]
   For any section Entitled "Acknowledgements" or "Dedications",
   Preserve the Title of the section, and preserve in the section all
   the substance and tone of each of the contributor acknowledgements
   and/or dedications given therein.
   
\item[L.]
   Preserve all the Invariant Sections of the Document,
   unaltered in their text and in their titles.  Section numbers
   or the equivalent are not considered part of the section titles.
   
\item[M.]
   Delete any section Entitled "Endorsements".  Such a section
   may not be included in the Modified Version.
   
\item[N.]
   Do not retitle any existing section to be Entitled "Endorsements"
   or to conflict in title with any Invariant Section.
   
\item[O.]
   Preserve any Warranty Disclaimers.
\end{itemize}

If the Modified Version includes new front-matter sections or
appendices that qualify as Secondary Sections and contain no material
copied from the Document, you may at your option designate some or all
of these sections as invariant.  To do this, add their titles to the
list of Invariant Sections in the Modified Version's license notice.
These titles must be distinct from any other section titles.

You may add a section Entitled "Endorsements", provided it contains
nothing but endorsements of your Modified Version by various
parties--for example, statements of peer review or that the text has
been approved by an organization as the authoritative definition of a
standard.

You may add a passage of up to five words as a Front-Cover Text, and a
passage of up to 25 words as a Back-Cover Text, to the end of the list
of Cover Texts in the Modified Version.  Only one passage of
Front-Cover Text and one of Back-Cover Text may be added by (or
through arrangements made by) any one entity.  If the Document already
includes a cover text for the same cover, previously added by you or
by arrangement made by the same entity you are acting on behalf of,
you may not add another; but you may replace the old one, on explicit
permission from the previous publisher that added the old one.

The author(s) and publisher(s) of the Document do not by this License
give permission to use their names for publicity for or to assert or
imply endorsement of any Modified Version.


\begin{center}
{\Large\bf 5. COMBINING DOCUMENTS}
%\addcontentsline{toc}{section}{5. COMBINING DOCUMENTS}
\end{center}


You may combine the Document with other documents released under this
License, under the terms defined in section 4 above for modified
versions, provided that you include in the combination all of the
Invariant Sections of all of the original documents, unmodified, and
list them all as Invariant Sections of your combined work in its
license notice, and that you preserve all their Warranty Disclaimers.

The combined work need only contain one copy of this License, and
multiple identical Invariant Sections may be replaced with a single
copy.  If there are multiple Invariant Sections with the same name but
different contents, make the title of each such section unique by
adding at the end of it, in parentheses, the name of the original
author or publisher of that section if known, or else a unique number.
Make the same adjustment to the section titles in the list of
Invariant Sections in the license notice of the combined work.

In the combination, you must combine any sections Entitled "History"
in the various original documents, forming one section Entitled
"History"; likewise combine any sections Entitled "Acknowledgements",
and any sections Entitled "Dedications".  You must delete all sections
Entitled "Endorsements".

\begin{center}
{\Large\bf 6. COLLECTIONS OF DOCUMENTS}
%\addcontentsline{toc}{section}{6. COLLECTIONS OF DOCUMENTS}
\end{center}

You may make a collection consisting of the Document and other documents
released under this License, and replace the individual copies of this
License in the various documents with a single copy that is included in
the collection, provided that you follow the rules of this License for
verbatim copying of each of the documents in all other respects.

You may extract a single document from such a collection, and distribute
it individually under this License, provided you insert a copy of this
License into the extracted document, and follow this License in all
other respects regarding verbatim copying of that document.


\begin{center}
{\Large\bf 7. AGGREGATION WITH INDEPENDENT WORKS}
%\addcontentsline{toc}{section}{7. AGGREGATION WITH INDEPENDENT WORKS}
\end{center}


A compilation of the Document or its derivatives with other separate
and independent documents or works, in or on a volume of a storage or
distribution medium, is called an "aggregate" if the copyright
resulting from the compilation is not used to limit the legal rights
of the compilation's users beyond what the individual works permit.
When the Document is included in an aggregate, this License does not
apply to the other works in the aggregate which are not themselves
derivative works of the Document.

If the Cover Text requirement of section 3 is applicable to these
copies of the Document, then if the Document is less than one half of
the entire aggregate, the Document's Cover Texts may be placed on
covers that bracket the Document within the aggregate, or the
electronic equivalent of covers if the Document is in electronic form.
Otherwise they must appear on printed covers that bracket the whole
aggregate.


\begin{center}
{\Large\bf 8. TRANSLATION}
%\addcontentsline{toc}{section}{8. TRANSLATION}
\end{center}


Translation is considered a kind of modification, so you may
distribute translations of the Document under the terms of section 4.
Replacing Invariant Sections with translations requires special
permission from their copyright holders, but you may include
translations of some or all Invariant Sections in addition to the
original versions of these Invariant Sections.  You may include a
translation of this License, and all the license notices in the
Document, and any Warranty Disclaimers, provided that you also include
the original English version of this License and the original versions
of those notices and disclaimers.  In case of a disagreement between
the translation and the original version of this License or a notice
or disclaimer, the original version will prevail.

If a section in the Document is Entitled "Acknowledgements",
"Dedications", or "History", the requirement (section 4) to Preserve
its Title (section 1) will typically require changing the actual
title.


\begin{center}
{\Large\bf 9. TERMINATION}
%\addcontentsline{toc}{section}{9. TERMINATION}
\end{center}


You may not copy, modify, sublicense, or distribute the Document except
as expressly provided for under this License.  Any other attempt to
copy, modify, sublicense or distribute the Document is void, and will
automatically terminate your rights under this License.  However,
parties who have received copies, or rights, from you under this
License will not have their licenses terminated so long as such
parties remain in full compliance.


\begin{center}
{\Large\bf 10. FUTURE REVISIONS OF THIS LICENSE}
%\addcontentsline{toc}{section}{10. FUTURE REVISIONS OF THIS LICENSE}
\end{center}


The Free Software Foundation may publish new, revised versions
of the GNU Free Documentation License from time to time.  Such new
versions will be similar in spirit to the present version, but may
differ in detail to address new problems or concerns.  See
http://www.gnu.org/copyleft/.

Each version of the License is given a distinguishing version number.
If the Document specifies that a particular numbered version of this
License "or any later version" applies to it, you have the option of
following the terms and conditions either of that specified version or
of any later version that has been published (not as a draft) by the
Free Software Foundation.  If the Document does not specify a version
number of this License, you may choose any version ever published (not
as a draft) by the Free Software Foundation.




%%% Local Variables: 
%%% mode: latex
%%% TeX-master: "Manual"
%%% End: 

}


\end{appendix}

%%% Local Variables: 
%%% mode: latex
%%% TeX-master: Manual.tex
%%% TeX-master: "Manual"
%%% End:
