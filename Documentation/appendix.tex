%%%%%%%%%%%%%%%%%%%%%%%%%%%%%%%%%%%
% Appendix
%%%%%%%%%%%%%%%%%%%%%%%%%%%%%%%%%%%

\begin{appendix}

\chapter{\GLOBES\ installation}
\label{app:installation}
\index{norm}{Installation|(}


The installation of \GLOBES\ is highly automated and there should not be
any problems on a decently up-to-date GNU/Linux system. The installation
has, however, only been tested on a limited number of platforms, mainly
running with various versions of SuSE Linux. Thus we would appreciate
to know your experiences with the installation on different platforms.
Please send an e-mail to <\bugs>.

\section*{Prerequisites for the installation of \GLOBES}
\index{norm}{Installation!prerequisites}

Besides the usual things such as a working libc, you need to have
\begin{itemize}
\item[gcc] The GNU compiler collection\\
\verb+gcc.gnu.org+
\item[BLAS] Basic Linear Algebra Subprograms\\
\verb+www.netlib.org/blas/+    
\item[LAPACK] Linear Algebra PACKage\\
\verb+www.netlib.org/lapack/+
\item[f2c] Fortran to C\\ 
\verb+www.netlib.org/f2c/+
\item[GSL] The GNU Scientific Library\\
\verb+www.gnu.org/software/gsl/+
\end{itemize}
The library \verb+libglobes+ should in principle compile with any
ANSI C/C++ compiler, but the \verb+globes+ binary uses the 
\verb+argp+ facility of \verb+glibc+ to parse its command line options.

All those libraries are also available as rpm's from the various
distributors of GNU/Linux: See their web sites for downloads. Due to
the rather difficult nature of the build process of BLAS and LAPACK,
it is recommended that one uses a suitable rpm instead. In addition,
there is a good chance that
gcc, f2c and GSL are already part of your installation. Furthmore, you need
a working \verb+make+ to build and install \GLOBES.

In future releases, it is planned to at least make LAPACK obsolete, because
then also BLAS and f2c would drop out. Furthemore, there is no priniciple
reason why one should not get rid of any C++ parts. 
This will greatly simplify
the build process and reduce the requirements for installation.

\section*{Installation Instructions}

\GLOBES\ follows the standard GNU installation procedure.  To compile \GLOBES\
you will need gcc.  After unpacking the distribution
the Makefiles can be prepared using the configure command,
\begin{quote}
{\tt
./configure
}
\end{quote}
You can then build the library by typing,
\begin{quote}
{\tt
make
}
\end{quote}
A shared  version of the library will be compiled by
default. 
 
The library can be installed using the command,
\begin{quote}
{\tt
make install
}
\end{quote}
The default install directory prefix is \verb+/usr/local+.  Consult the
"Further Information" section below for instructions on installing the
library in another location or changing other default compilation
options.

The install target also will install a program with name \verb+globes+ to
\verb+$prefix/bin+ and the files in the \verb+data+ directory of the tar-ball
to \verb+$prefix/share/globes+.
\index{norm}{glb files@{\tt glb}-files!installation}

If you are not using \verb+make install+ you will find the static libary
at \verb+source/.lib/libglobes.a+ which you can copy to any destination.
However keep in mind that the linking command will be somewhat different,
\ie\ you have to specify all the dynamically linked objects besides libglobes,
which are
\begin{quote}
{\tt
         -lm -lgsl -lgslcblas -lf2c -llapack -lblas
}
\end{quote}

In general it is advisable to use the shared libraries. If you don't
have root privileges, see the corresponding section 
(page~\pageref{inst_noroot}).
If you install \GLOBES\ with root privileges, do not forget to run 
{\tt ldconfig} after installation.
 
\section*{Basic Installation}

   The \verb+configure+ shell script attempts to guess correct values for
various system-dependent variables used during compilation.  It uses
these values to create a \verb+Makefile+ in each directory of the package.
Finally, it creates a shell script \verb+config.status+ that
you can run in the future to recreate the current configuration, a file
\verb+config.cache+ that saves the results of its tests to speed up
reconfiguring, and a file \verb+config.log+ containing compiler output
(useful mainly for debugging \verb+configure+).

   If you need to do unusual things to compile the package, please try
to figure out how \verb+configure+ could check whether to do them, and mail
diffs or instructions to \bugs\ so they can
be considered for the next release.  If at some point \verb+config.cache+
contains results you don't want to keep, you may remove or edit it.

   The file \verb+configure.in+ is used to create \verb+configure+ by a program
called \verb+autoconf+.  You only need \verb+configure.in+ 
if you want to change it or regenerate \verb+configure+ using a newer 
version of \verb+autoconf+.

The simplest way to compile this package is:
\begin{enumerate}
\item \verb+cd+ to the directory containing the package's source code and type
     \verb+./configure+ to configure the package for your system.  If you're
     using \verb+csh+ on an old version of System V, you might need to type
     \verb+sh ./configure+ instead to prevent \verb+csh+ from trying to execute
     \verb+configure+ itself.

     Running \verb+configure+ takes awhile.  While running, it prints some
     messages telling which features it is checking for. It also prints
     a reminder for things to do after installation.

\item Type \verb+make+ to compile the package.

\item Type \verb+make install+ to install the programs and any data files and
     documentation.

\item You can remove the program binaries and object files from the
     source code directory by typing \verb+make clean+.  To also remove the
     files that \verb+configure+ created (so you can compile the package for
     a different kind of computer), type \verb+make distclean+.  There is
     also a \verb+make maintainer-clean+ target, but that is intended mainly
     for the package's developers.  If you use it, you may have to get
     all sorts of other programs in order to regenerate files that came
     with the distribution.

\item Since you've installed a library don't forget to run \verb+ldconfig+ !
\end{enumerate}
%%%%%%%%%%%%%%%%%%%%%%%%%%%%%%%%%%%%%%%%%%%%%%%%%%%%%%%%%%%%%
\section*{Installation without root privilege}
\index{norm}{Installation!w/o root privilege}
\label{inst_noroot}

Install \GLOBES\ to a directory of your choice \verb+GLB_DIR+. This is done by
\begin{quote}
\verb+configure --prefix=GLB_DIR+
\end{quote}
and then follow the usual installation guide. The only remaining problem
is that you have to tell the compiler where to find the header files, and 
the linker where to find the library. Furthermore you have to make sure 
that the shared object files are found during execution. Running 
\verb+configure+
also produces a \verb+Makefile+ in the \verb+examples+ 
subdirectory which can serve as
a template for the compilation and linking process, since all necessary
flags are correctly filled in. 

Another solution is to set the environment variable \verb+LD_RUN_PATH+
during linking to \verb+GLB_DIR/lib+ . Best thing is to add this to your
shell dot-file (e.g. \verb+.bashrc+). Then you can use:
A typical compiler command like
\begin{quote}
\verb+gcc -c my_program.c -IGLB_DIR/include/+
\end{quote}
and a typical linker command like
\begin{quote}
\verb#g++ my_program.o -lglobes -LGLB_DIR/lib/ -o my_executable#
\end{quote}
More information on this issue can be obtained by having a look 
into the mentioned  \verb+Makefile+ in \verb+examples+. 

\begin{itemize}
\item[CAVEAT] It is in principle possible to have many installations 
on one machine. Especially the situation of having an installation by root and by a user at the same time might occur. However, it is strictly warned against this possibility, since it is \emph{extremely} likely to create some versioning problem at some time!
\end{itemize}

\section*{Advanced topics}

\subsection*{Compilers and Options}


   Some systems require unusual options for compilation or linking that
the \verb+configure+ script does not know about.  You can give \verb+configure+
initial values for variables by setting them in the environment.  Using
a Bourne-compatible shell, you can do that on the command line like
this
\begin{quote}
\verb+CC=c89 CFLAGS=-O2 LIBS=-lposix ./configure+
\end{quote}
Or on systems that have the `env' program, you can do it like this
\begin{quote}
\verb+env CPPFLAGS=-I/usr/local/include LDFLAGS=-s ./configure+
\end{quote}

\subsection*{Compiling For Multiple Architectures}

   You can compile the package for more than one kind of computer at the
same time, by placing the object files for each architecture in their
own directory.  To do this, you must use a version of \verb+make+ that
supports the \verb+VPATH+ variable, such as GNU \verb+make+.  `cd' to the
directory where you want the object files and executables to go and run
the \verb+configure+ script.  \verb+configure+ automatically checks for the
source code in the directory that \verb+configure+ is in and in `..'.

   If you have to use a \verb+make+ that does not supports the \verb+VPATH+
variable, you have to compile the package for one architecture at a time
in the source code directory.  After you have installed the package for
one architecture, use `make distclean' before reconfiguring for another
architecture.

\subsection*{Installation Names}

By default, \verb+make install+ will install the package's files in
\verb+/usr/local/bin+, \verb+/usr/local/include+, etc.  You can specify an
installation prefix other than \verb+/usr/local+ by giving \verb+configure+ the
option \verb+--prefix=PATH+.



\subsection*{Specifying the System Type}


   There may be some features \verb+configure+ can not figure out
automatically, but needs to determine by the type of host the package
will run on.  Usually \verb+configure+ can figure that out, but if it prints
a message saying it can not guess the host type, give it the
\verb+--host=TYPE+ option.  TYPE can either be a short name for the system
type, such as `sun4', or a canonical name with three fields: 
\verb+CPU-COMPANY-SYSTEM+

See the file \verb+config.sub+ for the possible values of each field. 
 
   If you are building compiler tools for cross-compiling, you can also
use the \verb+--target=TYPE+ option to select the type of system they will
produce code for, and the \verb+--build=TYPE+ option to select the type of
system on which you are compiling the package.

\subsection*{Sharing Defaults}
   If you want to set default values for \verb+configure+ scripts to share,
you can create a site shell script called \verb+config.site+ that gives
default values for variables like \verb+CC+,  \verb+cache_file+, and 
\verb+prefix+.
\verb+configure+ looks for \verb+PREFIX/share/config.site+ if it exists, then
\verb+PREFIX/etc/config.site+ if it exists.  Or, you can set the
\verb+CONFIG_SITE+ environment variable to the location of the site script.
A warning: not all \verb+configure+ scripts look for a site script.

\index{norm}{Installation|)}


{\footnotesize
\input{gpl}

\input{fdl}
}


\end{appendix}

%%% Local Variables: 
%%% mode: latex
%%% TeX-master: Manual.tex
%%% End: