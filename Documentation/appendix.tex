%%%%%%%%%%%%%%%%%%%%%%%%%%%%%%%%%%%
% Appendix
%%%%%%%%%%%%%%%%%%%%%%%%%%%%%%%%%%%

\begin{appendix}

\chapter{\GLOBES\ installation}
\label{app:installation}
\index{norm}{Installation|(}

\section*{Prerequisites for installation of \GLOBES}

Besides the usual things like a working libc you need to have
 \begin{itemize}
        \item[gcc]      The GNU compiler collection\\
                        \verb+gcc.gnu.org+
        \item[GSL]      The GNU Scientific Library\\
                        \verb+www.gnu.org/software/gsl/+
\end{itemize}
The library \verb+libglobes+ should in principle compile with any C/C++
compiler but the \verb+globes+ binary uses the \verb+argp+ facility of \verb+glibc+
to parse its command line options. However, on platforms where \verb+argp+
is lacking \GLOBES\ has replacement code, thus it should also work
there. \GLOBES\ is, however, using the C99 standard in order to handle
complex numbers, but that is the only feature of C99 used.

GSL is also available as rpm+s from the various distributors of
GNU/Linux, see their web sites for downloads. Chances are that gcc and
GSL are already part of your installation.  For building \GLOBES\ from
source, however, not only working libraries for above packages are
needed but also the headers, especially for GSL. Depending on your
installation, eg. on RedHat/Fedora, of GSL this may require to
additionally install a rpm-package named \verb+gsl-devel+. If GSL has been
installed from the tar-ball as provided by gnu.org no problems should
occur. Furthermore you need a working \verb+make+ to build and install
\GLOBES.


\section*{Installation Instructions}


\GLOBES\ follows the standard GNU installation procedure.  To compile
\GLOBES\ you will need an ANSI C-compiler.  After unpacking the
distribution the Makefiles can be prepared using the configure
command,\\
\verb+  ./configure+\\
You can then build the library by typing,\\
\verb+  make+\\
A shared  version of the library will be compiled by
default. 
 
The libraries and modules can be installed using the command,\\
\verb+  make install+\\
The install target also will install a program with name \verb+globes+
 to \verb+/usr/local/bin+

The default install directory prefix is \verb+/usr/local+.  Consult the
"Further Information" section below for instructions on installing the
library in another location or changing other default compilation
options.

Moreover a config-script called \verb+globes-config+ will be
installed. This script displays all information necessary to link any
program with \GLOBES.  For building static libraries and linking against
them see the corresponding section of this file.
 
                    

\section*{Basic Installation}



   The \verb+configure+ shell script attempts to guess correct values for
various system-dependent variables used during compilation.  It
uses those values to create a \verb+Makefile+ in each directory of the
package.  It may also create one or more \verb+.h+ files containing
system-dependent definitions.  Finally, it creates a shell script
\verb+config.status+ that you can run in the future to recreate the
current configuration, a file \verb+config.cache+ that saves the results
of its tests to speed up reconfiguring, and a file \verb+config.log+
containing compiler output (useful mainly for debugging
\verb+configure+).

   If you need to do unusual things to compile the package, please try
to figure out how \verb+configure+ could check whether to do them, and mail
diffs or instructions to the address given in the \verb+README+ so they can
be considered for the next release.  If at some point \verb+config.cache+
contains results you don't want to keep, you may remove or edit it.

   The file \verb+configure.in+ is used to create \verb+configure+ by a program
called \verb+autoconf+.  You only need \verb+configure.in+ if you want to change
it or regenerate \verb+configure+ using a newer version of \verb+autoconf+.

The simplest way to compile this package is:
\begin{enumerate}
 \item \verb+cd+ to the directory containing the package's source code and type
     \verb+./configure+ to configure the package for your system.  If you're
     using \verb+csh+ on an old version of System V, you might need to type
     \verb+sh ./configure+ instead to prevent \verb+csh+ from trying to execute
     \verb+configure+ itself.

     Running \verb+configure+ takes awhile.  While running, it prints some
     messages telling which features it is checking for.
\item Type \verb+make+ to compile the package.

 \item Type \verb+make install+ to install the programs and any data files and
     documentation.

\item You can remove the program binaries and object files from the
     source code directory by typing \verb+make clean+.  To also remove the
     files that \verb+configure+ created (so you can compile the package for
     a different kind of computer), type \verb+make distclean+.  There is
     also a \verb+make maintainer-clean+ target, but that is intended mainly
     for the package's developers.  If you use it, you may have to get
     all sorts of other programs in order to regenerate files that came
     with the distribution.

\item Since you have installed a library don't forget to run \verb+ldconfig+!
\end{enumerate}
\section*{Installation without root privilege}


Install \GLOBES\ to a directory of your choice \verb+GLB_DIR+. This is done by\\
\verb+  configure --prefix=GLB_DIR+\\ 
and then follow the usual
installation guide. The only remaining problem is that you
have to tell the compiler where to find the header files, and
the linker where to find the library. Furthermore you have to
make sure that the shared object files are found during
execution. Running \verb+configure+ also produces a \verb+Makefile+ in
the examples subdirectory which can serve as a template for
the compilation and linking process, since all necessary flags
are correctly filled in. Another solution is to set the
environment variable \verb+LD_RUN_PATH+ during linking to
\verb+GLB_DIR/lib/+. Best thing is to add this to your shell dot-file
(e.g. \verb+.bashrc+). Then you can use: A typical compiler command
like\\
\verb+  gcc -c my_program.c -IGLB_DIR/include/+\\
and a typical linker command like\\
\verb+  gcc my_program.o -lglobes -LGLB_DIR/lib/ -o my_executable+\\
More information on this issue can be obtained by having a look into
the output of make install.

{\bf CAVEAT}: It is in principle possible to have many installations on one
machine, especially the situation of having an installation by root
and by a user at the same time might occur. However it is strictly
warned against this possibility since it is \emph{extremely} likely to
create some versioning problem at some time!

\section*{Building and Using static versions of \GLOBES}


Under certain circumstances it may be useful to use a static version of
libglobes or any of the binaries, e.g. when running on a cluster.

The \verb+configure+ script accepts the option \verb+--disable-shared+, in
which case only static objects are built, i.e. only a static
version of libglobes. In case your system does not support shared
libraries the \verb+configure+ script recognizes this. If you give no
options to \verb+configure+, both shared and static versions are built
and will be installed. All binaries, however, will use dynamic
linking. If you want to build static binaries, use
LDFLAGS=+-all-static+ for building them.

Sometimes it is convenient, eg. for debugging purposes, to have a
statically linked version of a program using \GLOBES\, which is easiest
achieved by just linking with \verb+libglobes.a+. If you need a completely
statically linked version, please, have a look at the Makefile in the
\verb+examples+ directory.\\
\verb+  make example-static+\\ 
produces a statically linked program that should in principle run on
most Linuxes.  It should be straightforward to adapt this example to
your needs.

All these options rely on a working \verb+gcc+ installation. It seems that
\verb+gcc 3.x+ is broken in subtle way which makes it necessary to add a
symbolic link in the gcc library directory. The diagnostic for this
requirement is that building static programs fails with the error
message \verb+cannot find -lgcc_s+. In those cases, find \verb+libgcc.a+ and add
a symbolic link in the same directory where you found it (this
requires probably root privileges)\\
\verb+  ln -s libgcc.a libgcc_s.a+

If you can not write to this directory just use the following work
around. Add the same link as above to the directory where you
installed \GLOBES\ into\\
\verb+  cd prefix/lib+\\       
\verb+  ln -s path_to_libgcc.a/libgcc.a libgcc_s.a+\\
and then change back into the \verb+examples+ directory and type\\
\verb+  make LDFLAGS=-Lprefix/lib example-static+\\
and you are done.

\section*{\GLOBES\ and Condor}
\label{sec:condor}
\index{norm}{Condor|(}
\begin{quote}
  Condor is a specialized workload management system for
  compute-intensive jobs. Like other full-featured batch systems,
  Condor provides a job queuing mechanism, scheduling policy,
  priority scheme, resource monitoring, and resource management.
\end{quote} 

A Condor (\verb+www.cs.wisc.edu/condor/+) cluster is very well suited
to run large \GLOBES-based computation. The nature of the problems
addressed with \GLOBES\ is such that one typically ends up with a so
called 'embarrassingly parallel' program. That means, that one repeats
the same task $N$ times, where each execution is independent of the
other $N-1$. Therefore, this execution should become $M$ times faster
if one uses $M$ processors. For this class of problems running on a
dedicated cluster will not improve performance (but may reduce latency
and such).

In order to fully exploit the functionality offered by Condor one
should submit the jobs into the so called 'standard universe'. To do
this, it is necessary to re-link the application with the
Condor-library (this assumes that Condor is installed)\\ 
\verb+  condor_compile gcc your_object_files -static `globes-config --libs`+\\
It may be necessary to prefix the call of \verb+globes-config+ with
the path to it, in case that location is not in \verb+$PATH+.
\index{norm}{Condor|)}

\section*{GSL requirements}

Sometimes the GNU scientific library is not available or is installed
in a non-standard location. This situation can arise in an
installation without root privileges. In this case one can specify
\verb+--with-gsl-prefix=path_to_gsl+ as option to the \verb+configure+ script.
If one wants to use a shared version of \verb+libgsl+ then one has to make
sure that the linker find the library at run-time. This can be
achieved by setting the environment variable \verb+LD_LIBRARY_PATH+
correctly, i.e. (in bash)\\
\verb+  export LD_LIBRARY_PATH='path_to_gsl'+\\
You also can use a static version of GSL by either building \GLOBES\
with \verb+LDFLAG='-all-static'+ or by configuring GSL with
\verb+--disable-shared+. In both cases no further actions like setting any
environment variables is necessary.

\section*{Distributions}


\subsection*{RedHat (all versions)}

The standard rpm-based installation of GSL does not provide any header
files for GSL, which are however needed to compile \GLOBES\. You have to
install an additional rpm-package called \verb+gsl-devel+. Alternatively
you can install GSL from a tar-ball and use the \verb+--with-gsl-prefix+
option to the configure script of \GLOBES.

\section*{Platforms}


\GLOBES\ builds and installs on 64bit Linux systems.
\GLOBES\ should work on Mac OS.

\subsection*{Windows}


Currently \GLOBES\ is only able to work under Cygwin \verb+www.cygwin.com+.
Inside Cygwin \GLOBES\ needs to be built with these commands\\
\verb+  configure+\\
\verb+  make LDFLAGS=-no-undefined'+

\section*{Compilers and Options}

   Some systems require unusual options for compilation or linking that
the \verb+configure+ script does not know about.  You can give \verb+configure+
initial values for variables by setting them in the environment.  Using
a Bourne-compatible shell, you can do that on the command line like
this\\
\verb+  CC=c89 CFLAGS=-O2 LIBS=-lposix ./configure+\\

Or on systems that have the \verb+env+ program, you can do it like this\\
\verb+  env CPPFLAGS=-I/usr/local/include LDFLAGS=-s ./configure+

\section*{Compiling For Multiple Architectures}


   You can compile the package for more than one kind of computer at the
same time, by placing the object files for each architecture in their
own directory.  To do this, you must use a version of \verb+make+ that
supports the \verb+VPATH+ variable, such as GNU \verb+make+.  \verb+cd+ to the
directory where you want the object files and executables to go and run
the \verb+configure+ script.  \verb+configure+ automatically checks for the
source code in the directory that \verb+configure+ is in and in \verb+..+.

   If you have to use a \verb+make+ that does not supports the \verb+VPATH+
variable, you have to compile the package for one architecture at a time
in the source code directory.  After you have installed the package for
one architecture, use \verb+make distclean+ before reconfiguring for another
architecture.

\section*{Installation Names}


   By default, \verb+make install+ will install the package+s files in
\verb+/usr/local/bin+, \verb+/usr/local/man+, etc.  You can specify an
installation prefix other than \verb+/usr/local+ by giving \verb+configure+ the
option \verb+--prefix=PATH+.

   You can specify separate installation prefixes for
architecture-specific files and architecture-independent files.  If you
give \verb+configure+ the option \verb+--exec-prefix=PATH+, the package will use
PATH as the prefix for installing programs and libraries.
Documentation and other data files will still use the regular prefix.

   In addition, if you use an unusual directory layout you can give
options like \verb+--bindir=PATH+ to specify different values for particular
kinds of files.  Run \verb+configure --help+ for a list of the directories
you can set and what kinds of files go in them.

   If the package supports it, you can cause programs to be installed
with an extra prefix or suffix on their names by giving \verb+configure+ the
option \verb+--program-prefix=PREFIX+ or \verb+--program-suffix=SUFFIX+.

\section*{Optional Features}


   Some packages pay attention to \verb+--enable-FEATURE+ options to
\verb+configure+, where FEATURE indicates an optional part of the package.
They may also pay attention to \verb+--with-PACKAGE+ options, where PACKAGE
is something like \verb+gnu-as+ or \verb+x+ (for the X Window System).  The
\verb+README+ should mention any \verb+--enable-+ and \verb+--with-+ options that the
package recognizes.

   For packages that use the X Window System, \verb+configure+ can usually
find the X include and library files automatically, but if it doesn't,
you can use the \verb+configure+ options \verb+--x-includes=DIR+ and
\verb+--x-libraries=DIR+ to specify their locations.


\subsection*{Building a perl extension}


This feature is experimental and your mileage my vary!

This feature allows to build a perl binding of \GLOBES\, i.e. you will
in the end have a perl module from which you can use \GLOBES\ from within
any perl program.

If(!) everything works as intended, all you have to do is to provide
\verb+--enable-perl+ to \verb+configure+ and type \verb+make install+. Now have a
look at \verb+globes/example.pl+ and you should see how that works in
principle. 

The trick here, is that we use SWIG (\verb+www.swig.org+) to generate
a wrapper file for \GLOBES. The wrapper file is part of the \GLOBES\
tar-ball (\verb+globes/globes_perl.c+) and hence you should not need SWIG to
be installed on your system.

All the tricks employed to get perl extension working should in some
form be applicable to building other extensions, like python. If
you want to try that you will need SWIG.

\subsection*{Building RPMs}


This feature is experimental and your mileage my vary!

Many people find binary RPMs useful, therefore we provide an optional
feature \verb+--enable-rpm-rules+ which should produce all the necessary
makefile rules for RPM building. To actually build RPMs requires that
your system is properly setup for that. How you can do that you can
learn at \verb+http://www.rpm.org+. You then can use \verb+make rpm+, most likely
you will need to be root to do that (\verb+sudo+ won't work!).

{\bf NOTE} to people packaging \GLOBES\ RPMs: Please, use the provided spec
file and do include the headers!

\section*{Specifying the System Type}


   There may be some features \verb+configure+ can not figure out
automatically, but needs to determine by the type of host the package
will run on.  Usually \verb+configure+ can figure that out, but if it prints
a message saying it can not guess the host type, give it the
\verb+--host=TYPE+ option.  TYPE can either be a short name for the system
type, such as \verb+sun4+, or a canonical name with three fields\\
\verb+  CPU-COMPANY-SYSTEM+

See the file \verb+config.sub+ for the possible values of each field.  If
\verb+config.sub+ isn't included in this package, then this package doesn't
need to know the host type.

   If you are building compiler tools for cross-compiling, you can also
use the \verb+--target=TYPE+ option to select the type of system they will
produce code for and the \verb+--build=TYPE+ option to select the type of
system on which you are compiling the package.

\section*{Sharing Defaults}

   If you want to set default values for \verb+configure+ scripts to share,
you can create a site shell script called \verb+config.site+ that gives
default values for variables like \verb+CC+, \verb+cache_file+, and \verb+prefix+.
\verb+configure+ looks for \verb+PREFIX/share/config.site+ if it exists, then
\verb+PREFIX/etc/config.site+ if it exists.  Or, you can set the
\verb+CONFIG_SITE+ environment variable to the location of the site script.
A warning: not all \verb+configure+ scripts look for a site script.


\index{norm}{Installation|)}




%%%%%%%%%%%%%%%%%%%%%%%%%%%%%%%%%%%%%%%%%%%%%
\chapter{Flux normalization in \GLOBES }
\index{norm}{Normalization of fluxes}
\label{app:flux}

A common issue with \GLOBES\ is confusion about the proper units for
the input flux files for use in \AEDL\ experiment descriptions. Source of
the confusion is an undocumented factor 5.2 in \GLOBES\ versions older
than 3.0 the fluxes are multiplied with (see below). In Version 3.0 and 
higher, the alternative flux environment {\tt nuflux} is provided, 
which does not contain this factor. The following material is
based on the old environment {\tt flux}. For the use of {\tt nuflux},
replace the factor 5.2 by unity.

\section*{Historical problem}

One problem for the design of \AEDL\ was initially that meaningful
units for flux data strongly depend on the given type of experiment,
but also on relatively arbitrary decisions. For accelerator beams
based on pion decay, one frequently defines the beam luminosity in
{\it protons on target (pot)} since this number has a one-to-one
correspondence with the number of neutrinos produced. Another sensible
unit could be {\it megawatt on target (MW)}, again this number is directly
correlated with the number of neutrinos and moreover there is a unique
relation to {\it pot} for a given accelerator. Of course,
what matters is the integrated luminosity. In some  cases the
neutrino flux is given per $10^7\,\mathrm{s}$. However,
most experiments will run for several years, hence also this number
has to enter somewhere. For neutrino factories the proper number is
useful muon decays per unit time, and for reactor experiments it is
the thermal power of the reactor {\it asf}. This demonstrates that it is
reasonable to keep the flux definition flexible.

\section*{Implementation in \GLOBES }

In understanding how one still can figure out what the correct units
are for each case, it is a good starting point to look at what
\GLOBES\ does with the input files. The cross section in the file is given as
differential cross section divided by energy $x=\sigma/E$, and the flux file
gives $f$. The differential number of events per GeV $n$ as computed in
\GLOBES\ without oscillation and efficiencies is given by
\begin{eqnarray}
n &=& 5.2\times x\times E\times f\times \nonumber \\
&&\mathtt{@norm}\times\mathtt{@power}\times\mathtt{@stored\_muons}\times\mathtt{@time}\times\mathtt{\$target\_mass}\times(\mathtt{\$baseline})^{-2} \nonumber
\end{eqnarray}
Note that $5.2$ is a undocumented fudge factor!

It is the sole responsibility of the author of the \AEDL\ file and its
supporting files, to ensure that the result makes sense. In
principle, it is possible to divide, for example, {\tt @time}  by
$\pi$ and fix that by redefining the flux file by multiplying
it with $\pi$. Modifications like that have happened in the past
and still happen, and many of them are not properly commented.
 
\section*{Writing \AEDL\ files}

The task is to choose the value of  {\tt @norm} such that all the
variables in the \AEDL\ file have the proper units, \eg , {\tt
  @time} has proper unit years.

\GLOBES\ assumes that the cross
section $x$ is given in $10^{-38}\,\mathrm{cm}^2$ and that all fluxes are
given at a distance of $1\,\mathrm{km}$. In addition, it assumes that
the number of target nuclei $\tau$ (or protons or whatever applies to the
given cross section) per unit target mass $m_u$ (which usually is $kt$)
 are properly accounted for.\footnote{Note that the cross sections which are delivered with
  \GLOBES\ always are per nucleon.}  Assuming that in the flux file
the data is given as number of neutrinos per unit area $A$ and
energy bin of width $\Delta E$ at a distance $L$ from the source, one
obtains
\begin{eqnarray}
\mathtt{@norm}=\frac{1}{5.2}\left(\frac{\mathrm{GeV}}{\Delta
      E}\right)
\left(\frac{\mathrm{cm}^2}{A}\right)\left(\frac{L}{\mathrm{km}}\right)^2\left(\frac{\tau}{m_u}\right)\times10^{-38}\times\left(\frac{\mathcal{L}_u}{\mathcal{L}}\right)
\end{eqnarray}
where $\mathcal{L}$ absorbs all factors in the flux file related to
the integrated luminosity, and $\mathcal{L}_u$ is the unit chosen for
it.  The concept of integrated luminosity is
nicely described in the \GLOBES\ manual in \Sec~\ref{sec:source}.
A little example illustrates this concept:
The flux is given for $10^{21}\, \mathrm{pot}\,\mathrm{y}^{-1}$ of
$10\,\mathrm{GeV}$ protons, thus a good choice for the units $\mathcal{L}_u$
is $\mathrm{MW}\,\mathrm{y}^{-1}$, which means that
$\mathcal{L}/\mathcal{L}_u$ is given by (assuming a $10^7\,\mathrm{s}$
year)
\begin{equation}
\frac{\mathcal{L}}{\mathcal{L}_u}=\frac{10\,\mathrm{GeV}\,10^{21}\,\mathrm{pot}\,\mathrm{y}}{10^7\,\mathrm{s}}\times(\mathrm{MW}\,\mathrm{y}^{-1})^{-1}=0.16\ldots
\end{equation}

\section*{Moving from {\tt flux} to {\tt nuflux}}

In order to change the older {\tt flux} environment to the
new {\tt nuflux} (\GLOBES\ 3.0 and higher), replace all user-defined fluxes, such as
\begin{quote}
{\tt flux(\#user)<}\\
{\tt \tb @flux\_file = "user\_file\_1.dat"\\
\tb @time = 2.0\\
\tb @power = 4.0\\
\tb @norm = 1e+8}\\
{\tt >}
\end{quote}
by
\begin{quote}
{\tt FF=5.1989} \\
\\
{\tt nuflux(\#user)<}\\
{\tt \tb @flux\_file = "user\_file\_1.dat"\\
\tb @time = 2.0\\
\tb @power = 4.0\\
\tb @norm = FF*1e+8}\\
{\tt >}
\end{quote}

This replacement is not necessary for neutrino factory built-in fluxes,
and built-in beta beams fluxes where not supported by earlier versions
of \GLOBES .

IS THAT CORRECT? MR: WHAT HAPPENS TO OLD BETA BEAM FILES [WW]?


%%%%%%%%%%%%%%%%%%%%%%%%%%%%%%%%%%%%%%%%%%%%

{\footnotesize
\input{gpl}

\input{fdl}
}


\end{appendix}

%%% Local Variables: 
%%% mode: latex
%%% TeX-master: Manual.tex
%%% TeX-master: "Manual"
%%% End: