%%%%%%%%%%%%%%%%%%%%%%%%%%%%%%%%%%%
% Appendix
%%%%%%%%%%%%%%%%%%%%%%%%%%%%%%%%%%%

%\begin{appendix}
\appendix
\chapter{\GLOBES\ installation}
\label{app:installation}
\index{norm}{Installation|(}

\section{Prerequisites for installation of \GLOBES}

Besides the usual things like a working libc you need to have
 \begin{itemize}
        \item[gcc]      The GNU compiler collection\\
                        \verb+gcc.gnu.org+
        \item[GSL]      The GNU Scientific Library\\
                        \verb+www.gnu.org/software/gsl/+
\end{itemize}
The library \verb+libglobes+ should in principle compile with any C/C++
compiler but the \verb+globes+ binary uses the \verb+argp+ facility of \verb+glibc+
to parse its command line options. However, on platforms where \verb+argp+
is lacking, \GLOBES\ has replacement code, thus it should also work
there. \GLOBES\ is, however, using the C99 standard in order to handle
complex numbers, but that is the only feature of C99 used.

GSL is also available as rpm's from the various distributors of
GNU/Linux, see their web sites for downloads. Chances are that gcc and
GSL are already part of your installation.  For building \GLOBES\ from
source, however, not only working libraries for the above packages are
needed, but also the headers, especially for GSL. For some installations
of GSL, eg. on RedHat/Fedora, this may require to
additionally install a rpm-package named \verb+gsl-devel+. If GSL has been
installed from the tar-ball as provided by gnu.org, no problems should
occur. Furthermore you need a working \verb+make+ to build and install
\GLOBES.


\section{Installation Instructions}


\GLOBES\ follows the standard GNU installation procedure.  To compile
\GLOBES\ you will need an ANSI C-compiler.  After unpacking the
distribution, the Makefiles can be prepared using the configure
command,\\
\verb+  ./configure+\\
You can then build the library by typing,\\
\verb+  make+\\
A shared  version of the library will be compiled by
default. 
 
The libraries and modules can be installed using the command,\\
\verb+  make install+\\
The install target also will install a program with name \verb+globes+
 to \verb+/usr/local/bin+

The default install directory prefix is \verb+/usr/local+.  Consult the
"Further Information" section below for instructions on installing the
library in another location or changing other default compilation
options.

Moreover a config-script called \verb+globes-config+ will be
installed. This script displays all information necessary to link any
program with \GLOBES.  For building static libraries and linking against
them see the corresponding section of this file.
 
                    

\section*{Basic Installation}



   The \verb+configure+ shell script attempts to guess correct values for
various system-dependent variables used during compilation.  It
uses those values to create a \verb+Makefile+ in each directory of the
package.  It may also create one or more \verb+.h+ files containing
system-dependent definitions.  Finally, it creates a shell script
\verb+config.status+ that you can run in the future to recreate the
current configuration, a file \verb+config.cache+ that saves the results
of its tests to speed up reconfiguring, and a file \verb+config.log+
containing compiler output (useful mainly for debugging
\verb+configure+).

   If you need to do unusual things to compile the package, please try
to figure out how \verb+configure+ could check whether to do them, and mail
diffs or instructions to the address given in the \verb+README+ so they can
be considered for the next release.  If at some point \verb+config.cache+
contains results you don't want to keep, you may remove or edit it.

   The file \verb+configure.in+ is used to create \verb+configure+ by a program
called \verb+autoconf+.  You only need \verb+configure.in+ if you want to change
it or regenerate \verb+configure+ using a newer version of \verb+autoconf+.

The simplest way to compile this package is:
\begin{enumerate}
 \item \verb+cd+ to the directory containing the package's source code and type
     \verb+./configure+ to configure the package for your system.  If you're
     using \verb+csh+ on an old version of System V, you might need to type
     \verb+sh ./configure+ instead to prevent \verb+csh+ from trying to execute
     \verb+configure+ itself.

     Running \verb+configure+ takes awhile.  While running, it prints some
     messages telling which features it is checking for.
\item Type \verb+make+ to compile the package.

 \item Type \verb+make install+ to install the programs and any data files and
     documentation.

\item You can remove the program binaries and object files from the
     source code directory by typing \verb+make clean+.  To also remove the
     files that \verb+configure+ created (so you can compile the package for
     a different kind of computer), type \verb+make distclean+.  There is
     also a \verb+make maintainer-clean+ target, but that is intended mainly
     for the package's developers.  If you use it, you may have to get
     all sorts of other programs in order to regenerate files that came
     with the distribution.

\item Since you have installed a library don't forget to run \verb+ldconfig+!
\end{enumerate}
\section*{Installation without root privilege}


Install \GLOBES\ to a directory of your choice \verb+GLB_DIR+. This is done by\\
\verb+  configure --prefix=GLB_DIR+\\ 
and then follow the usual
installation guide. The only remaining problem is that you
have to tell the compiler where to find the header files, and
the linker where to find the library. Furthermore you have to
make sure that the shared object files are found during
execution. Running \verb+configure+ also produces a \verb+Makefile+ in
the examples subdirectory which can serve as a template for
the compilation and linking process, since all necessary flags
are correctly filled in. Another solution is to set the
environment variable \verb+LD_RUN_PATH+ during linking to
\verb+GLB_DIR/lib/+. Best thing is to add this to your shell dot-file
(e.g. \verb+.bashrc+). Then you can use: A typical compiler command
like\\
\verb+  gcc -c my_program.c -IGLB_DIR/include/+\\
and a typical linker command like\\
\verb+  gcc my_program.o -lglobes -LGLB_DIR/lib/ -o my_executable+\\
More information on this issue can be obtained by having a look into
the output of make install.

{\bf CAVEAT}: It is in principle possible to have many installations on one
machine, especially the situation of having an installation by root
and by a user at the same time might occur. However it is strictly
warned against this possibility since it is \emph{extremely} likely to
create some versioning problem at some time!

\section*{Building and Using static versions of \GLOBES}


Under certain circumstances it may be useful to use a static version of
libglobes or any of the binaries, e.g.\ when running on a cluster.

The \verb+configure+ script accepts the option \verb+--disable-shared+, in
which case only static objects are built, i.e.\ only a static
version of libglobes. In case your system does not support shared
libraries the \verb+configure+ script recognizes this. If you give no
options to \verb+configure+, both shared and static versions are built
and will be installed. All binaries, however, will use dynamic
linking. If you want to build static binaries, use
\verb+LDFLAGS='-all-static'+ for building them.

Sometimes it is convenient, eg. for debugging purposes, to have a
statically linked version of a program using \GLOBES, which is easiest
achieved by just linking with \verb+libglobes.a+. If you need a completely
statically linked version, please, have a look at the Makefile in the
\verb+examples+ directory.\\
\verb+  make example-static+\\ 
produces a statically linked program that should in principle run on
most Linuxes.  It should be straightforward to adapt this example to
your needs.

All these options rely on a working \verb+gcc+ installation. It seems that
\verb+gcc 3.x+ is broken in a subtle way which makes it necessary to add a
symbolic link in the gcc library directory. The diagnostic for this
requirement is that building static programs fails with the error
message \verb+cannot find -lgcc_s+. In those cases, find \verb+libgcc.a+ and add
a symbolic link in the same directory where you found it (this
requires probably root privileges)\\
\verb+  ln -s libgcc.a libgcc_s.a+

If you can not write to this directory just use the following work
around. Add the same link as above to the directory where you
installed \GLOBES\ into\\
\verb+  cd prefix/lib+\\       
\verb+  ln -s path_to_libgcc.a/libgcc.a libgcc_s.a+\\
and then change back into the \verb+examples+ directory and type\\
\verb+  make LDFLAGS=-Lprefix/lib example-static+\\
and you are done.

\section*{\GLOBES\ and Condor}
\label{sec:condor}
\index{norm}{Condor|(}
\begin{quote}
  Condor is a specialized workload management system for
  compute-intensive jobs. Like other full-featured batch systems,
  Condor provides a job queuing mechanism, scheduling policy,
  priority scheme, resource monitoring, and resource management.
\end{quote} 

A Condor (\verb+www.cs.wisc.edu/condor/+) cluster is very well suited
to run large \GLOBES-based computation. The nature of the problems
addressed with \GLOBES\ is such that one typically ends up with a so
called 'embarrassingly parallel' program. That means, that one repeats
the same task $N$ times, where each execution is independent of the
other $N-1$. Therefore, this execution should become $M$ times faster
if one uses $M$ processors. For this class of problems running on a
dedicated cluster will not improve performance (but may reduce latency
and such).

In order to fully exploit the functionality offered by Condor one
should submit the jobs into the so called 'standard universe'. To do
this, it is necessary to re-link the application with the
Condor-library (this assumes that Condor is installed)\\ 
\verb+  condor_compile gcc your_object_files -static `globes-config --libs`+\\
It may be necessary to prefix the call of \verb+globes-config+ with
the path to it, in case that this location is not in \verb+$PATH+.
\index{norm}{Condor|)}

\section*{GSL requirements}

Sometimes, the GNU scientific library is not available or is installed
in a non-standard location. This situation can arise in an
installation without root privileges. In this case, one can specify
\verb+--with-gsl-prefix=path_to_gsl+ as option to the \verb+configure+ script.
If one wants to use a shared version of \verb+libgsl+ then one has to make
sure that the linker can find the library at run-time. This can be
achieved by setting the environment variable \verb+LD_LIBRARY_PATH+
correctly, i.e. (in bash)\\
\verb+  export LD_LIBRARY_PATH='path_to_gsl'+\\
You also can use a static version of GSL by either building \GLOBES\
with \verb+LDFLAG='-all-static'+ or by configuring GSL with
\verb+--disable-shared+. In both cases no further actions like setting any
environment variables is necessary.

\section*{Distributions}


\subsection*{RedHat (all versions)}

The standard rpm-based installation of GSL does not provide any header
files for GSL, which are however needed to compile \GLOBES. You have to
install an additional rpm-package called \verb+gsl-devel+. Alternatively
you can install GSL from a tar-ball and use the \verb+--with-gsl-prefix+
option to the configure script of \GLOBES.

\section*{Platforms}


\GLOBES\ builds and installs on 64bit Linux systems.
\GLOBES\ should work on Mac OS.

\subsection*{Windows}


Currently \GLOBES\ is only able to work under Cygwin \verb+www.cygwin.com+.
Inside Cygwin \GLOBES\ needs to be built with these commands\\
\verb+  configure+\\
\verb+  make LDFLAGS=-no-undefined'+

\section*{Compilers and Options}

   Some systems require unusual options for compilation or linking that
the \verb+configure+ script does not know about.  You can give \verb+configure+
initial values for variables by setting them in the environment.  Using
a Bourne-compatible shell, you can do that on the command line like
this\\
\verb+  CC=c89 CFLAGS=-O2 LIBS=-lposix ./configure+\\

Or on systems that have the \verb+env+ program, you can do it like this\\
\verb+  env CPPFLAGS=-I/usr/local/include LDFLAGS=-s ./configure+

\section*{Compiling For Multiple Architectures}


   You can compile the package for more than one kind of computer at the
same time, by placing the object files for each architecture in their
own directory.  To do this, you must use a version of \verb+make+ that
supports the \verb+VPATH+ variable, such as GNU \verb+make+.  \verb+cd+ to the
directory where you want the object files and executables to go and run
the \verb+configure+ script.  \verb+configure+ automatically checks for the
source code in the directory that \verb+configure+ is in and in \verb+..+.

   If you have to use a \verb+make+ that does not supports the \verb+VPATH+
variable, you have to compile the package for one architecture at a time
in the source code directory.  After you have installed the package for
one architecture, use \verb+make distclean+ before reconfiguring for another
architecture.

\section*{Installation Names}


   By default, \verb+make install+ will install the package's files in
\verb+/usr/local/bin+, \verb+/usr/local/man+, etc.  You can specify an
installation prefix other than \verb+/usr/local+ by giving \verb+configure+ the
option \verb+--prefix=PATH+.

   You can specify separate installation prefixes for
architecture-specific files and architecture-independent files.  If you
give \verb+configure+ the option \verb+--exec-prefix=PATH+, the package will use
\verb+PATH+ as the prefix for installing programs and libraries.
Documentation and other data files will still use the regular prefix.

   In addition, if you use an unusual directory layout you can give
options like \verb+--bindir=PATH+ to specify different values for particular
kinds of files.  Run \verb+configure --help+ for a list of the directories
you can set and what kinds of files go in them.

   If the package supports it, you can cause programs to be installed
with an extra prefix or suffix on their names by giving \verb+configure+ the
option \verb+--program-prefix=PREFIX+ or \verb+--program-suffix=SUFFIX+.

\section*{Optional Features}


   Some packages pay attention to \verb+--enable-FEATURE+ options to
\verb+configure+, where \verb+FEATURE+ indicates an optional part of the package.
They may also pay attention to \verb+--with-PACKAGE+ options, where \verb+PACKAGE+
is something like \verb+gnu-as+ or \verb+x+ (for the X Window System).  The
\verb+README+ should mention any \verb+--enable-+ and \verb+--with-+ options that the
package recognizes.

   For packages that use the X Window System, \verb+configure+ can usually
find the X include and library files automatically, but if it doesn't,
you can use the \verb+configure+ options \verb+--x-includes=DIR+ and
\verb+--x-libraries=DIR+ to specify their locations.


\subsection*{Building a perl extension}


This feature is experimental and your mileage may vary!

This feature allows to build a perl binding of \GLOBES\, i.e. you will
in the end have a perl module from which you can use \GLOBES\ from within
any perl program.

If(!) everything works as intended, all you have to do is to provide
\verb+--enable-perl+ to \verb+configure+ and type \verb+make install+. Now have a
look at \verb+globes/example.pl+ and you should see how that works in
principle. 

The trick here is, that we use SWIG (\verb+www.swig.org+) to generate
a wrapper file for \GLOBES. The wrapper file is part of the \GLOBES\
tar-ball (\verb+globes/globes_perl.c+) and hence you should not need SWIG to
be installed on your system.

All the tricks employed to get perl extension working should in some
form be applicable to building other extensions, like python. If
you want to try that you will need SWIG.

\subsection*{Building RPMs}


This feature is experimental and your mileage may vary!

Many people find binary RPMs useful, therefore we provide an optional
feature \verb+--enable-rpm-rules+ which should produce all the necessary
Makefile rules for RPM building. To actually build RPMs requires that
your system is properly setup for that. You can learn how to do that
at \verb+http://www.rpm.org+. You then can use \verb+make rpm+, most likely
you will need to be root to do that (\verb+sudo+ won't work!).

{\bf NOTE} to people packaging \GLOBES\ RPMs: Please, use the provided spec
file and do include the headers!

\section*{Specifying the System Type}


   There may be some features \verb+configure+ can not figure out
automatically, but needs to determine by the type of host the package
will run on.  Usually \verb+configure+ can figure that out, but if it prints
a message saying it can not guess the host type, give it the
\verb+--host=TYPE+ option. \verb+TYPE+ can either be a short name for the system
type, such as \verb+sun4+, or a canonical name with three fields\\
\verb+  CPU-COMPANY-SYSTEM+

See the file \verb+config.sub+ for the possible values of each field.  If
\verb+config.sub+ isn't included in this package, then this package doesn't
need to know the host type.

   If you are building compiler tools for cross-compiling, you can also
use the \verb+--target=TYPE+ option to select the type of system they will
produce code for and the \verb+--build=TYPE+ option to select the type of
system on which you are compiling the package.

\section*{Sharing Defaults}

   If you want to set default values for \verb+configure+ scripts to share,
you can create a site shell script called \verb+config.site+ that gives
default values for variables like \verb+CC+, \verb+cache_file+, and \verb+prefix+.
\verb+configure+ looks for \verb+PREFIX/share/config.site+ if it exists, then
\verb+PREFIX/etc/config.site+ if it exists.  Or, you can set the
\verb+CONFIG_SITE+ environment variable to the location of the site script.
A warning: not all \verb+configure+ scripts look for a site script.


\index{norm}{Installation|)}


%%%%%%%%%%%%%%%%%%%%%%%%%%%%%%%%%%%%%%%%%%%%%
\chapter{Catalogue of {\sf AEDL}-Files}
\label{app:aedlfiles}
Along with the \GLOBES\ package comes a catalogue of pre-defined experiment \AEDL\ files for different
future experiments and different beam and detector technologies. These include the planned superbeam experiments
and their possible upgrades, different reactor experiment setups, different $\beta$-beam setups, and different 
neutrino factory setups. A complete list
of all pre-defined experiment files can be found in \tabl{experiments}. More detailed descriptions of the
corresponding files, the assumptions, requirements, and references are given in the following. 
   
\section{Superbeam experiments}
\subsection*{T2K -- {\tt T2K.glb}}

The \TtoK\ experiment can be simulated with the file {\tt T2K.glb}. This file tries to approximate as closely as
possible the LOI \cite{Itow:2001ee}, and the basic version was used within \cite{Huber:2002mx}. These references
should be cited if the file {\tt T2K.glb} is used for a scientific publication or a talk. For calculations that
involve {\tt T2K.glb}, the following additional files are required: 
\bi
\item {\tt JHFplus.dat} (neutrino flux from J-PARC -- $\nu_\mu$)
\item {\tt JHFminus.dat} (neutrino flux from J-PARC -- $\bar{\nu}_\mu$)
\item {\tt XCC.dat} (charged current cross sections)
\item {\tt XNC.dat} (neutral current cross sections)
\item {\tt XQE.dat} (quasi elastic cross sections)
\ei
The \TtoK\ neutrino beam is produced at J-PARC and directed towards the Super-Kamiokande detector. The target
power is 0.77~MW and 2~years $\nu$-running and 6~years $\bar{\nu}$-running is assumed. The fiducial mass of the
Super-Kamiokande Water Cerenkov detector is taken to be $\mathrm{m_{det} = 22.5 \,kt}$ at a baseline of
L~=~295~km. The appearance measurement involves the total rates data from all CC events and the spectral data
from the QE sample with a free normalization at an energy resolution of $\mathrm{\sigma_e=0.085\, GeV}$ due to the
Fermi Motion. The normalization of the QE samples is kept free in order to avoid double counting of events since all QE
events are also contained in the CC samples. The free normalization is introduced within a rule with the line 
{\tt \@signalerror = 10.:0.0001}. In case of systematics switched on the systematics functions {\tt chiSpectrumTilt}
for the QE sample and {\tt chiTotalRatesTilt} for the CC sample are used. However, for systematics switched off, only the
systematics function for the CC sample is changed to {\tt chiNoSysTotalRates}, but the systematics function for the QE
sample stays {\tt chiSpectrumTilt}. Note, that otherwise the free normalization would be swiched off and all events from
the QE sample would be counted twice. A detailed discussion of this QE/CC sample splitting can be found in~\cite{Huber:2002mx}. 
For the disappearance channels only the QE sample is used since statistics is already quite large and a treatment like for the
appearance channels would only shlightly modify the results. The quantitative treatment of systematics is similar to \cite{Ishitsuka:2005qi}. The following rules are 
defined within {\tt T2K.glb}:  
\begin{center}
\begin{tabular}{|l|ll|c|c|}
\hline \hline
\multicolumn{3}{|l|}{Disappearance (+)} & $\sigma_\mathrm{norm}$ & $\sigma_\mathrm{cal}$ \\ \hline 
Signal & $0.9 \, \otimes \, (\nu_\mu\rightarrow\nu_\mu)_{\mathrm{QE}}$ & & 0.025 & $10^{-4}$\\
 & &  & &\\
Background & $0.0056 \, \otimes \, (\nu_\mu \rightarrow \nu_x)_\mathrm{NC}$ & & 0.2 & $10^{-4}$ \\ \hline \hline 
\multicolumn{3}{|l|}{Appearance (+) -- Spectrum}  & & \\ \hline
Signal & $0.505 \, \otimes \, (\nu_\mu \rightarrow \nu_e)_\mathrm{QE}$ & & 10.0 & $10^{-4}$\\
 & & & & \\
Background & $0.0056 \, \otimes \, (\nu_\mu \rightarrow \nu_x)_\mathrm{NC}$ & $3.3\cdot 10^{-4} \, \otimes \, (\nu_\mu\rightarrow\nu_\mu)_{\mathrm{CC}}$ & 0.05 & 0.05\\
Beam background & $0.505 \, \otimes \, (\nu_e\rightarrow \nu_e)_\mathrm{CC}$ & $0.505 \, \otimes \, (\bar{\nu}_e\rightarrow \bar{\nu}_e)_\mathrm{CC}$  & 0.05 & 0.05\\ \hline \hline
\multicolumn{3}{|l|}{Appearance (+) -- Total Rates}  & & \\ \hline
Signal & $0.505 \, \otimes \, (\nu_\mu \rightarrow \nu_e)_\mathrm{CC}$ & & 0.05 & $10^{-4}$\\
 & & & & \\
Background & $0.0056 \, \otimes \, (\nu_\mu \rightarrow \nu_x)_\mathrm{NC}$ & $3.3\cdot 10^{-4} \, \otimes \, (\nu_\mu\rightarrow\nu_\mu)_{\mathrm{CC}}$ & 0.05 & $10^{-4}$\\
Beam background & $0.505 \, \otimes \, (\nu_e\rightarrow \nu_e)_\mathrm{CC}$ & $0.505 \, \otimes \, (\bar{\nu}_e\rightarrow \bar{\nu}_e)_\mathrm{CC}$  & 0.05 & $10^{-4}$\\ \hline \hline
\multicolumn{3}{|l|}{Disappearance (--)} & & \\ \hline 
Signal & $0.9 \, \otimes \, (\nu_\mu\rightarrow\nu_\mu)_{\mathrm{QE}}$ & & 0.025 & $10^{-4}$\\
 & &  & &\\
Background & $0.0056 \, \otimes \, (\nu_\mu \rightarrow \nu_x)_\mathrm{NC}$ & & 0.2 & $10^{-4}$ \\ \hline \hline 
\multicolumn{3}{|l|}{Appearance (--) -- Spectrum}  & & \\ \hline
Signal & $0.505 \, \otimes \, (\bar{\nu}_\mu \rightarrow \bar{\nu}_e)_\mathrm{QE}$ & & 10.0 & $10^{-4}$\\
 & & & & \\
Background & $0.0056 \, \otimes \, (\bar{\nu}_\mu \rightarrow \bar{\nu}_x)_\mathrm{NC}$ & $3.3\cdot 10^{-4} \, \otimes \, (\bar{\nu}_\mu\rightarrow\bar{\nu}_\mu)_{\mathrm{CC}}$ & 0.05 & 0.05\\
Beam background & $0.505 \, \otimes \, (\bar{\nu}_e\rightarrow \bar{\nu}_e)_\mathrm{CC}$ & $0.505 \, \otimes \, (\nu_e\rightarrow \nu_e)_\mathrm{CC}$  & 0.05 & 0.05\\ \hline \hline
\multicolumn{3}{|l|}{Appearance (--) -- Total Rates}  & & \\ \hline
Signal & $0.505 \, \otimes \, (\bar{\nu}_\mu \rightarrow \bar{\nu}_e)_\mathrm{CC}$ & & 0.05 & $10^{-4}$\\
 & & & & \\
Background & $0.0056 \, \otimes \, (\bar{\nu}_\mu \rightarrow \bar{\nu}_x)_\mathrm{NC}$ & $3.3\cdot 10^{-4} \, \otimes \, (\bar{\nu}_\mu\rightarrow\bar{\nu}_\mu)_{\mathrm{CC}}$ & 0.05 & $10^{-4}$\\
Beam background & $0.505 \, \otimes \, (\bar{\nu}_e\rightarrow \bar{\nu}_e)_\mathrm{CC}$ & $0.505 \, \otimes \, (\nu_e\rightarrow \nu_e)_\mathrm{CC}$  & 0.05 & $10^{-4}$\\ \hline \hline
\end{tabular}
\end{center}

\subsection*{T2HK -- {\tt T2HK.glb}}

\TtoHK\ is the superbeam upgrade of the \TtoK\ experiment and can be simulated with the file {\tt T2HK.glb}. 
The target power is 4~MW and 
4~years $\nu$-running and 4~years $\bar{\nu}$-running is assumed.
The fiducial mass of the Water Cerenkov detector is taken to be $\mathrm{m_{det} = 500 \,kt}$ at the same
baseline as the \TtoK\ experiment. Besides these changes, the file {\tt T2HK.glb} is similar to {\tt T2K.glb}
and the same additional files are required. 
The basic version was used within \cite{Huber:2002mx} which should be cited if the file {\tt T2K.glb} is 
used for a scientific publication or a talk. 

\subsection*{NO$\nu$A -- {\tt NOvA.glb}}

The \NOVA\ experiment can be simulated with the file {\tt NOvA.glb}. The description of the disapperance 
channels is taken from \cite{Yang_2004} and the description of the appearance channels follows 
the proposal \cite{Ambats:2004js}. These references
should be cited if the file {\tt NOvA.glb} is used for a scientific publication or a talk. For calculations that
involve {\tt NOvA.glb}, the following additional files are required: 
\bi
\item {\tt NOvAplus.dat} (NuMI neutrino flux -- $\nu_\mu$)
\item {\tt NOvAminus.dat} (NuMI neutrino flux -- $\bar{\nu}_\mu$)
\item {\tt XCC.dat} (charged current cross sections)
\item {\tt XNC.dat} (neutral current cross sections)
\ei
The \NOVA\ experiment uses a neutrino beam from the Fermilab NuMI beamline. The target power is 
1.12~MV which results in $10^{21}$~pot~$\mathrm{yr^{-1}}$ and 3~years $\nu$-running and 3~years $\bar{\nu}$-running 
is assumed. The fiducial mass of the Totally Liquid Scintillator Detector (TASD) is taken to be 
$\mathrm{m_{det} = 25 \,kt}$ at a baseline of L~=~812~km approximately 12~km off-axis to the beamline. The
energy resolution is $\mathrm{\sigma_e=10 \% \cdot \sqrt{E}}$ for electrons and  $\mathrm{\sigma_e=5 \% \cdot \sqrt{E}}$
for muons. The following rules are defined within {\tt NOvA.glb}: 
\begin{center}
\begin{tabular}{|l|ll|c|c|}
\hline \hline
\multicolumn{3}{|l|}{Disappearance (+)} & $\sigma_\mathrm{norm}$ & $\sigma_\mathrm{cal}$ \\ \hline
Signal & $0.8 \, \otimes \, (\nu_\mu\rightarrow\nu_\mu)_{\mathrm{CC}}$ & & 0.05 & 0.025 \\
 & & & & \\
Background & $0.0015 \, \otimes \, (\nu_\mu \rightarrow \nu_x)_\mathrm{NC}$ & & 0.05 & 0.025 \\ \hline \hline 
\multicolumn{3}{|l|}{Appearance (+)} & & \\ \hline
Signal & $0.24 \, \otimes \, (\nu_\mu \rightarrow \nu_e)_\mathrm{CC}$ & & 0.05 & 0.025\\
 & & & & \\
Background & $0.0015 \, \otimes \, (\nu_\mu \rightarrow \nu_x)_\mathrm{NC}$ & $1.0\cdot 10^{-4} \, \otimes \,
(\nu_\mu\rightarrow\nu_\mu)_{\mathrm{CC}}$ & 0.05 & 0.025 \\
Beam background & $0.12 \, \otimes \, (\nu_e\rightarrow \nu_e)_\mathrm{CC}$ & & 0.05 & 0.025 \\ \hline \hline
\multicolumn{3}{|l|}{Disappearance (--)} &  &  \\ \hline
Signal & $0.8 \, \otimes \, (\bar{\nu}_\mu\rightarrow\bar{\nu}_\mu)_{\mathrm{CC}}$ & & 0.05 & 0.025 \\
 & & & & \\
Background & $0.0015 \, \otimes \, (\bar{\nu}_\mu \rightarrow \bar{\nu}_x)_\mathrm{NC}$ & & 0.05 & 0.025 \\ \hline \hline 
\multicolumn{3}{|l|}{Appearance (--)} & & \\ \hline
Signal & $0.37 \, \otimes \, (\bar{\nu}_\mu \rightarrow \bar{\nu}_e)_\mathrm{CC}$ & & 0.05 & 0.025\\
 & & & & \\
Background & $0.0037 \, \otimes \, (\bar{\nu}_\mu \rightarrow \bar{\nu}_x)_\mathrm{NC}$ & $1.0\cdot 10^{-4} \, \otimes \,
(\bar{\nu}_\mu\rightarrow\bar{\nu}_\mu)_{\mathrm{CC}}$ & 0.05 & 0.025 \\
Beam background & $0.12 \, \otimes \, (\bar{\nu}_e\rightarrow \bar{\nu}_e)_\mathrm{CC}$ & & 0.05 & 0.025 \\ \hline \hline
\end{tabular}
\end{center}

\subsection*{SPL - {\tt SPL.glb}}

The \SPL\ experiment can be simulated with the file {\tt SPL.glb}. This file was used in \cite{Campagne:2006yx}
and follows the experiment description from \cite{Mezzetto:2003mm, Campagne:2004wt}. These references
should be cited if the file {\tt SPL.glb} is used for a scientific publication or a talk. For calculations that
involve {\tt SPL.glb}, the following additional files are required: 
\bi
\item {\tt SPLplus.dat} (neutrino flux from CERN -- $\nu_\mu$)
\item {\tt SPLminus.dat} (neutrino flux from CERN -- $\bar{\nu}_\mu$)
\item {\tt Mig\_WC\_numu.dat} (migration matrix -- $\nu_\mu$)
\item {\tt Mig\_WC\_numubar.dat} (migration matrix -- $\bar{\nu}_\mu$)
\item {\tt Mig\_WC\_nue.dat} (migration matrix -- $\nu_e$)
\item {\tt Mig\_WC\_nuebar.dat} (migration matrix -- $\bar{\nu}_e$)
\item {\tt XCC\_spl.dat} (charged current cross sections)
\item {\tt XNC\_spl.dat} (neutral current cross sections)
\ei
The \SPL\ experiment uses a neutrino beam from the CERN to Fr\'{e}jus. The target power is 
4~MV and 2~years $\nu$-running and 8~years $\bar{\nu}$-running 
is assumed. The fiducial mass of the Water Cerenkov detector is taken to be 
$\mathrm{m_{det} = 500 \,kt}$ at a baseline of L~=~130~km. The energy resolution is introduced manually by
four migration matrices (for $\nu_e , \bar{\nu}_e ,\nu_{\mu} ,\bar{\nu}_\mu$) that describe energy smearing mainly
from Fermi Motion. The following rules are defined within {\tt SPL.glb}: 
\begin{center}
\begin{tabular}{|l|ll|c|c|}
\hline \hline
\multicolumn{3}{|l|}{Disappearance (+)} & $\sigma_\mathrm{norm}$ & $\sigma_\mathrm{cal}$ \\ \hline 
Signal & $1.0 \, \otimes \, (\nu_\mu\rightarrow\nu_\mu)_{\mathrm{CC}}$ & (energy dep. efficiency) & 0.02 & $10^{-4}$\\
 & &  & &\\
Background & $4.3\cdot 10^{-5} \, \otimes \, (\nu_\mu \rightarrow \nu_\mu)_\mathrm{CC}$ & & 0.02 & $10^{-4}$ \\ \hline \hline 
\multicolumn{3}{|l|}{Appearance (+)} &  & \\ \hline
Signal & $0.707 \, \otimes \, (\nu_\mu\rightarrow \nu_e)_\mathrm{CC}$ & & 0.02 & $10^{-4}$\\
&&&&\\
Background & $6.5\cdot 10^{-4} \, \otimes \, (\nu_\mu \rightarrow \nu_x)_\mathrm{NC}$ & $5.4\cdot 10^{-4} \, \otimes \, (\nu_\mu \rightarrow \nu_\mu)_\mathrm{CC}$ & 0.02 & $10^{-4}$ \\
 & $0.7 \, \otimes \, (\bar{\nu}_\mu \rightarrow \bar{\nu}_e )_\mathrm{CC}$ & & 0.02 &  $10^{-4}$ \\
Beam Background & $0.677 \, \otimes \, (\bar{\nu}_e \rightarrow \bar{\nu}_e )_\mathrm{CC}$ & $0.707 \, \otimes \, (\nu_e \rightarrow \nu_e )_\mathrm{CC}$ & 0.02 & $10^{-4}$ \\ \hline \hline
\end{tabular}
\end{center}
\begin{center}
\begin{tabular}{|l|ll|c|c|}
\hline \hline
\multicolumn{3}{|l|}{Disappearance (--)} & & \\ \hline 
Signal & $1.0 \, \otimes \, (\bar{\nu}_\mu\rightarrow\bar{\nu}_\mu)_{\mathrm{CC}}$ & (energy dep. efficiency) & 0.02 & $10^{-4}$\\
 & &  & &\\
Background & $4.3\cdot 10^{-5} \, \otimes \, (\bar{\nu}_\mu \rightarrow \bar{\nu}_\mu)_\mathrm{CC}$ & & 0.02 & $10^{-4}$ \\ \hline \hline 
\multicolumn{3}{|l|}{Appearance (--)} &  & \\ \hline
Signal & $0.677 \, \otimes \, (\bar{\nu}_\mu\rightarrow \bar{\nu}_e)_\mathrm{CC}$ & & 0.02 & $10^{-4}$\\
&&&&\\
Background & $0.0025 \, \otimes \, (\bar{\nu}_\mu \rightarrow \bar{\nu}_x)_\mathrm{NC}$ & $5.4\cdot 10^{-4} \, \otimes \, (\bar{\nu}_\mu \rightarrow \bar{\nu}_\mu)_\mathrm{CC}$ & 0.02 & $10^{-4}$ \\
 & $0.7 \, \otimes \, (\nu_\mu \rightarrow \nu_e )_\mathrm{CC}$ & & 0.02 &  $10^{-4}$ \\
Beam Background & $0.677 \, \otimes \, (\bar{\nu}_e \rightarrow \bar{\nu}_e )_\mathrm{CC}$ & $0.707 \, \otimes \, (\nu_e \rightarrow \nu_e )_\mathrm{CC}$ & 0.02 & $10^{-4}$ \\ \hline \hline
\end{tabular}
\end{center}

\section{Reactor experiments}
\subsection*{Small reactor experiment -- {\tt Reactor1.glb}}

The file {\tt Reactor1.glb} allows to simulate a small $\bar{\nu}_e$-disappearance reactor experiment. The basic
version of this file was used within \cite{Huber:2003pm} which should be cited if the file {\tt Reactor1.glb} is
used for a scientific publication or a talk. For calculations that involve {\tt Reactor1.glb}, the following
additional files are required:
\bi
\item {\tt Reactor.dat} (neutrino flux from reactor)
\item {\tt XCCreactor.dat} (charged current cross sections for low energies)
\ei
The neutrino source is the core of a nuclear power reactor. The integrated luminosity is assumed to be
$\mathrm{\mathcal{L} = 400 \,t \, GW\, yr}$, \eg\ a 20~t detector, a reactor with a thermal power of 4~GW
and a running period of 5 years. As detector technology, a liquid scintillator 
detector is assumed, a far detector at a baseline of L~=~1.7~km and a near detector which is
assumed to be identical to the far detector (maybe apart from the size) in order to minimize the impact of systematical uncertainties. The normalization error
used in the file {\tt Reactor1.glb} has to be considered as an effective error, receiving contributions from
individual uncertainties (see~\cite{Huber:2003pm}). The energy resolution is
$\mathrm{\sigma_e=5\%\cdot\sqrt{E_{vis}}}$ and the choice for {\tt sigma\_function} is {\tt \#inverse\_beta}. The following rules are defined within {\tt Reactor1.glb}: 
\begin{center}
\begin{tabular}{|l|ll|c|c|}
\hline \hline
\multicolumn{3}{|l|}{Disappearance} & $\sigma_\mathrm{norm}$ & $\sigma_\mathrm{cal}$ \\ \hline 
Signal & $1.0 \, \otimes \, (\bar{\nu}_e\rightarrow\bar{\nu}_e)_{\mathrm{CC}}$ & \hspace{5.5cm} & 0.008 & 0.005\\
 & &  & &\\
Background & $5.8\cdot 10^{-5} \, \otimes \, (\bar{\nu}_e\rightarrow\bar{\nu}_e)_\mathrm{CC}$ & \hspace{5.5cm} &
$10^{-6}$ & $10^{-6}$ \\ \hline \hline 
\end{tabular}
\end{center}

\subsection*{Large reactor experiment -- {\tt Reactor2.glb}}

The file {\tt Reactor2.glb} allows to simulate a large $\bar{\nu}_e$-disappearance reactor experiment. The basic
version of this file was used within \cite{Huber:2003pm} which should be cited if the file {\tt Reactor2.glb} is
used for a scientific publication or a talk. The integrated luminosity is assumed to be
$\mathrm{\mathcal{L} = 8000 \,t \, GW\, yr}$, \eg\ a 100~t detector, a reactor with a thermal power of 10~GW
and a running period of 8 years. Besides the higher integrated luminosity, the attributes of {\tt Reactor2.glb}
are similar to the ones of {\tt Reactor1.glb}.

\subsection*{DoubleCHOOZ -- {\tt D-Chooz\_near.glb} and {\tt D-Chooz\_far.glb}}

The files {\tt D-Chooz\_near.glb} and {\tt D-Chooz\_far.glb} allow to simulate the \DC\ reactor experiment in France. 
The basic versions of these files were used within \cite{Huber:2006vr}, which should be cited if the files 
{\tt D-Chooz\_near.glb} and {\tt D-Chooz\_far.glb} are
used for a scientific publication or a talk. For calculations that involve {\tt D-Chooz\_near.glb} and/or {\tt D-Chooz\_far.glb}, the following
additional files are required:
\bi
\item {\tt Reactor.dat} (neutrino flux from reactor)
\item {\tt XCCreactor.dat} (charged current cross sections for low energies)
\ei
The \DC\ experiment is located at the Chooz reactor complex, and the two reactor cores serve as $\bar{\nu}_e$
neutrino source, so the thermal power is $\mathrm{2\cdot4.2\,GW}$. Two identical liquid scintillator detectors
with a fiducial mass of $\mathrm{m_{det} = 10.16 \,t}$ are used as near and far detector. The far detector
is planned to be located in the {\sc Chooz} cavern at a baseline of L~=~1.05~km from the two reactor cores and
the near detector is assumed to be located at a distance of 0.1~km to the cores. The total running time of the
experiment is assumed to be 5 years. So, the integrated luminosity at the far detector yields $\mathrm{\mathcal{L} \approx
427 \,t \, GW\, yr}$.
Here, the total running time of 5 years is assumed within {\tt D-Chooz\_near.glb} and {\tt D-Chooz\_far.glb}, so near
and far detector are assumed to start the mode of operation simultaneously. For the simulation of \DC\ and
considering a delayed start of data taking at the near detector the file {\tt D-Chooz\_near.glb} has to be
modified. The cancellation of systematical uncertainties is considered by the manual definition of a $\chi^2$
as described in \cite{Huber:2006vr} with the treatment of user-defined systematics as described in \Sec~\ref{sec:userchi}   
The energy resolution is
$\mathrm{\sigma_e=5\%\cdot\sqrt{E_{vis}}}$ and the choice for {\tt sigma\_function} is {\tt \#inverse\_beta}. The following rules are defined within {\tt D-Chooz\_near.glb} and {\tt D-Chooz\_far.glb}: 
\begin{center}
\begin{tabular}{|l|ll|c|c|c|c|c|}
\hline \hline
\multicolumn{3}{|l|}{Disappearance} & $\sigma_\mathrm{flux}$ & $\sigma^N_\mathrm{fid}$ & $\sigma^F_\mathrm{fid}$ &
$\sigma^N_\mathrm{cal}$ & $\sigma^F_\mathrm{cal}$ \\ \hline 
Signal & $1.0 \, \otimes \, (\bar{\nu}_e\rightarrow\bar{\nu}_e)_{\mathrm{CC}}$ & & 0.02 & 0.006 & 0.006 & 0.005 &
0.005 \\
 & & & & & & & \\
Background & neglected & (sys. dominates) & -- & -- & -- & -- & -- \\ \hline \hline 
\end{tabular}
\end{center}

\section{Beta beam experiments}
\subsection*{CERN-Fr\'{e}jus baseline scenario -- {\tt BB\_100.glb}}

The $\gamma=100$ $\beta$-beam baseline scenario from CERN to Fr\'{e}jus can be simulated with the file {\tt
BB\_100.glb}. The basic version of this file was used within \cite{Campagne:2006yx}. This reference
should be cited if the file {\tt BB\_100.glb} is used for a scientific publication or a talk. For calculations that
involve {\tt BB\_100.glb}, the following additional files are required: 
\bi
\item {\tt BB100flux\_Ne.dat} ($\beta$-beam neutrino flux -- $^{18}$Ne stored at $\gamma=100$)
\item {\tt BB100flux\_He.dat} ($\beta$-beam neutrino flux -- $^{6}$He stored at $\gamma=100$)
\item {\tt BeamBckg\_100.dat} (beam background)
\item {\tt AtmBckg\_100.dat} (atmospheric background)
\item {\tt Mig\_WC\_numu.dat} (migration matrix -- $\nu_\mu$)
\item {\tt Mig\_WC\_numubar.dat} (migration matrix -- $\bar{\nu}_\mu$)
\item {\tt Mig\_WC\_nue.dat} (migration matrix -- $\nu_e$)
\item {\tt Mig\_WC\_nuebar.dat} (migration matrix -- $\bar{\nu}_e$)
\item {\tt XCC\_Nuance.dat} (charged current cross sections)
\item {\tt XNC\_Nuance.dat} (neutral current cross sections)
\item {\tt Null.dat} (auxiliary file)
\ei
The neutrino beam is produced at CERN and directed towards a a megaton Water Cerenkov detector at Fr\'{e}jus. 
The neutrinos originate from the decays of accelerated isotopes $^{18}$Ne ($\nu_e$) and $^6$He ($\bar{\nu}_e$).
The acceleration factor is $\gamma=100$ for both types of isotopes and $2.2\cdot10^{18}$ $^{18}$Ne decays per
year and $5.8\cdot10^{18}$ $^{6}$He decays per year are assumed. The CERN-Fr\'{e}jus baseline is L~=~130~km, the
fiducial mass of the detector is $\mathrm{m_{det} = 500 \,kt}$ and 4~years $\nu$-running and 4~years
$\bar{\nu}$-running are assumed. The energy resolution is introduced manually by
four migration matrices (for $\nu_e , \bar{\nu}_e ,\nu_{\mu} ,\bar{\nu}_\mu$) that describes energy smearing mainly
from Fermi Motion. The following rules are defined within {\tt BB\_100.glb}:
\begin{center}
\begin{tabular}{|l|ll|c|c|}
\hline \hline
\multicolumn{3}{|l|}{Disappearance -- $^{18}$Ne stored} & $\sigma_\mathrm{norm}$ & $\sigma_\mathrm{cal}$ \\ \hline
Signal & $0.707 \, \otimes \, (\nu_e\rightarrow\nu_e)_{\mathrm{CC}}$ & & 0.02 & $10^{-4}$ \\
 & & & & \\
Background & $4.3\cdot 10^{-5} \, \otimes \, (\nu_e \rightarrow \nu_e)_\mathrm{CC}$ & & 0.02 & $10^{-4}$ \\ \hline \hline 
\multicolumn{3}{|l|}{Appearance -- $^{18}$Ne stored} & & \\ \hline
Signal & $1.0 \, \otimes \, (\nu_e \rightarrow \nu_\mu)_\mathrm{CC}$ & (energy dep. efficiency) & 0.02 & $10^{-4}$ \\
 & & & & \\
Background & $1.0 \, \otimes \, (\nu_e \rightarrow \nu_x)_\mathrm{NC}$ & (from external file) & 0.02 & $10^{-4}$ \\
Atm. background & & (from external file) & 0.02 & $10^{-4}$ \\ \hline \hline
\end{tabular}
\end{center}
\begin{center}
\begin{tabular}{|l|ll|c|c|}
\hline \hline
\multicolumn{3}{|l|}{Disappearance -- $^6$He stored} & &  \\ \hline
Signal & $0.677 \, \otimes \, (\bar{\nu}_e\rightarrow\bar{\nu}_e)_{\mathrm{CC}}$ & & 0.02 & $10^{-4}$ \\
 & & & & \\
Background & $4.3\cdot 10^{-5} \, \otimes \, (\bar{\nu}_e \rightarrow \nu_e)_\mathrm{CC}$ & & 0.02 & $10^{-4}$ \\ \hline \hline 
\multicolumn{3}{|l|}{Appearance -- $^6$He stored} & & \\ \hline
Signal & $1.0 \, \otimes \, (\bar{\nu}_e \rightarrow \bar{\nu}_\mu)_\mathrm{CC}$ & (energy dep. efficiency) & 0.02 & $10^{-4}$ \\
 & & & & \\
Background & $1.0 \, \otimes \, (\bar{\nu}_e \rightarrow \bar{\nu}_x)_\mathrm{NC}$ & (from external file) & 0.02 & $10^{-4}$ \\
Atm. background & & (from external file) & 0.02 & $10^{-4}$ \\ \hline \hline
\end{tabular}
\end{center}

\subsection*{Higher gamma scenario -- {\tt BB\_350.glb}}

A $\gamma=350$ medium gamma $\beta$-beam scenario involving a megaton Water Cerenkov detector can be simulated with the file {\tt
BB\_350.glb}. This file tries to approximate as closely as possible the scenario {\sf Setup III} from
\cite{Burguet-Castell:2005pa}. This reference
should be cited if the file {\tt BB\_350.glb} is used for a scientific publication or a talk. For calculations that
involve {\tt BB\_350.glb}, the following additional files are required: 
\bi
\item {\tt BB350flux.dat} ($\beta$-beam neutrino flux -- $\gamma=350$ )
\item {\tt NeEffMig350.dat} (migration matrix -- $\nu_\mu$)
\item {\tt NeBckgRej350.dat} (migration matrix -- background)
\item {\tt NeDisEff350.dat} (migration matrix -- $\nu_e$)
\item {\tt HeEffMig350.dat} (migration matrix -- $\bar{\nu}_\mu$)
\item {\tt HeBckgRej350.dat} (migration matrix -- background)
\item {\tt HeDisEff350.dat} (migration matrix -- $\bar{\nu}_e$)
\item {\tt XCC.dat} (charged current cross sections)
\item {\tt XNC.dat} (neutral current cross sections)
\ei
The neutrinos originate from the decays of accelerated isotopes $^{18}$Ne ($\nu_e$) and $^6$He ($\bar{\nu}_e$).
The acceleration factor is $\gamma=350$ for both types of isotopes and $2.2\cdot10^{18}$ $^{18}$Ne decays per
year and $5.8\cdot10^{18}$ $^{6}$He decays per year are assumed. The ion acceleration would require either a refurbished SPS (with
superconducting magnets) or a more powerful accelerator, such as the
Tevatron or LHC. The baseline is L~=~730~km, the
fiducial mass of the detector is $\mathrm{m_{det} = 500 \,kt}$ and 4~years $\nu$-running and 4~years
$\bar{\nu}$-running are assumed. The energy resolution is introduced manually by
six migration matrices (for $\nu_e , \bar{\nu}_e ,\nu_{\mu} ,\bar{\nu}_\mu$, and the background from NC events
for $^{18}$Ne and $^6$He) that describes energy smearing. These migration matrices also already include energy
dependend efficiencies and background rejection factors. They are taken from the appendix of
\cite{Burguet-Castell:2005pa}. The following rules are defined within {\tt BB\_350.glb}:
\begin{center}
\begin{tabular}{|l|ll|c|c|}
\hline \hline
\multicolumn{3}{|l|}{Disappearance -- $^{18}$Ne stored} & $\sigma_\mathrm{norm}$ & $\sigma_\mathrm{cal}$ \\ \hline
Signal & $1.0 \, \otimes \, (\nu_e\rightarrow\nu_e)_{\mathrm{CC}}$ & (migration matrix) & 0.025 & $10^{-4}$ \\
 & & & & \\
Background & neglected & (systematic uncertainty dominates) & -- & -- \\ \hline \hline 
\multicolumn{3}{|l|}{Appearance -- $^{18}$Ne stored} & & \\ \hline
Signal & $1.0 \, \otimes \, (\nu_e \rightarrow \nu_\mu)_\mathrm{CC}$ & (migration matrix) & 0.025 & $10^{-4}$ \\
 & & & & \\
Background & $1.0 \, \otimes \, (\nu_e \rightarrow \nu_x)_\mathrm{NC}$ & (migration matrix) & 0.05 & $10^{-4}$
\\ \hline \hline
\multicolumn{3}{|l|}{Disappearance -- $^6$He stored} & &  \\ \hline
Signal & $1.0 \, \otimes \, (\bar{\nu}_e\rightarrow\bar{\nu}_e)_{\mathrm{CC}}$ & (migration matrix) & 0.025 & $10^{-4}$ \\
 & & & & \\
Background & neglected & (systematic uncertainty dominates) & -- & -- \\ \hline \hline 
\multicolumn{3}{|l|}{Appearance -- $^6$He stored} & & \\ \hline
Signal & $1.0 \, \otimes \, (\bar{\nu}_e \rightarrow \bar{\nu}_\mu)_\mathrm{CC}$ & (migration matrix)  &
0.025 & $10^{-4}$ \\
 & & & & \\
Background & $1.0 \, \otimes \, (\bar{\nu}_e \rightarrow \bar{\nu}_x)_\mathrm{NC}$ & (migration matrix) & 0.05 & $10^{-4}$ \\ \hline \hline
\end{tabular}
\end{center}

\subsection*{Variable beta beam (Water Cerenkov) -- {\tt BBvar\_WC.glb}}

A variable $\beta$-beam scenario involving a megaton Water Cerenkov detector can be simulated with the file 
{\tt BBvar\_WC.glb}. The basic file was used within \cite{Huber:2005jk}. This reference
should be cited if the file {\tt BBvar\_WC.glb} is used for a scientific publication or a talk. For calculations that
involve {\tt BBvar\_WC.glb}, the following additional files are required:
\bi
\item {\tt BckgMig\_var.dat} (migration matrix -- background)
\item {\tt XCC.dat} (charged current cross sections)
\item {\tt XNC.dat} (neutral current cross sections)
\item {\tt XQE.dat} (quasi elastic cross sections)
\ei
and the values of the following {\sf AEDL}-Variables have to be set:
\bi
\item {\tt gammafactor} (acceleration factor $\gamma$)
\item {\tt EXP\_FACTOR} (parameter of ion decay scaling)
\item {\tt baselinefactor} (baseline parameter L/$\gamma\,\left[\mathrm{km}\right]$)
\ei
The neutrinos originate from the decays of accelerated isotopes $^{18}$Ne ($\nu_e$) and $^6$He ($\bar{\nu}_e$).
The acceleration factor is $\gamma=${\tt gammafactor} for both types of isotopes and
$(100/\gamma)^\alpha\cdot2.2\cdot10^{18}$ $^{18}$Ne decays per
year and $(60/\gamma)^\alpha\cdot5.8\cdot10^{18}$ $^{6}$He decays per year are assumed where $\alpha$~=~{\tt
EXP\_FACTOR} is a parameter that describes ion decay scaling. As default value {\tt EXP\_FACTOR=0} should be chosen.
See \cite{Huber:2005jk} for a detailed discussion of this parameter. The baseline is
L~=~{\tt baselinefactor}$\cdot\gamma$~km, the
fiducial mass of the detector is $\mathrm{m_{det} = 500 \,kt}$ and 4~years $\nu$-running and 4~years
$\bar{\nu}$-running are assumed. The appearance measurement involves the total rates data from all CC events and the spectral data
from the QE sample with a free normalization at an energy resolution of $\mathrm{\sigma_e=0.085GeV}$ due to the
Fermi Motion identical to the treatment of systematics within the \TtoK\ and \TtoHK\ files. Note, that here also the systematics
function for the appearance sample is {\tt chiSpectrumTilt} in case of systematics on {\em and} off to avoid double counting of the
QE events for systematics switched off. The following rules are defined within {\tt BBvar\_WC.glb}:
\begin{center}
\begin{tabular}{|l|ll|c|c|}
\hline \hline
\multicolumn{3}{|l|}{Disappearance -- $^{18}$Ne stored} & $\sigma_\mathrm{norm}$ & $\sigma_\mathrm{cal}$ \\ \hline
Signal & $0.55 \, \otimes \, (\nu_e\rightarrow\nu_e)_{\mathrm{QE}}$ & \hspace{6cm} & 0.025 & $10^{-4}$ \\
 & & & & \\
Background & $0.003 \, \otimes \, (\nu_e \rightarrow \nu_x)_\mathrm{NC}$ & & 0.05 & $10^{-4}$ \\ \hline \hline 
\multicolumn{3}{|l|}{Appearance -- $^{18}$Ne stored -- Spectrum} & & \\ \hline
Signal & $0.55 \, \otimes \, (\nu_e \rightarrow \nu_\mu)_\mathrm{QE}$ & & 10.0 & $10^{-4}$ \\
 & & & & \\
Background & $0.003 \, \otimes \, (\nu_e \rightarrow \nu_x)_\mathrm{NC}$ & & 0.05 & $10^{-4}$
\\ \hline \hline
\multicolumn{3}{|l|}{Appearance -- $^{18}$Ne stored -- Total Rates} & & \\ \hline
Signal & $0.55 \, \otimes \, (\nu_e \rightarrow \nu_\mu)_\mathrm{CC}$ & & 0.025 & $10^{-4}$ \\
 & & & & \\
Background & $0.003 \, \otimes \, (\nu_e \rightarrow \nu_x)_\mathrm{NC}$ & & 0.05 & $10^{-4}$
\\ \hline \hline\multicolumn{3}{|l|}{Disappearance -- $^6$He stored} & &  \\ \hline
Signal & $0.75 \, \otimes \, (\bar{\nu}_e\rightarrow\bar{\nu}_e)_{\mathrm{QE}}$ & & 0.025 & $10^{-4}$ \\
 & & & & \\
Background & $0.0025 \, \otimes \, (\bar{\nu}_e \rightarrow \bar{\nu}_x)_\mathrm{NC}$ & & 0.05 & $10^{-4}$ \\ \hline \hline 
\multicolumn{3}{|l|}{Appearance -- $^6$He stored -- Spectrum} & & \\ \hline
Signal & $0.75 \, \otimes \, (\bar{\nu}_e \rightarrow \bar{\nu}_\mu)_\mathrm{QE}$ & &
10.0 & $10^{-4}$ \\
 & & & & \\
Background & $0.0025 \, \otimes \, (\bar{\nu}_e \rightarrow \bar{\nu}_x)_\mathrm{NC}$ & & 0.05 & $10^{-4}$ \\ \hline \hline
\multicolumn{3}{|l|}{Appearance -- $^6$He stored -- Total Rates} & & \\ \hline
Signal & $0.75 \, \otimes \, (\bar{\nu}_e \rightarrow \bar{\nu}_\mu)_\mathrm{CC}$ & &
0.025 & $10^{-4}$ \\
 & & & & \\
Background & $0.0025 \, \otimes \, (\bar{\nu}_e \rightarrow \bar{\nu}_x)_\mathrm{NC}$ & & 0.05 & $10^{-4}$ \\ \hline \hline
\end{tabular}
\end{center}

\subsection*{Variable beta beam (TASD) -- {\tt BBvar\_TASD.glb}}

A variable $\beta$-beam scenario involving a \NOVA-like TASD detector can be simulated with the file 
{\tt BBvar\_TASD.glb}. The basic file was used within \cite{Huber:2005jk}. This reference
should be cited if the file {\tt BBvar\_TASD.glb} is used for a scientific publication or a talk. For calculations that
involve {\tt BBvar\_TASD.glb}, the following additional files are required:
\bi
\item {\tt XCC.dat} (charged current cross sections)
\item {\tt XNC.dat} (neutral current cross sections)
\ei
and the values of the following {\sf AEDL}-Variables have to be set:
\bi
\item {\tt gammafactor} (acceleration factor $\gamma$)
\item {\tt EXP\_FACTOR} (parameter of ion decay scaling)
\item {\tt baselinefactor} (baseline parameter L/$\gamma\,\left[\mathrm{km}\right]$)
\ei
The neutrinos originate from the decays of accelerated isotopes $^{18}$Ne ($\nu_e$) and $^6$He ($\bar{\nu}_e$).
The acceleration factor is $\gamma=${\tt gammafactor} for both types of isotopes and
$(100/\gamma)^\alpha\cdot2.2\cdot10^{18}$ $^{18}$Ne decays per
year and $(60/\gamma)^\alpha\cdot5.8\cdot10^{18}$ $^{6}$He decays per year are assumed where $\alpha$~=~{\tt
EXP\_FACTOR} is a parameter that describes ion decay scaling. As default value {\tt EXP\_FACTOR=0} should be chosen.
See \cite{Huber:2005jk} for a detailed discussion of this parameter. The baseline is L~=~{\tt baselinefactor}$\cdot\gamma$~km, the
fiducial mass of the detector is $\mathrm{m_{det} = 50 \,kt}$ and 4~years $\nu$-running and 4~years
$\bar{\nu}$-running are assumed. The energy resolution is $\mathrm{\sigma_e=6 \% \cdot \sqrt{E}}$ for electrons and  $\mathrm{\sigma_e=3 \% \cdot \sqrt{E}}$
for muons.  The following rules are defined within {\tt BBvar\_TASD.glb}:
\begin{center}
\begin{tabular}{|l|ll|c|c|}
\hline \hline
\multicolumn{3}{|l|}{Disappearance -- $^{18}$Ne stored} & $\sigma_\mathrm{norm}$ & $\sigma_\mathrm{cal}$ \\ \hline
Signal & $0.2 \, \otimes \, (\nu_e\rightarrow\nu_e)_{\mathrm{CC}}$ & \hspace{6cm} & 0.025 & $10^{-4}$ \\
 & & & & \\
Background & $0.001 \, \otimes \, (\nu_e \rightarrow \nu_x)_\mathrm{NC}$ & & 0.05 & $10^{-4}$ \\ \hline \hline 
\multicolumn{3}{|l|}{Appearance -- $^{18}$Ne stored} & & \\ \hline
Signal & $0.8 \, \otimes \, (\nu_e \rightarrow \nu_\mu)_\mathrm{CC}$ & & 0.025 & $10^{-4}$ \\
 & & & & \\
Background & $0.001 \, \otimes \, (\nu_e \rightarrow \nu_x)_\mathrm{NC}$ & & 0.05 & $10^{-4}$
\\ \hline \hline
\multicolumn{3}{|l|}{Disappearance -- $^6$He stored} & &  \\ \hline
Signal & $0.2 \, \otimes \, (\bar{\nu}_e\rightarrow\bar{\nu}_e)_{\mathrm{CC}}$ & & 0.025 & $10^{-4}$ \\
 & & & & \\
Background & $0.001 \, \otimes \, (\bar{\nu}_e \rightarrow \bar{\nu}_x)_\mathrm{NC}$ & & 0.05 & $10^{-4}$  \\ \hline \hline 
\multicolumn{3}{|l|}{Appearance -- $^6$He stored} & & \\ \hline
Signal & $0.8 \, \otimes \, (\bar{\nu}_e \rightarrow \bar{\nu}_\mu)_\mathrm{CC}$ & &
0.025 & $10^{-4}$ \\
 & & & & \\
Background & $0.001 \, \otimes \, (\bar{\nu}_e \rightarrow \bar{\nu}_x)_\mathrm{NC}$ & & 0.05 & $10^{-4}$ \\ \hline \hline
\end{tabular}
\end{center}

\section{Neutrino factory experiments}
\subsection*{Standard neutrino factory -- {\tt NFstandard.glb}}

A standard neutrino factory scenario can be simulated with the file {\tt NFstandard.glb}. The basic version was
used within \cite{Huber:2002mx} (NuFact-II scenario). This reference should be cited if the file {\tt
NFstandard.glb} is used for a scientific publication or a talk. For calculations that involve {\tt
NFstandard.glb}, the following additional files are required:
\bi
\item {\tt XCC.dat} (charged current cross sections)
\item {\tt XNC.dat} (neutral current cross sections)
\ei
The neutrino beam is produced by the decay of muons stored in a storage ring at a parent energy of
$\mathrm{E_\mu = 50\,GeV}$. A number of $1.06\cdot10^{21}$ useful muon decays per year are assumed in each
polarity (corresponding to $5.3\cdot10^{20}$ useful muon decays per year and polarity for simultaneous operation
with both polarities) and 4~years $\nu$-running and 4~years $\bar{\nu}$-running is
assumed. The fiducial mass of the MID detector is taken to be $\mathrm{m_{det} = 50 \,kt}$ at a baseline of
L~=~3000~km. The energy resolution is $\mathrm{\sigma_e=15\%\cdot E}$. The following rules are defined within
{\tt NFstandard.glb}:
\begin{center}
\begin{tabular}{|l|ll|c|c|}
\hline \hline
\multicolumn{3}{|l|}{Disappearance -- $\mu^+$-stored} & $\sigma_\mathrm{norm}$ & $\sigma_\mathrm{cal}$ \\ \hline
Signal & $0.35 \, \otimes \, (\bar{\nu}_\mu \rightarrow \bar{\nu}_\mu)_\mathrm{CC}$ & & 0.025 & $10^{-4}$ \\
 & & & & \\
Background & $1.0\cdot 10^{-5} \, \otimes \, (\bar{\nu}_\mu \rightarrow \bar{\nu}_x)_\mathrm{NC}$ & & 0.2 & $10^{-4}$ \\ \hline \hline
\multicolumn{3}{|l|}{Appearance -- $\mu^+$-stored} & & \\ \hline
Signal &  $0.45 \, \otimes \, (\nu_e \rightarrow \nu_\mu)_\mathrm{CC}$ & & 0.025 & $10^{-4}$ \\
 & & & & \\
Background &  $5.0\cdot 10^{-6} \, \otimes \, (\bar{\nu}_\mu \rightarrow \bar{\nu}_x)_\mathrm{NC}$ &  $5.0\cdot
10^{-6} \, \otimes \, (\bar{\nu}_\mu \rightarrow\bar{\nu}_\mu)_\mathrm{CC}$ & 0.2& $10^{-4}$\\ \hline \hline
\multicolumn{3}{|l|}{Disappearance -- $\mu^-$-stored} & & \\ \hline
Signal &  $0.45 \, \otimes \, (\nu_\mu \rightarrow \nu_\mu)_\mathrm{CC}$ & & 0.025& $10^{-4}$\\
 & & & & \\
Background &  $1.0\cdot 10^{-5} \, \otimes \, (\nu_\mu \rightarrow \nu_x)_\mathrm{NC}$ & & 0.2& $10^{-4}$\\ \hline \hline
\multicolumn{3}{|l|}{Appearance -- $\mu^-$-stored} & & \\ \hline
Signal & $0.35 \, \otimes \, (\bar{\nu}_e \rightarrow \bar{\nu}_\mu)_\mathrm{CC}$  & & 0.025& $10^{-4}$\\
 & & & & \\
Background &  $5.0\cdot 10^{-6} \, \otimes \, (\nu_\mu \rightarrow \nu_x)_\mathrm{NC}$ & $5.0\cdot 10^{-6} \, \otimes \, (\nu_\mu \rightarrow
\nu_\mu)_\mathrm{CC}$  & 0.2& $10^{-4}$\\ \hline \hline
\end{tabular}
\end{center}

\subsection*{Variable neutrino factory -- {\tt NFvar.glb}}

A variable neutrino factory scenario can be simulated with the file {\tt NFvar.glb}. The basic version was
used within \cite{Huber:2006wb} and follows the neutrino factory scenarios from \cite{Huber:2002mx}. These
references should be cited if the file {\tt
NFvar.glb} is used for a scientific publication or a talk. For calculations that involve {\tt
NFvar.glb}, the following additional files are required:
\bi
\item {\tt XCC.dat} (charged current cross sections)
\item {\tt XNC.dat} (neutral current cross sections)
\ei
and the values of the following {\sf AEDL}-Variables have to be set:
\bi
\item {\tt emax} (parent energy of the stored muons $\left[\mathrm{km}\right]$)
\item {\tt BASELINE} (experiment baseline $\left[\mathrm{GeV}\right]$)
\ei
The neutrino beam is produced by the decay of muons stored in a storage ring at a parent energy of
$\mathrm{E_\mu =}$~{\tt emax}. The parent energy of the muons can be set in the range
$\mathrm{10\,GeV\lesssim E_\mu\lesssim 80\, GeV}$. The baseline of the scenario is L~=~{\tt BASELINE}. 
Besides these settings the other attributes of {\tt NFvar.glb} are similar to the ones from {\tt
NFstandard.glb}. Only the treatment of the disappearance channels is different. Whereas the CID threshold is
also applied in the disappearance channels of {\tt NFstandard.glb}, this is not the case for the disapperance
channel definition within {\tt NFvar.glb}, where no CID is applied, and a threshold similar to the threshold of
the MINOS experiment \cite{Ables:1995wq} is applied. The following rules are defined within
{\tt NFvar.glb}:
\begin{center}
\begin{tabular}{|l|ll|c|c|}
\hline \hline
\multicolumn{3}{|l|}{Disappearance -- $\mu^+$-stored} & $\sigma_\mathrm{norm}$ & $\sigma_\mathrm{cal}$ \\ \hline
Signal & $0.9 \, \otimes \, (\bar{\nu}_\mu \rightarrow \bar{\nu}_\mu)_\mathrm{CC}$ & $0.9 \, \otimes \, (\nu_e \rightarrow \nu_\mu)_\mathrm{CC}$& 0.025 & $10^{-4}$ \\
 & & & & \\
Background & $1.0\cdot 10^{-5} \, \otimes \, (\bar{\nu}_\mu \rightarrow \bar{\nu}_x)_\mathrm{NC}$ & & 0.2 & $10^{-4}$ \\ \hline \hline
\multicolumn{3}{|l|}{Appearance -- $\mu^+$-stored} & & \\ \hline
Signal &  $0.45 \, \otimes \, (\nu_e \rightarrow \nu_\mu)_\mathrm{CC}$ & & 0.025 & $10^{-4}$ \\
 & & & & \\
Background &  $5.0\cdot 10^{-6} \, \otimes \, (\bar{\nu}_\mu \rightarrow \bar{\nu}_x)_\mathrm{NC}$ &  $5.0\cdot
10^{-6} \, \otimes \, (\bar{\nu}_\mu \rightarrow\bar{\nu}_\mu)_\mathrm{CC}$ & 0.2& $10^{-4}$\\ \hline \hline
\multicolumn{3}{|l|}{Disappearance -- $\mu^-$-stored} & & \\ \hline
Signal &  $0.9 \, \otimes \, (\nu_\mu \rightarrow \nu_\mu)_\mathrm{CC}$ & $0.9 \, \otimes \, (\bar{\nu}_e
\rightarrow \bar{\nu}_\mu)_\mathrm{CC}$& 0.025& $10^{-4}$\\
 & & & & \\
Background &  $1.0\cdot 10^{-5} \, \otimes \, (\nu_\mu \rightarrow \nu_x)_\mathrm{NC}$ & & 0.2& $10^{-4}$\\ \hline \hline
\multicolumn{3}{|l|}{Appearance -- $\mu^-$-stored} & & \\ \hline
Signal & $0.35 \, \otimes \, (\bar{\nu}_e \rightarrow \bar{\nu}_\mu)_\mathrm{CC}$  & & 0.025& $10^{-4}$\\
 & & & & \\
Background &  $5.0\cdot 10^{-6} \, \otimes \, (\nu_\mu \rightarrow \nu_x)_\mathrm{NC}$ & $5.0\cdot 10^{-6} \, \otimes \, (\nu_\mu \rightarrow
\nu_\mu)_\mathrm{CC}$  & 0.2& $10^{-4}$\\ \hline \hline
\end{tabular}
\end{center}

\subsection*{Variable neutrino factory with Silver Channel -- \\{\tt NF\_GoldSilver.glb}}

A variable neutrino factory scenario that includes the golden and silver appearance channels can be 
simulated with the file {\tt NF\_GoldSilver.glb}. The basic version was
used within \cite{Huber:2006wb} and the golden channel follows the neutrino factory scenarios from \cite{Huber:2002mx}
and the description of the silver channel follows \cite{Autiero:2003fu}. These
references should be cited if the file {\tt
NFvar\_GoldSilver.glb} is used for a scientific publication or a talk. For calculations that involve {\tt
NFvar\_GoldSilver.glb}, the following additional files are required:
\bi
\item {\tt XCC.dat} (charged current cross sections)
\item {\tt XNC.dat} (neutral current cross sections)
\ei
and the values of the following {\sf AEDL}-Variables have to be set:
\bi
\item {\tt emax} (parent energy of the stored muons $\left[\mathrm{km}\right]$)
\item {\tt BASELINE} (experiment baseline $\left[\mathrm{GeV}\right]$)
\ei
The beam and golden channel attributes are similar to {\tt NFvar.glb}. The parent energy of the muons can be set
in the range $\mathrm{10\,GeV\lesssim E_\mu\lesssim 80\, GeV}$. The baseline is set by the {\sf
AEDL}-Variable {\tt BASELINE}. For the silver channel, an additional ECC detector with a fiducial mass
$\mathrm{m_{ECC} = 5\,kt}$ is assumed to be located at the same baseline as the MID detector. The energy
resolution of the silver channel is set to $\mathrm{\sigma_e=20\%\cdot E}$. The following additional rule
compared to {\tt NFvar.glb} is introduced in {\tt NF\_GoldSilver.glb}:
\begin{center}
\begin{tabular}{|l|ll|c|c|}
\hline \hline
\multicolumn{3}{|l|}{$\tau$-Appearance -- $\mu^+$-stored} & $\sigma_\mathrm{norm}$ & $\sigma_\mathrm{cal}$ \\ \hline
Signal & $0.096 \, \otimes \, (\nu_e \rightarrow \nu_\tau)_\mathrm{CC}$ & & 0.15 & $10^{-4}$ \\
 & & & & \\
Background & $3.1\cdot 10^{-8} \, \otimes \, (\nu_e \rightarrow \nu_e)_\mathrm{CC}$ & $2.0\cdot 10^{-8} \,
\otimes \, (\nu_e \rightarrow \nu_\mu)_\mathrm{CC}$& 0.2 & $10^{-4}$ \\ 
 & $3.7\cdot 10^{-6} \, \otimes \, (\bar{\nu}_\mu \rightarrow \bar{\nu}_\mu)_\mathrm{CC}$ & $1.0\cdot 10^{-3} \,
 \otimes \, (\bar{\nu}_\mu \rightarrow \bar{\nu}_\tau)_\mathrm{CC}$& 0.2 & $10^{-4}$ \\ 
 & $7.0 \cdot 10^{-7} \, \otimes \, (\bar{\nu}_\mu \rightarrow \bar{\nu}_x)_\mathrm{NC}$ & $7.0\cdot 10^{-7} \, \otimes \, (\nu_e \rightarrow \nu_x)_\mathrm{NC}$& 0.2 & $10^{-4}$ \\ \hline \hline
\end{tabular}
\end{center}

\subsection*{High Resolution/Low Threshold neutrino factory -- \\{\tt NF\_hR\_lT.glb}}

A variable neutrino factory hybrid detector scenario can be simulated with the file {\tt NF\_hR\_lT.glb}. The basic version was
used within \cite{Huber:2006wb} and follows the neutrino factory scenarios from \cite{Huber:2002mx}. These
references should be cited if the file {\tt NF\_hR\_lT.glb} is used for a scientific publication or a talk.
For calculations that involve {\tt NF\_hR\_lT.glb}, the same additional files as for {\tt NFvar.glb} are
required and the {\sf AEDL}-Variables {\tt emax} and {\tt BASELINE} have to be set. {\tt NF\_hR\_lT.glb} 
implements a lower threshold ($\mathrm{\sim\,1\,GeV}$) at a higher energy resolution
$\mathrm{\sigma_e=15\%\cdot E + 0.085\, MeV}$, where the constant term represents 
the effects from Fermi Motion. The background rejection is energy dependent according to 
$10^{-3}/\mathrm{E^2}$, and matches {\tt NFvar.glb} at higher energies. The following rules are defined within {\tt NF\_hR\_lT.glb}: 
\begin{center}
\begin{tabular}{|l|ll|c|c|}
\hline \hline
\multicolumn{3}{|l|}{Disappearance -- $\mu^+$-stored} & $\sigma_\mathrm{norm}$ & $\sigma_\mathrm{cal}$ \\ \hline
Signal & $0.9 \, \otimes \, (\bar{\nu}_\mu \rightarrow \bar{\nu}_\mu)_\mathrm{CC}$ & $0.9 \, \otimes \, (\nu_e \rightarrow \nu_\mu)_\mathrm{CC}$& 0.025 & $10^{-4}$ \\
 & & & & \\
Background & $1.0 \, \otimes \, (\bar{\nu}_\mu \rightarrow \bar{\nu}_x)_\mathrm{NC}$ & & 0.2 & $10^{-4}$ \\ \hline \hline
\multicolumn{3}{|l|}{Appearance -- $\mu^+$-stored} & & \\ \hline
Signal &  $0.5 \, \otimes \, (\nu_e \rightarrow \nu_\mu)_\mathrm{CC}$ & & 0.025 & $10^{-4}$ \\
 & & & & \\
Background &  $1.0 \, \otimes \, (\bar{\nu}_\mu \rightarrow \bar{\nu}_x)_\mathrm{NC}$ & (energy dep. rejection)
& 0.2 & $10^{-4}$\\
 &  $1.0 \, \otimes \, (\bar{\nu}_\mu \rightarrow\bar{\nu}_\mu)_\mathrm{CC}$ & (energy dep. rejection)
&0.2& $10^{-4}$\\ \hline \hline
\multicolumn{3}{|l|}{Disappearance -- $\mu^-$-stored} & & \\ \hline
Signal &  $0.9 \, \otimes \, (\nu_\mu \rightarrow \nu_\mu)_\mathrm{CC}$ & $0.9 \, \otimes \, (\bar{\nu}_e
\rightarrow \bar{\nu}_\mu)_\mathrm{CC}$& 0.025& $10^{-4}$\\
 & & & & \\
Background &  $1.0\cdot 10^{-5} \, \otimes \, (\nu_\mu \rightarrow \nu_x)_\mathrm{NC}$ & & 0.2& $10^{-4}$\\ \hline \hline
\multicolumn{3}{|l|}{Appearance -- $\mu^-$-stored} & & \\ \hline
Signal & $0.5 \, \otimes \, (\bar{\nu}_e \rightarrow \bar{\nu}_\mu)_\mathrm{CC}$  & & 0.025& $10^{-4}$\\
 & & & & \\
Background &  $1.0 \, \otimes \, (\nu_\mu \rightarrow \nu_x)_\mathrm{NC}$ & (energy dep. rejection)
& 0.2 & $10^{-4}$\\
 & $1.0 \, \otimes \, (\nu_\mu \rightarrow
\nu_\mu)_\mathrm{CC}$  & (energy dep. rejection)
&0.2& $10^{-4}$\\ \hline \hline
\end{tabular}
\end{center}









%%%%%%%%%%%%%%%%%%%%%%%%%%%%%%%%%%%%%%%%%%%%%
\chapter{Flux normalization in \GLOBES }
\index{norm}{Normalization of fluxes}
\label{app:flux}

A common issue with \GLOBES\ is confusion about the proper units for
the input flux files for use in \AEDL\ experiment descriptions. Source of
the confusion is an undocumented factor 5.2 with which the fluxes
are multiplied in \GLOBES\ versions older than 3.0 (see below). In Version 3.0 and 
higher, the alternative flux environment {\tt nuflux} is provided, 
which does not contain this factor. The following material is
based on the old environment {\tt flux}. For the use of {\tt nuflux},
replace the factor 5.2 by unity.

\section*{Historical problem}

One problem for the design of \AEDL\ was initially that meaningful
units for flux data strongly depend on the given type of experiment,
but also on relatively arbitrary decisions. For accelerator beams
based on pion decay, one frequently defines the beam luminosity in
{\it protons on target (pot)} since this number has a one-to-one
correspondence with the number of neutrinos produced. Another sensible
unit could be {\it megawatt on target (MW)}, again this number is directly
correlated with the number of neutrinos and moreover there is a unique
relation to {\it pot} for a given accelerator. Of course,
what matters is the integrated luminosity. In some  cases the
neutrino flux is given per $10^7\,\mathrm{s}$. However,
most experiments will run for several years, hence also this number
has to enter somewhere. For neutrino factories the proper number is
useful muon decays per unit time, and for reactor experiments it is
the thermal power of the reactor {\it asf}. This demonstrates that it is
reasonable to keep the flux definition flexible.

\section*{Implementation in \GLOBES }

In understanding how one still can figure out what the correct units
are for each case, it is a good starting point to look at what
\GLOBES\ does with the input files. The cross section in the file is given as
differential cross section divided by energy $x=\sigma/E$, and the flux file
gives $f$. The differential number of events per GeV $n$ as computed in
\GLOBES\ without oscillation and efficiencies is given by
\begin{eqnarray}
n &=& 5.2\times x\times E\times f\times \nonumber \\
&&\mathtt{@norm}\times\mathtt{@power}\times\mathtt{@stored\_muons}\times\mathtt{@time}\times\mathtt{\$target\_mass}\times(\mathtt{\$baseline})^{-2} \nonumber
\end{eqnarray}
Note that $5.2$ is a undocumented fudge factor!

It is the sole responsibility of the author of the \AEDL\ file and its
supporting files, to ensure that the result makes sense. In
principle, it is possible to divide, for example, {\tt @time}  by
$\pi$ and fix that by redefining the flux file by multiplying
it with $\pi$. Modifications like that have happened in the past
and still happen, and many of them are not properly commented.
 
\section*{Writing \AEDL\ files}

The task is to choose the value of  {\tt @norm} such that all the
variables in the \AEDL\ file have the proper units, \eg , {\tt
  @time} has proper unit years.

\GLOBES\ assumes that the cross
section $x$ is given in $10^{-38}\,\mathrm{cm}^2$ and that all fluxes are
given at a distance of $1\,\mathrm{km}$. In addition, it assumes that
the number of target nuclei $\tau$ (or protons or whatever applies to the
given cross section) per unit target mass $m_u$ (which usually is $kt$)
 are properly accounted for.\footnote{Note that the cross sections which are delivered with
  \GLOBES\ always are per nucleon.}  Assuming that in the flux file
the data is given as number of neutrinos per unit area $A$ and
energy bin of width $\Delta E$ at a distance $L$ from the source, one
obtains
\begin{eqnarray}
\mathtt{@norm}=\frac{1}{5.2}\left(\frac{\mathrm{GeV}}{\Delta
      E}\right)
\left(\frac{\mathrm{cm}^2}{A}\right)\left(\frac{L}{\mathrm{km}}\right)^2\left(\frac{\tau}{m_u}\right)\times10^{-38}\times\left(\frac{\mathcal{L}_u}{\mathcal{L}}\right)
\end{eqnarray}
where $\mathcal{L}$ absorbs all factors in the flux file related to
the integrated luminosity, and $\mathcal{L}_u$ is the unit chosen for
it.  The concept of integrated luminosity is
nicely described in the \GLOBES\ manual in \Sec~\ref{sec:source}.
A little example illustrates this concept:
The flux is given for $10^{21}\, \mathrm{pot}\,\mathrm{y}^{-1}$ of
$10\,\mathrm{GeV}$ protons, thus a good choice for the units $\mathcal{L}_u$
is $\mathrm{MW}\,\mathrm{y}^{-1}$, which means that
$\mathcal{L}/\mathcal{L}_u$ is given by (assuming a $10^7\,\mathrm{s}$
year)
\begin{equation}
\frac{\mathcal{L}}{\mathcal{L}_u}=\frac{10\,\mathrm{GeV}\,10^{21}\,\mathrm{pot}\,\mathrm{y}}{10^7\,\mathrm{s}}\times(\mathrm{MW}\,\mathrm{y}^{-1})^{-1}=0.16\ldots
\end{equation}

\section*{Moving from {\tt flux} to {\tt nuflux}}

In order to change the older {\tt flux} environment to the
new {\tt nuflux} (\GLOBES\ 3.0 and higher), replace all user-defined fluxes, such as
\begin{quote}
{\tt flux(\#user)<}\\
{\tt \tb @flux\_file = "user\_file\_1.dat"\\
\tb @time = 2.0\\
\tb @power = 4.0\\
\tb @norm = 1e+8}\\
{\tt >}
\end{quote}
by
\begin{quote}
{\tt FF=5.1989} \\
\\
{\tt nuflux(\#user)<}\\
{\tt \tb @flux\_file = "user\_file\_1.dat"\\
\tb @time = 2.0\\
\tb @power = 4.0\\
\tb @norm = FF*1e+8}\\
{\tt >}
\end{quote}

This replacement is not necessary for neutrino factory built-in fluxes,
and built-in beta beam fluxes were not supported by earlier versions
of \GLOBES .



%%%%%%%%%%%%%%%%%%%%%%%%%%%%%%%%%%%%%%%%%%%%

{\footnotesize
\input{gpl}

\input{fdl}
}


%\end{appendix}

%%% Local Variables: 
%%% mode: latex
%%% TeX-master: Manual.tex
%%% TeX-master: "Manual"
%%% End:
